%==============================================================================
% PREAMBLE: LaTeX Synthesis Project
% Project: Unified Field Theory - Aether, Genesis, Pais Frameworks
% Date: 2025-10-21
% Purpose: Package declarations, macros, custom commands
%==============================================================================

%------------------------------------------------------------------------------
% ENCODING AND FONTS
%------------------------------------------------------------------------------
\usepackage[utf8]{inputenc}
\usepackage[T1]{fontenc}
\usepackage{lmodern}

%------------------------------------------------------------------------------
% MATHEMATICS
%------------------------------------------------------------------------------
\usepackage{amsmath}
\usepackage{amssymb}
\usepackage{amsthm}
\usepackage{mathtools}
\usepackage{cancel}       % Strike-through in equations
\usepackage{physics}      % Dirac notation, derivatives

%------------------------------------------------------------------------------
% COLORS (must come before graphics)
%------------------------------------------------------------------------------
\usepackage{xcolor}
% PhysicsForge standard colors
\definecolor{scalarblue}{RGB}{0,102,204}      % Scalar-vacuum sector
\definecolor{symmgreen}{RGB}{34,139,34}       % Exceptional symmetries
\definecolor{gemorange}{RGB}{255,140,0}       % Gravitoelectromagnetism
\definecolor{forgepurple}{RGB}{128,0,128}     % PhysicsForge unified
% Legacy color names (backward compatibility)
\definecolor{aetherblue}{RGB}{0,102,204}
\definecolor{genesisgreen}{RGB}{34,139,34}
\definecolor{paisred}{RGB}{204,0,0}
\definecolor{unifiedpurple}{RGB}{128,0,128}

%------------------------------------------------------------------------------
% GRAPHICS AND FIGURES
%------------------------------------------------------------------------------
\usepackage{graphicx}
\usepackage{tikz}
\usepackage{pgfplots}
\pgfplotsset{compat=1.18}
\usepgfplotslibrary{fillbetween}  % For contour prepared functionality
\usepgfplotslibrary{colormaps}    % For enhanced colormap support
\usetikzlibrary{arrows.meta,positioning,calc,decorations.pathmorphing,shapes.geometric,cd}
\usepackage{circuitikz}  % For circuit diagrams
\usepackage{tikz-cd}     % For commutative diagrams

%------------------------------------------------------------------------------
% TABLES
%------------------------------------------------------------------------------
\usepackage{array}
\usepackage{booktabs}
\usepackage{longtable}
\usepackage{multirow}

%------------------------------------------------------------------------------
% UNITS AND FORMATTING
%------------------------------------------------------------------------------
\usepackage{siunitx}
\sisetup{
  separate-uncertainty = true,
  multi-part-units = single
}

%------------------------------------------------------------------------------
% CROSS-REFERENCING AND HYPERLINKS
%------------------------------------------------------------------------------
% Neutralize stray zero-width BOM/ZWSP if they sneak into sources
% Note: \DeclareUnicodeCharacter only works with pdflatex; LuaLaTeX and XeLaTeX handle Unicode natively
\usepackage{iftex}
\ifPDFTeX
  \DeclareUnicodeCharacter{FEFF}{}
  \DeclareUnicodeCharacter{200B}{}
\fi
\usepackage[pdfencoding=auto,bookmarks=false]{hyperref}
\hypersetup{
  colorlinks=true,
  linkcolor=blue,
  citecolor=blue,
  urlcolor=blue,
  pdftitle={Unified Field Theory Synthesis},
  pdfauthor={Research Team},
  pdfsubject={Theoretical Physics},
  pdfkeywords={E8, Aether, Genesis, Pais, Unified Theory}
}
\usepackage[capitalize,noabbrev]{cleveref}
% Workaround: neutralize hyperref internal \@@BOOKMARK to avoid runaway argument from stray BOM remnants.
\makeatletter
\def\@@BOOKMARK#1#2#3#4#5{}% drop bookmark writes safely (bookmarks already disabled)
\makeatother

%------------------------------------------------------------------------------
% BIBLIOGRAPHY
%------------------------------------------------------------------------------
\usepackage{natbib}

%------------------------------------------------------------------------------
% INDEX
%------------------------------------------------------------------------------
\usepackage{makeidx}
% Use \providecommand to avoid conflicts with hyperref's internal definition
\providecommand{\hyperxindexformat}[2]{}
\makeindex

% Provide abstract environment for book class (tests expect it)
\newenvironment{abstract}{\begin{center}\bfseries Abstract\end{center}\vspace{0.5em}\small}{\par\vspace{1em}}

%------------------------------------------------------------------------------
% CODE LISTINGS (for appendix simulation code)
%------------------------------------------------------------------------------
\usepackage{listings}
\lstset{
  basicstyle=\ttfamily\small,
  keywordstyle=\color{blue},
  commentstyle=\color{green!60!black},
  stringstyle=\color{red},
  showstringspaces=false,
  breaklines=true,
  frame=single
}

%------------------------------------------------------------------------------
% CUSTOM THEOREM ENVIRONMENTS
%------------------------------------------------------------------------------
\theoremstyle{plain}
\newtheorem{theorem}{Theorem}[chapter]
\newtheorem{lemma}[theorem]{Lemma}
\newtheorem{proposition}[theorem]{Proposition}
\newtheorem{corollary}[theorem]{Corollary}

\theoremstyle{definition}
\newtheorem{definition}[theorem]{Definition}
\newtheorem{example}[theorem]{Example}
\newtheorem{exercise}[theorem]{Exercise}

\theoremstyle{remark}
\newtheorem{remark}[theorem]{Remark}
\newtheorem{note}[theorem]{Note}

%------------------------------------------------------------------------------
% CUSTOM MACROS: FRAMEWORK ATTRIBUTION (PhysicsForge Standard Nomenclature)
%------------------------------------------------------------------------------
% NEW STANDARD PHYSICS NOMENCLATURE
\newcommand{\scalarattr}{\textsuperscript{\color{scalarblue}[S]}}        % Scalar-vacuum sector
\newcommand{\symmattr}{\textsuperscript{\color{symmgreen}[X]}}           % eXceptional symmetries
\newcommand{\gemattr}{\textsuperscript{\color{gemorange}[G]}}            % Gravitoelectromagnetism
\newcommand{\physforgeattr}{\textsuperscript{\color{forgepurple}[P]}}    % PhysicsForge unified
\newcommand{\mathgenericattr}{\textsuperscript{[M]}}                     % Mathematical/generic

% Inline framework names (for narrative text)
\newcommand{\scalarsector}{\textcolor{scalarblue}{scalar-vacuum}}
\newcommand{\symmsector}{\textcolor{symmgreen}{symmetry}}
\newcommand{\gemsector}{\textcolor{gemorange}{GEM}}
\newcommand{\physforge}{\textcolor{forgepurple}{PhysicsForge}}

% BACKWARD COMPATIBILITY (temporary during migration)
% Legacy macro names preserved as aliases
\let\aetherattr\scalarattr
\let\genesisattr\symmattr
\let\paisattr\gemattr
\let\unifiedattr\physforgeattr
\let\aether\scalarsector
\let\genesis\symmsector
\let\pais\gemsector
\let\unified\physforge

%------------------------------------------------------------------------------
% CUSTOM MACROS: EQUATION TAGS
%------------------------------------------------------------------------------
% Usage: \eqtag{FRAMEWORK}{DOMAIN}{STATUS}
% FRAMEWORK: A (Aether), G (Genesis), P (Pais), M (Math), U (Unified)
% DOMAIN: QM, GR, EM, MATH, COSMO, EXP
% STATUS: T (Theoretical), E (Experimental), S (Speculative), V (Validated)
% Revised: avoid \tag* duplication of hyperref anchors; append inline text instead.
\newcommand{\eqtag}[3]{\quad\text{\scriptsize\color{gray}[#1:#2:#3]}}

%------------------------------------------------------------------------------
% CUSTOM MACROS: MATHEMATICAL OPERATORS
%------------------------------------------------------------------------------
% \DeclareMathOperator{\Tr}{Tr}
\DeclareMathOperator{\Det}{Det}
\DeclareMathOperator{\Aut}{Aut}
\DeclareMathOperator{\End}{End}
\DeclareMathOperator{\Hom}{Hom}
\DeclareMathOperator{\Sym}{Sym}
\DeclareMathOperator{\Alt}{Alt}
\DeclareMathOperator{\sgn}{sgn}
% \DeclareMathOperator{\rank}{rank}
\DeclareMathOperator{\nullity}{nullity}
\DeclareMathOperator{\diag}{diag}
\DeclareMathOperator{\Span}{span}

%------------------------------------------------------------------------------
% CUSTOM MACROS: SPECIAL SETS AND SPACES
%------------------------------------------------------------------------------
\newcommand{\R}{\mathbb{R}}        % Real numbers
\newcommand{\C}{\mathbb{C}}        % Complex numbers
\newcommand{\Q}{\mathbb{Q}}        % Rational numbers
\newcommand{\Z}{\mathbb{Z}}        % Integers
\newcommand{\N}{\mathbb{N}}        % Natural numbers
\newcommand{\HH}{\mathbb{H}}       % Quaternions
\newcommand{\OO}{\mathbb{O}}       % Octonions
\newcommand{\PP}{\mathbb{P}}       % Projective space or Pathions
\newcommand{\M}{\mathbb{M}}        % Monster Group

%------------------------------------------------------------------------------
% CUSTOM MACROS: PHYSICS NOTATION
%------------------------------------------------------------------------------
% \newcommand{\ket}[1]{\left| #1 \right\rangle}
% \newcommand{\bra}[1]{\left\langle #1 \right|}
% \newcommand{\braket}[2]{\left\langle #1 \middle| #2 \right\rangle}
% \newcommand{\expval}[1]{\left\langle #1 \right\rangle}

% \newcommand{\pdv}[2]{\frac{\partial #1}{\partial #2}}
% \newcommand{\dv}[2]{\frac{d #1}{d #2}}
% \newcommand{\pdvsq}[2]{\frac{\partial^2 #1}{\partial #2^2}}

%------------------------------------------------------------------------------
% CUSTOM MACROS: VECTOR CALCULUS
%------------------------------------------------------------------------------
% \renewcommand{\div}{\nabla \cdot}
% \newcommand{\curl}{\nabla \times}
% \newcommand{\grad}{\nabla}
% \newcommand{\laplacian}{\nabla^2}

%------------------------------------------------------------------------------
% CUSTOM MACROS: SPECIAL FUNCTIONS
%------------------------------------------------------------------------------
% \DeclareMathOperator{\erf}{erf}
% \DeclareMathOperator{\erfc}{erfc}
\DeclareMathOperator{\sinc}{sinc}
\DeclareMathOperator{\Li}{Li}           % Polylogarithm
\DeclareMathOperator{\Ei}{Ei}           % Exponential integral
\newcommand{\ee}{\mathrm{e}}            % Euler's number (exponential base)
\newcommand{\imag}{\mathrm{i}}          % Imaginary unit

%------------------------------------------------------------------------------
% CUSTOM MACROS: FRAMEWORK-SPECIFIC
%------------------------------------------------------------------------------
% Aether framework
\newcommand{\ZPE}{\text{ZPE}}
\newcommand{\phiscalar}{\phi}
\newcommand{\Laether}{\mathcal{L}_{\text{Aether}}}

% Genesis framework
\newcommand{\Kgenesis}{K_{\text{Genesis}}}
\newcommand{\nodespace}{\mathcal{N}}
\newcommand{\Lgenesis}{\mathcal{L}_{\text{Genesis}}}

% Pais framework
\newcommand{\GEM}{\text{GEM}}
\newcommand{\Lpais}{\mathcal{L}_{\text{Pais}}}

% Unified
\newcommand{\Kunified}{K_{\text{unified}}}
\newcommand{\Lunified}{\mathcal{L}_{\text{unified}}}

%------------------------------------------------------------------------------
% PAGE LAYOUT
%------------------------------------------------------------------------------
\usepackage{geometry}
\geometry{
  a4paper,
  left=1.5in,
  right=1.0in,
  top=1.0in,
  bottom=1.0in,
  headheight=25.2232pt
}

%------------------------------------------------------------------------------
% HEADERS AND FOOTERS
%------------------------------------------------------------------------------
\usepackage{fancyhdr}
\pagestyle{fancy}
\fancyhf{}
\fancyhead[LE]{\slshape \leftmark}
\fancyhead[RO]{\slshape \rightmark}
\fancyfoot[C]{\thepage}
\renewcommand{\headrulewidth}{0.4pt}
\renewcommand{\footrulewidth}{0pt}

%------------------------------------------------------------------------------
% CHAPTER AND SECTION FORMATTING
%------------------------------------------------------------------------------
\usepackage{titlesec}
\titleformat{\chapter}[display]
  {\normalfont\huge\bfseries}{\chaptertitlename\ \thechapter}{20pt}{\Huge}
\titleformat{\section}
  {\normalfont\Large\bfseries}{\thesection}{1em}{}
\titleformat{\subsection}
  {\normalfont\large\bfseries}{\thesubsection}{1em}{}

%------------------------------------------------------------------------------
% MISC PACKAGES
%------------------------------------------------------------------------------
\usepackage{tcolorbox}     % Colored boxes for examples and highlights
\usepackage{lipsum}        % For placeholder text during development
\usepackage{enumitem}      % Better control over lists
\usepackage{caption}       % Better caption formatting
\usepackage{subcaption}    % Subfigures

%------------------------------------------------------------------------------
% TOC DEPTH
%------------------------------------------------------------------------------
\setcounter{tocdepth}{2}     % Include subsections in TOC
\setcounter{secnumdepth}{3}  % Number subsubsections

%------------------------------------------------------------------------------
% SHADED BOX ENVIRONMENT
%------------------------------------------------------------------------------
\usepackage{framed}
% xcolor already loaded above
\definecolor{shadecolor}{gray}{0.95}
\newenvironment{shadedbox}%
  {\begin{shaded}\begin{quote}}%
  {\end{quote}\end{shaded}}%

%==============================================================================
% END PREAMBLE
%==============================================================================
