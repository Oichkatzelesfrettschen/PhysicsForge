% app_c_units_constants.tex
% Unit systems and physical constants reference
% Author: Computernonymouse

\chapter{Unit Systems and Physical Constants}
\label{app:units-constants}

This appendix provides a comprehensive reference for unit systems and physical constants used throughout this monograph. We include conversion factors between different unit systems and the latest CODATA values for fundamental constants.

\section{Natural Unit Systems}

\subsection{Planck Units}

In Planck units, fundamental constants are normalized to unity:
\begin{align}
    c &= 1 \quad \text{(speed of light)} \\
    \hbar &= 1 \quad \text{(reduced Planck constant)} \\
    G &= 1 \quad \text{(gravitational constant)} \\
    k_B &= 1 \quad \text{(Boltzmann constant)} \\
    k_e &= 1 \quad \text{(Coulomb constant)}
\end{align}

Derived Planck scales:
\begin{align}
    \ell_P &= \sqrt{\frac{\hbar G}{c^3}} = 1.616255 \times 10^{-35} \text{ m} \\
    t_P &= \sqrt{\frac{\hbar G}{c^5}} = 5.391247 \times 10^{-44} \text{ s} \\
    m_P &= \sqrt{\frac{\hbar c}{G}} = 2.176434 \times 10^{-8} \text{ kg} \\
    E_P &= \sqrt{\frac{\hbar c^5}{G}} = 1.956081 \times 10^{9} \text{ J} \\
    T_P &= \sqrt{\frac{\hbar c^5}{G k_B^2}} = 1.416784 \times 10^{32} \text{ K}
\end{align}

\subsection{Particle Physics Units}

In particle physics, we typically set $c = \hbar = 1$, measuring:
\begin{itemize}
    \item Mass in GeV: $1 \text{ GeV} = 1.78266192 \times 10^{-27}$ kg
    \item Length in GeV$^{-1}$: $1 \text{ GeV}^{-1} = 1.97326980 \times 10^{-16}$ m
    \item Time in GeV$^{-1}$: $1 \text{ GeV}^{-1} = 6.58211957 \times 10^{-25}$ s
\end{itemize}

Useful conversion: $\hbar c = 197.3269804$ MeV$\cdot$fm

\subsection{Geometrized Units}

In general relativity, setting $c = G = 1$:
\begin{itemize}
    \item Length = Time = $[L]$
    \item Mass = Energy = $[L]$
    \item Angular momentum = $[L^2]$
    \item Charge = $[L]$ (if $4\pi\epsilon_0 = 1$)
\end{itemize}

Solar mass in geometrized units: $M_\odot = 1.47668$ km

\section{Electromagnetic Unit Systems}

\subsection{SI (International System)}

Maxwell's equations in SI:
\begin{align}
    \nabla \cdot \vec{E} &= \frac{\rho}{\epsilon_0} \\
    \nabla \cdot \vec{B} &= 0 \\
    \nabla \times \vec{E} &= -\frac{\partial \vec{B}}{\partial t} \\
    \nabla \times \vec{B} &= \mu_0 \vec{J} + \mu_0 \epsilon_0 \frac{\partial \vec{E}}{\partial t}
\end{align}

Key constants:
\begin{align}
    \epsilon_0 &= 8.8541878128 \times 10^{-12} \text{ F/m} \\
    \mu_0 &= 1.25663706212 \times 10^{-6} \text{ H/m} \\
    c &= \frac{1}{\sqrt{\mu_0 \epsilon_0}}
\end{align}

\subsection{Gaussian-CGS Units}

Maxwell's equations in Gaussian units:
\begin{align}
    \nabla \cdot \vec{E} &= 4\pi\rho \\
    \nabla \cdot \vec{B} &= 0 \\
    \nabla \times \vec{E} &= -\frac{1}{c}\frac{\partial \vec{B}}{\partial t} \\
    \nabla \times \vec{B} &= \frac{4\pi}{c}\vec{J} + \frac{1}{c}\frac{\partial \vec{E}}{\partial t}
\end{align}

Conversion factors:
\begin{itemize}
    \item Electric field: $E_{\text{Gaussian}} = \sqrt{4\pi\epsilon_0} \, E_{\text{SI}} \approx \frac{E_{\text{SI}}}{2.998 \times 10^4}$
    \item Magnetic field: $B_{\text{Gaussian}} = \sqrt{4\pi/\mu_0} \, B_{\text{SI}} \approx 10^4 \, B_{\text{SI}}$
    \item Charge: $q_{\text{Gaussian}} = q_{\text{SI}}/\sqrt{4\pi\epsilon_0} \approx 2.998 \times 10^9 \, q_{\text{SI}}$
\end{itemize}

\section{Fundamental Constants (CODATA 2022)}

\subsection{Defining Constants (Exact Values)}

\begin{table}[h]
\centering
\begin{tabular}{lll}
\hline
\textbf{Quantity} & \textbf{Symbol} & \textbf{Value} \\
\hline
Speed of light in vacuum & $c$ & $299\,792\,458$ m/s \\
Planck constant & $h$ & $6.626\,070\,15 \times 10^{-34}$ J$\cdot$s \\
Elementary charge & $e$ & $1.602\,176\,634 \times 10^{-19}$ C \\
Boltzmann constant & $k_B$ & $1.380\,649 \times 10^{-23}$ J/K \\
Avogadro constant & $N_A$ & $6.022\,140\,76 \times 10^{23}$ mol$^{-1}$ \\
\hline
\end{tabular}
\end{table}

\subsection{Measured Constants}

\begin{table}[h]
\centering
\begin{tabular}{llll}
\hline
\textbf{Quantity} & \textbf{Symbol} & \textbf{Value} & \textbf{Uncertainty} \\
\hline
Gravitational constant & $G$ & $6.674\,30(15) \times 10^{-11}$ & m$^3$ kg$^{-1}$ s$^{-2}$ \\
Fine structure constant & $\alpha$ & $7.297\,352\,5693(11) \times 10^{-3}$ & -- \\
Electron mass & $m_e$ & $9.109\,383\,7015(28) \times 10^{-31}$ & kg \\
Proton mass & $m_p$ & $1.672\,621\,923\,69(51) \times 10^{-27}$ & kg \\
Neutron mass & $m_n$ & $1.674\,927\,498\,04(95) \times 10^{-27}$ & kg \\
Rydberg constant & $R_\infty$ & $10\,973\,731.568\,160(21)$ & m$^{-1}$ \\
Bohr radius & $a_0$ & $5.291\,772\,109\,03(80) \times 10^{-11}$ & m \\
Classical electron radius & $r_e$ & $2.817\,940\,3262(13) \times 10^{-15}$ & m \\
Compton wavelength & $\lambda_C$ & $2.426\,310\,238\,67(73) \times 10^{-12}$ & m \\
\hline
\end{tabular}
\end{table}

\subsection{Cosmological Parameters}

\begin{table}[h]
\centering
\begin{tabular}{llll}
\hline
\textbf{Quantity} & \textbf{Symbol} & \textbf{Value} & \textbf{Source} \\
\hline
Hubble constant & $H_0$ & $67.4 \pm 0.5$ km/s/Mpc & Planck 2020 \\
Dark energy density & $\Omega_\Lambda$ & $0.6847 \pm 0.0073$ & Planck 2020 \\
Matter density & $\Omega_m$ & $0.3153 \pm 0.0073$ & Planck 2020 \\
Baryon density & $\Omega_b$ & $0.0493 \pm 0.0003$ & Planck 2020 \\
CMB temperature & $T_0$ & $2.72548 \pm 0.00057$ K & FIRAS \\
Age of universe & $t_0$ & $13.797 \pm 0.023$ Gyr & Planck 2020 \\
\hline
\end{tabular}
\end{table}

\section{Conversion Factors}

\subsection{Energy Conversions}

\begin{align}
    1 \text{ eV} &= 1.602176634 \times 10^{-19} \text{ J} \\
    1 \text{ erg} &= 10^{-7} \text{ J} \\
    1 \text{ cal} &= 4.184 \text{ J} \\
    1 \text{ kWh} &= 3.6 \times 10^6 \text{ J} \\
    1 \text{ BTU} &= 1055.06 \text{ J} \\
    1 \text{ ton TNT} &\approx 4.184 \times 10^9 \text{ J}
\end{align}

\subsection{Length Conversions}

\begin{align}
    1 \text{ fm} &= 10^{-15} \text{ m} \\
    1 \text{ \AA} &= 10^{-10} \text{ m} \\
    1 \text{ nm} &= 10^{-9} \text{ m} \\
    1 \text{ AU} &= 1.495978707 \times 10^{11} \text{ m} \\
    1 \text{ ly} &= 9.4607304725808 \times 10^{15} \text{ m} \\
    1 \text{ pc} &= 3.0856775814913673 \times 10^{16} \text{ m}
\end{align}

\subsection{Mass-Energy Equivalence}

Key particle masses in energy units:
\begin{align}
    m_e c^2 &= 0.51099895000(15) \text{ MeV} \\
    m_p c^2 &= 938.27208816(29) \text{ MeV} \\
    m_n c^2 &= 939.56542052(54) \text{ MeV} \\
    m_W c^2 &= 80.379(12) \text{ GeV} \\
    m_Z c^2 &= 91.1876(21) \text{ GeV} \\
    m_H c^2 &= 125.25(17) \text{ GeV}
\end{align}

\section{Dimensional Analysis Examples}

\subsection{Casimir Force}

Dimensional analysis for parallel plates:
\begin{equation}
    F \sim \frac{\hbar c}{a^4} \times \text{Area}
\end{equation}

Exact result: $F = -\frac{\pi^2 \hbar c}{240 a^4} \times \text{Area}$

\subsection{Hawking Temperature}

Black hole temperature from dimensional analysis:
\begin{equation}
    k_B T \sim \frac{\hbar c^3}{G M}
\end{equation}

Exact result: $k_B T = \frac{\hbar c^3}{8\pi G M}$

\subsection{Quantum Chromodynamics Scale}

QCD coupling runs with energy scale $\mu$:
\begin{equation}
    \alpha_s(\mu) = \frac{g_s^2(\mu)}{4\pi} = \frac{4\pi}{\beta_0 \ln(\mu^2/\Lambda_{QCD}^2)}
\end{equation}

where $\Lambda_{QCD} \approx 217$ MeV and $\beta_0 = 11 - 2n_f/3$.

\section{Astrophysical Scales}

\subsection{Stellar Parameters}

\begin{itemize}
    \item Solar mass: $M_\odot = 1.98892 \times 10^{30}$ kg
    \item Solar radius: $R_\odot = 6.96 \times 10^8$ m
    \item Solar luminosity: $L_\odot = 3.828 \times 10^{26}$ W
    \item Earth mass: $M_\oplus = 5.97237 \times 10^{24}$ kg
    \item Earth radius: $R_\oplus = 6.3781 \times 10^6$ m
\end{itemize}

\subsection{Critical Densities}

\begin{itemize}
    \item Nuclear density: $\rho_{nuclear} \approx 2.3 \times 10^{17}$ kg/m$^3$
    \item White dwarf limit: $M_{Ch} = 1.4 M_\odot$ (Chandrasekhar mass)
    \item Neutron star density: $\rho_{NS} \sim 10^{18}$ kg/m$^3$
    \item Critical density of universe: $\rho_c = \frac{3H_0^2}{8\pi G} = 8.5 \times 10^{-27}$ kg/m$^3$
\end{itemize}