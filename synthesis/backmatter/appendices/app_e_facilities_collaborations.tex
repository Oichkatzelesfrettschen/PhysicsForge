% app_e_facilities_collaborations.tex
% Experimental facilities and international collaborations
% Author: Computernonymouse

\chapter{Experimental Facilities and Collaborations}
\label{app:facilities}

This appendix provides detailed information about the experimental facilities, research institutions, and international collaborations relevant to testing and developing the theoretical frameworks presented in this monograph. We include current capabilities, planned upgrades, and contact information for key research groups.

\section{Particle Collider Facilities}

\subsection{Large Hadron Collider (LHC) - CERN}

\textbf{Location}: Geneva, Switzerland/France border\\
\textbf{Circumference}: 26.7 km\\
\textbf{Maximum energy}: 13.6 TeV (Run 3, 2022-2025)\\
\textbf{Luminosity}: $2 \times 10^{34}$ cm$^{-2}$s$^{-1}$

\subsubsection{Main Experiments}

\begin{itemize}
    \item \textbf{ATLAS (A Toroidal LHC ApparatuS)}
    \begin{itemize}
        \item General-purpose detector
        \item Dimensions: 46m long, 25m diameter
        \item Weight: 7,000 tonnes
        \item Key for: Extra dimensions, microscopic black holes
    \end{itemize}

    \item \textbf{CMS (Compact Muon Solenoid)}
    \begin{itemize}
        \item General-purpose detector
        \item Dimensions: 21m long, 15m diameter
        \item Weight: 14,000 tonnes
        \item Superconducting solenoid: 4 Tesla
    \end{itemize}

    \item \textbf{LHCb (Large Hadron Collider beauty)}
    \begin{itemize}
        \item B-physics and CP violation
        \item Forward spectrometer design
        \item Precision: $\Delta m/m \sim 10^{-4}$
    \end{itemize}

    \item \textbf{ALICE (A Large Ion Collider Experiment)}
    \begin{itemize}
        \item Heavy-ion collisions
        \item Quark-gluon plasma studies
        \item Temperature reach: $> 10^{12}$ K
    \end{itemize}
\end{itemize}

\subsubsection{High-Luminosity LHC (HL-LHC) Upgrade}

\textbf{Timeline}: 2029-2041\\
\textbf{Target luminosity}: $5-7 \times 10^{34}$ cm$^{-2}$s$^{-1}$\\
\textbf{Integrated luminosity}: 3000 fb$^{-1}$\\
\textbf{Key technologies}: Nb$_3$Sn superconducting magnets, crab cavities

\subsection{Future Circular Collider (FCC)}

\textbf{Status}: Design phase\\
\textbf{Circumference}: 91 km\\
\textbf{Timeline}: 2040s (FCC-ee), 2050s (FCC-hh)

\subsubsection{FCC-ee (Electron-Positron)}
\begin{itemize}
    \item Energy range: 91-365 GeV
    \item Precision measurements: Z, W, Higgs, top
    \item Luminosity: $2.3 \times 10^{36}$ cm$^{-2}$s$^{-1}$ at Z pole
\end{itemize}

\subsubsection{FCC-hh (Hadron-Hadron)}
\begin{itemize}
    \item Center-of-mass energy: 100 TeV
    \item Discovery reach: up to 30 TeV particles
    \item Integrated luminosity goal: 30 ab$^{-1}$
\end{itemize}

\subsection{International Linear Collider (ILC)}

\textbf{Status}: Technical design complete, awaiting approval\\
\textbf{Location}: Proposed for Japan (Kitakami mountains)\\
\textbf{Length}: 31 km (initial), expandable to 50 km\\
\textbf{Energy}: 250 GeV (initial), upgradeable to 1 TeV

Key advantages for framework testing:
\begin{itemize}
    \item Clean $e^+e^-$ collisions
    \item Precision Higgs measurements
    \item Polarized beams (80\% $e^-$, 30\% $e^+$)
    \item Low background for exotic physics
\end{itemize}

\section{Gravitational Wave Observatories}

\subsection{LIGO (Laser Interferometer Gravitational-Wave Observatory)}

\textbf{Locations}:
\begin{itemize}
    \item Hanford, Washington (LHO)
    \item Livingston, Louisiana (LLO)
\end{itemize}

\textbf{Arm length}: 4 km\\
\textbf{Strain sensitivity}: $10^{-23}$ Hz$^{-1/2}$ at 100 Hz\\
\textbf{Frequency range}: 10 Hz - 5 kHz

\subsubsection{Advanced LIGO Plus (A+)}
\textbf{Timeline}: 2024-2027\\
\textbf{Improvements}:
\begin{itemize}
    \item Frequency-dependent squeezing
    \item New mirror coatings (reduced thermal noise)
    \item 50\% sensitivity improvement
    \item Detection rate: 1 per day (binary neutron stars)
\end{itemize}

\subsection{Virgo}

\textbf{Location}: Cascina, Italy (EGO)\\
\textbf{Arm length}: 3 km\\
\textbf{Partners}: CNRS (France), INFN (Italy)

\subsection{KAGRA (Kamioka Gravitational Wave Detector)}

\textbf{Location}: Kamioka mine, Japan\\
\textbf{Unique features}:
\begin{itemize}
    \item Underground location (200m depth)
    \item Cryogenic mirrors (20 K)
    \item Reduced seismic noise
\end{itemize}

\subsection{LISA (Laser Interferometer Space Antenna)}

\textbf{Status}: Approved, launch 2037\\
\textbf{Configuration}: 3 spacecraft, triangular formation\\
\textbf{Arm length}: 2.5 million km\\
\textbf{Frequency range}: 0.1 mHz - 1 Hz

Target sources:
\begin{itemize}
    \item Supermassive black hole mergers
    \item Extreme mass ratio inspirals
    \item Galactic binaries
    \item Cosmological backgrounds
\end{itemize}

\subsection{Cosmic Explorer (Next Generation)}

\textbf{Status}: Conceptual design\\
\textbf{Timeline}: 2035-2040\\
\textbf{Arm length}: 40 km\\
\textbf{Sensitivity}: 10$\times$ better than Advanced LIGO

\section{Casimir Force Experiments}

\subsection{Yale University - Lamoreaux Group}

\textbf{PI}: Prof. Steve Lamoreaux\\
\textbf{Location}: New Haven, Connecticut, USA\\
\textbf{Capabilities}:
\begin{itemize}
    \item Torsion pendulum apparatus
    \item Force sensitivity: $10^{-14}$ N
    \item Gap range: 0.5-10 $\mu$m
    \item Temperature control: mK stability
\end{itemize}

Recent achievements:
\begin{itemize}
    \item 1\% precision Casimir force measurements
    \item Thermal Casimir effect detection
    \item Patch potential mapping
\end{itemize}

\subsection{Purdue University - Decca Group}

\textbf{PI}: Prof. Ricardo Decca\\
\textbf{Location}: West Lafayette, Indiana, USA\\
\textbf{Specialization}: Precision Casimir measurements with MEMS

Key apparatus:
\begin{itemize}
    \item Microelectromechanical torsional oscillator
    \item In-situ electrostatic calibration
    \item Sub-nm distance control
    \item Gradient force detection
\end{itemize}

\subsection{TU Delft - Quantum Nanomechanics}

\textbf{PI}: Prof. Simon Groeblacher\\
\textbf{Location}: Delft, Netherlands\\
\textbf{Focus}: Quantum optomechanics and Casimir forces

Unique capabilities:
\begin{itemize}
    \item Photonic crystal cavities
    \item Single-photon strong coupling
    \item Cryogenic operation (mK temperatures)
    \item Suspended nanostructures
\end{itemize}

\subsection{IMDEA Nanoscience - Casimir Lab}

\textbf{Location}: Madrid, Spain\\
\textbf{Capabilities}:
\begin{itemize}
    \item Dynamic Casimir effect studies
    \item Metamaterial Casimir forces
    \item Repulsive Casimir configurations
\end{itemize}

\section{Time Crystal Research}

\subsection{Google Quantum AI}

\textbf{Location}: Santa Barbara, California, USA\\
\textbf{Platform}: Sycamore quantum processor\\
\textbf{Qubits}: 70 superconducting qubits

Achievements:
\begin{itemize}
    \item First discrete time crystal (2021)
    \item 100 cycles of periodic evolution
    \item Error mitigation protocols
    \item Open-source Cirq framework
\end{itemize}

\subsection{IBM Quantum Network}

\textbf{Headquarters}: Yorktown Heights, New York, USA\\
\textbf{Cloud access}: 20+ quantum systems\\
\textbf{Largest system}: 433-qubit Osprey

Time crystal experiments:
\begin{itemize}
    \item Floquet many-body localization
    \item Prethermal time crystals
    \item Dissipative time crystals
\end{itemize}

\subsection{University of Maryland - JQI}

\textbf{PI}: Prof. Christopher Monroe\\
\textbf{Location}: College Park, Maryland, USA\\
\textbf{Platform}: Trapped ion quantum computer

Capabilities:
\begin{itemize}
    \item 32 trapped Yb$^+$ ions
    \item All-to-all connectivity
    \item Individual addressing
    \item 99.9\% two-qubit gate fidelity
\end{itemize}

\subsection{QuTech - TU Delft/TNO}

\textbf{Location}: Delft, Netherlands\\
\textbf{Focus}: Topological quantum computing\\
\textbf{Platform}: Spin qubits in silicon

Research directions:
\begin{itemize}
    \item Continuous time crystals
    \item Topological time crystals
    \item Quantum error correction
\end{itemize}

\section{Analogue Gravity Systems}

\subsection{Technion - Steinhauer Lab}

\textbf{PI}: Prof. Jeff Steinhauer\\
\textbf{Location}: Haifa, Israel\\
\textbf{System}: Bose-Einstein condensates

Achievements:
\begin{itemize}
    \item Hawking radiation observation (2016)
    \item Stimulated Hawking radiation (2019)
    \item Temperature measurement of Hawking radiation
    \item Entanglement verification
\end{itemize}

\subsection{University of British Columbia}

\textbf{PI}: Prof. Silke Weinfurtner\\
\textbf{Location}: Vancouver, Canada\\
\textbf{System}: Water wave tank analogue

Capabilities:
\begin{itemize}
    \item 3m $\times$ 1.5m wave flume
    \item Draining vortex black holes
    \item White hole-black hole pairs
    \item Superradiance measurements
\end{itemize}

\subsection{University of Nottingham}

\textbf{Location}: Nottingham, UK\\
\textbf{System}: Optical analogues in nonlinear media

Research focus:
\begin{itemize}
    \item Fiber-optical event horizons
    \item Rogue wave generation
    \item Optical black hole lasers
    \item Hawking radiation signatures
\end{itemize}

\section{International Collaborations}

\subsection{Event Horizon Telescope (EHT)}

\textbf{Type}: VLBI network\\
\textbf{Telescopes}: 11 facilities worldwide\\
\textbf{Baseline}: Earth diameter\\
\textbf{Resolution}: 20 microarcseconds

Member institutions:
\begin{itemize}
    \item MIT Haystack Observatory
    \item Max Planck Institute for Radio Astronomy
    \item ALMA (Atacama Large Millimeter Array)
    \item South Pole Telescope
    \item James Clerk Maxwell Telescope
\end{itemize}

\subsection{COST Action: Quantum Gravity Phenomenology}

\textbf{Code}: CA18108\\
\textbf{Duration}: 2019-2023\\
\textbf{Countries}: 31 participating nations\\
\textbf{Working groups}:
\begin{itemize}
    \item WG1: Quantum gravity theories
    \item WG2: Phenomenology
    \item WG3: Experimental tests
    \item WG4: Cosmological observations
\end{itemize}

\subsection{Quantum Flagship}

\textbf{Funding}: EU Horizon Europe\\
\textbf{Budget}: €1 billion (2018-2028)\\
\textbf{Projects relevant to frameworks}:
\begin{itemize}
    \item ASTERIQS: Diamond quantum sensors
    \item macQsimal: Miniature atomic vapor cells
    \item iqClock: Optical lattice clocks
    \item UNIQORN: Quantum communication systems
\end{itemize}

\section{Contact Information and Resources}

\subsection{Key Conferences}

\begin{itemize}
    \item \textbf{Marcel Grossmann Meeting}: Triennial, general relativity
    \item \textbf{COSMO}: Annual, particle cosmology
    \item \textbf{Loops}: Biennial, quantum gravity
    \item \textbf{GR23}: 2025, Beijing (general relativity)
\end{itemize}

\subsection{Online Resources}

\begin{itemize}
    \item \textbf{arXiv.org}: Preprint server (gr-qc, hep-th, quant-ph)
    \item \textbf{INSPIRE-HEP}: High energy physics database
    \item \textbf{GraceDB}: Gravitational wave candidate event database
    \item \textbf{Quantum Computing Report}: Industry updates
\end{itemize}