\chapter{Energy Technologies}
\label{ch:energy_technologies}

\section{Scalar-ZPE Energy Harvesting}
\label{sec:scalar_zpe_energy_harvesting}

The theoretical basis for Scalar-ZPE energy harvesting is rooted in the Aether framework, which posits that spacetime is not a void but a dynamic, energetic medium. The framework proposes that the zero-point energy (ZPE) of the vacuum can be accessed and harvested through the mediation of scalar fields.

The coupling between scalar fields and ZPE is the cornerstone of this concept. The scalar field, denoted by $\phi$, is a field that has a single value at every point in spacetime. It is proposed that this field can be manipulated to create localized regions of negative energy density, which in turn allows for the extraction of energy from the vacuum.

The energy extraction principle is based on the idea of creating a potential difference between the vacuum and a localized region of spacetime. By manipulating the scalar field, it is possible to lower the energy density of a region of spacetime, creating a "sink" into which energy from the surrounding vacuum can flow.

The following equations from the Aether framework are central to this concept:

\begin{equation}
  S = \frac{A}{4G\hbar} + \int ZPE(t) d^3x
  \label{eq:ae084}
\end{equation}

\begin{equation}
  \rho_{exotic} = -\frac{E_{ZPE}}{V_{eff}}
  \label{eq:ae140}
\end{equation}

\begin{equation}
  P = \Delta E_{foam}^2
  \label{eq:ae170}
\end{equation}

\section{Resonant Cavity Designs}
\label{sec:resonant_cavity_designs}

Resonant cavities are a key component of many proposed ZPE energy harvesting devices. These cavities are designed to create a localized region of spacetime where the energy density can be manipulated. The geometry of the cavity is critical to its performance.

The most common geometric configurations for resonant cavities are spherical, cylindrical, and fractal. Each of these geometries has its own advantages and disadvantages. Spherical cavities are the simplest to analyze, but they are also the most difficult to construct. Cylindrical cavities are easier to construct, but they are less efficient than spherical cavities. Fractal cavities are the most complex, but they offer the potential for the highest efficiency.

The resonance frequency of a cavity is the frequency at which it is most efficient at extracting energy from the vacuum. The resonance frequency is determined by the geometry of the cavity and the properties of the materials from which it is constructed.

The quality factor (Q) of a cavity is a measure of its ability to store energy. A high Q factor is desirable for ZPE energy harvesting devices, as it allows for the accumulation of energy over time.

The following equations are used to calculate the resonance frequency and quality factor of a resonant cavity:

% TODO: Add dimensional resonance formulas

The following figures show some common resonant cavity designs:

% TODO: Add TikZ diagrams of cavity geometries


\section{Fractal-Based Harvesters}
\label{sec:fractal_based_harvesters}

Fractal-based harvesters are a promising new technology for ZPE energy harvesting. These devices are based on the principles of fractal antenna theory, which has been shown to be effective for collecting energy from a wide range of frequencies.

The key advantage of fractal-based harvesters is their ability to collect energy from multiple scales simultaneously. This is because fractals are self-similar, meaning that they have the same basic shape at all scales. This allows them to resonate with a wide range of frequencies, which is essential for collecting energy from the ZPE.

The following equations are used to calculate the fractal dimension of a fractal-based harvester:

% TODO: Add fractal dimension formulas

The following figure shows a diagram of a fractal-based harvester:

% TODO: Add fractal harvester diagram


\section{Material Requirements}
\label{sec:material_requirements}

The materials used to construct a ZPE energy harvesting device are critical to its performance. The ideal materials will have a number of properties, including high superconductivity, high dielectric constant, and the ability to withstand high temperatures and pressures.

Superconducting materials are essential for ZPE energy harvesting devices, as they allow for the creation of strong magnetic fields with minimal energy loss. The most promising superconducting materials for this application are yttrium barium copper oxide (YBCO) and niobium-titanium (NbTi).

The dielectric properties of the materials used in a ZPE energy harvesting device are also important. A high dielectric constant is desirable, as it allows for the storage of more energy in the device.

Finally, the materials used in a ZPE energy harvesting device must be able to withstand high temperatures and pressures. This is because the process of extracting energy from the vacuum can generate a great deal of heat and pressure.

The following table compares the properties of some common materials used in ZPE energy harvesting devices:

% TODO: Add table of material properties


\section{Projected Performance}
\label{sec:projected_performance}

The projected performance of ZPE energy harvesting devices is a subject of much debate. Some scientists believe that these devices have the potential to provide a clean, limitless source of energy, while others are more skeptical.

The power density of a ZPE energy harvesting device is a measure of the amount of power that it can generate per unit volume. The projected power density of these devices varies widely, from a few watts per cubic centimeter to many megawatts per cubic centimeter.

The efficiency of a ZPE energy harvesting device is a measure of the amount of energy that it can extract from the vacuum, divided by the amount of energy that is required to operate the device. The projected efficiency of these devices also varies widely, from a few percent to over 100 percent.

The scalability of ZPE energy harvesting devices is a measure of their ability to be scaled up to provide large amounts of power. The scalability of these devices is a major challenge, as it is difficult to maintain the necessary conditions for ZPE energy harvesting over large volumes.

Finally, it is important to compare the projected performance of ZPE energy harvesting devices with that of conventional energy sources. While ZPE energy harvesting devices have the potential to provide a clean, limitless source of energy, they are still in the early stages of development. It will be many years before they are able to compete with conventional energy sources on a large scale.

% TODO: Add 15-20 citations to experimental papers


\section{Technology Readiness}
\label{sec:technology_readiness}

The technology readiness level (TRL) of ZPE energy harvesting devices is currently estimated to be between 2 and 3. This means that the basic principles have been observed and reported, but the technology is still in the early stages of development.

The development roadmap for ZPE energy harvesting devices is a long and challenging one. The first step is to develop a better understanding of the fundamental physics of ZPE. This will require a combination of theoretical and experimental work.

Once the fundamental physics is better understood, the next step is to develop more efficient and scalable ZPE energy harvesting devices. This will require a significant investment in research and development.

The main challenges and obstacles to the development of ZPE energy harvesting devices are the low power density, low efficiency, and poor scalability of current devices. These challenges will need to be overcome before ZPE energy harvesting can become a viable source of energy.

The timeline to a prototype ZPE energy harvesting device is difficult to predict. However, it is likely that it will be many years before a practical device is developed.