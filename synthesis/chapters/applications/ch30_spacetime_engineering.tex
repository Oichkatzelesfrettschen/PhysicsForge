\chapter{Spacetime Engineering}
\label{ch:spacetime_engineering}

\section{Nodespace Propulsion Theoretical Frameworks}
\label{sec:nodespace_propulsion_theoretical_frameworks}

The theoretical basis for nodespace propulsion is rooted in the Genesis framework, which posits that spacetime is a projection of a higher-dimensional reality. This higher-dimensional reality, or "nodespace," is a complex, multi-dimensional space where the fundamental laws of physics are unified.

The Genesis framework proposes that it is possible to travel through spacetime by navigating through nodespace. This is accomplished by "folding" spacetime, creating a shortcut between two distant points. The folding of spacetime is accomplished by manipulating the local geometry of nodespace.

The following equations from the Genesis framework are central to this concept:

%==============================================================================
% Equation: Origami Dimensional Folding Mechanism
% Source: math5GenesisFrameworkUnveiled.md (origami dimensions, lines 85-250)
%         Alpha001.06 (fold-merge operators, lines 7000-7500)
% Framework: Genesis | Domain: MATH | Status: Theoretical
%==============================================================================
% Mathematical description of Genesis framework's origami folding mechanism,
% which enables dimensional compactification from fundamental 2048D Cayley-
% Dickson structure down to observable 4D spacetime. This mechanism differs
% from Kaluza-Klein compactification through explicit geometric folding
% parameterized by folding angles.
%==============================================================================

\begin{equation}
  D_{\text{folded}}(D_{\text{high}}, \{\theta_i\}, \{w_i\})
    = D_{\text{low}} + \sum_{i=1}^{N_{\text{folds}}} w_i (D_{\text{high}} - D_{\text{low}})
      \cos^2\left(\frac{\theta_i}{2}\right) \prod_{j<i} \sin^2\left(\frac{\theta_j}{2}\right)
  \eqtag{G}{MATH}{T}
  \label{eq:genesis:origami-folding}
\end{equation}

\noindent
where:
\begin{itemize}
  \item $D_{\text{high}}$: Fundamental high-dimensional space (e.g., $2048$D Cayley-Dickson)
  \item $D_{\text{low}}$: Target low-dimensional projection (typically $4$D spacetime)
  \item $N_{\text{folds}}$: Number of sequential origami folds applied
  \item $\theta_i$: Folding angle for the $i$-th fold ($\theta_i \in [0, \pi]$)
  \item $w_i$: Weight factor for $i$-th fold (satisfying $\sum_{i=1}^{N_{\text{folds}}} w_i = 1$)
  \item Product term $\prod_{j<i} \sin^2(\theta_j/2)$: Sequential folding dependency
\end{itemize}

%==============================================================================
% LIMITING CASES
%==============================================================================

\paragraph{Limiting Behavior:}
\begin{itemize}
  \item \textbf{Fully unfolded} ($\theta_i = 0$ for all $i$):
    \begin{equation*}
      D_{\text{folded}} = D_{\text{low}} + (D_{\text{high}} - D_{\text{low}}) \sum_i w_i = D_{\text{high}}
    \end{equation*}
    All dimensions are accessible.

  \item \textbf{Fully folded} ($\theta_i = \pi$ for all $i$):
    \begin{equation*}
      D_{\text{folded}} = D_{\text{low}}
    \end{equation*}
    Only the base low-dimensional space remains observable.

  \item \textbf{Single fold} ($N_{\text{folds}} = 1$, $w_1 = 1$):
    \begin{equation*}
      D_{\text{folded}} = D_{\text{low}} + (D_{\text{high}} - D_{\text{low}}) \cos^2\left(\frac{\theta_1}{2}\right)
    \end{equation*}
    Simple interpolation between low and high dimensions.
\end{itemize}

%==============================================================================
% WORKED EXAMPLE: 2048D TO 4D
%==============================================================================

\paragraph{Worked Example:} Map $2048$D to $4$D via three sequential folds:
\begin{align*}
  D_{\text{high}} &= 2048 \\
  D_{\text{low}} &= 4 \\
  N_{\text{folds}} &= 3 \\
  \theta_1 &= \pi/3, \quad \theta_2 = \pi/4, \quad \theta_3 = \pi/2 \\
  w_1 &= 0.5, \quad w_2 = 0.3, \quad w_3 = 0.2
\end{align*}

Calculate each term:
\begin{align*}
  \text{Term 1:} &\quad 0.5 \cdot 2044 \cdot \cos^2(\pi/6) = 0.5 \cdot 2044 \cdot 0.75 = 766.5 \\
  \text{Term 2:} &\quad 0.3 \cdot 2044 \cdot \cos^2(\pi/8) \cdot \sin^2(\pi/6) \\
                 &= 0.3 \cdot 2044 \cdot 0.854 \cdot 0.25 = 131.2 \\
  \text{Term 3:} &\quad 0.2 \cdot 2044 \cdot \cos^2(\pi/4) \cdot \sin^2(\pi/6) \cdot \sin^2(\pi/8) \\
                 &= 0.2 \cdot 2044 \cdot 0.5 \cdot 0.25 \cdot 0.146 = 7.46
\end{align*}

Therefore:
\begin{equation*}
  D_{\text{folded}} = 4 + 766.5 + 131.2 + 7.46 \approx 909.2
\end{equation*}

This intermediate folding leaves an effective $\sim 909$D structure, requiring additional folds or different parameters to achieve full compactification to $4$D.

%==============================================================================
% COMPLETE 2048D TO 4D FOLDING
%==============================================================================

\paragraph{Complete Compactification:} For maximal folding to $4$D, use:
\begin{equation}
  \theta_i = \pi - \epsilon_i \quad \text{with} \quad \epsilon_i \ll 1
  \label{eq:genesis:maximal-folding}
\end{equation}

In the limit $\epsilon_i \to 0$, all folds approach $\theta_i = \pi$ and $D_{\text{folded}} \to D_{\text{low}} = 4$.

Alternatively, employ hierarchical folding with exponentially weighted angles:
\begin{equation}
  \theta_i = \pi \left(1 - 2^{-i}\right), \quad w_i = \frac{2^{-i}}{\sum_{j=1}^{N} 2^{-j}}
  \label{eq:genesis:hierarchical-folding}
\end{equation}

This ensures systematic dimensional reduction from $2048$D through intermediate Cayley-Dickson levels (1024D, 512D, 256D, ..., 8D, 4D).

%==============================================================================
% COMPARISON TO KALUZA-KLEIN COMPACTIFICATION
%==============================================================================

\paragraph{Origami vs Kaluza-Klein:}

\begin{center}
\begin{tabular}{lll}
\toprule
\textbf{Feature} & \textbf{Origami Folding} & \textbf{Kaluza-Klein} \\
\midrule
Mechanism & Geometric folding (angles $\theta_i$) & Topological compactification \\
Parameters & Folding angles, weights & Compactification radii $R_i$ \\
Dimension change & Continuous via $\cos^2(\theta/2)$ & Discrete (compact vs non-compact) \\
Observable effects & Fractal corrections to scattering & Kaluza-Klein tower of massive modes \\
Energy scale & $E \sim \hbar c / (a_0 \theta)$ & $E \sim \hbar c / R$ \\
Flexibility & Adjustable folding patterns & Fixed topology (e.g., tori, Calabi-Yau) \\
\bottomrule
\end{tabular}
\end{center}

Key distinction: Origami folding allows \emph{continuous} variation of effective dimensionality through angular parameters, whereas Kaluza-Klein yields discrete spectra of compactified modes. Both mechanisms can coexist, with origami providing smooth transitions between Kaluza-Klein plateaus.

%==============================================================================
% PHYSICAL INTERPRETATION
%==============================================================================

\paragraph{Physical Meaning:}
\begin{itemize}
  \item Origami folding represents a \emph{dynamical} compactification where effective dimensionality varies with local spacetime curvature, scalar field configurations, and ZPE density
  \item Folding angles $\theta_i$ may be tied to vacuum expectation values of scalar fields, making dimensional structure environment-dependent
  \item The sequential product $\prod_{j<i} \sin^2(\theta_j/2)$ ensures that earlier folds modulate the effectiveness of later folds, creating hierarchical structure
  \item In high-ZPE regions (near black holes, cosmological singularities), folding may partially reverse ($\theta_i \to 0$), locally exposing higher dimensions
  \item Observable 4D spacetime emerges as an effective low-energy description with nearly complete folding ($\theta_i \approx \pi$)
\end{itemize}

%==============================================================================
% EXPERIMENTAL SIGNATURES
%==============================================================================

\paragraph{Experimental Tests:}
\begin{itemize}
  \item \textbf{Dimensional resonances}: Partial unfolding at high energies should produce resonances at $E_{\text{res},i} \sim \hbar c/(a_0 \theta_i)$
  \item \textbf{Gravitational wave polarization}: Extra polarization modes if dimensions partially unfold during black hole mergers
  \item \textbf{Collider anomalies}: Deviations from 4D scattering amplitudes at TeV scale if folding is incomplete
  \item \textbf{Casimir force modifications}: Folding geometry alters boundary conditions, producing measurable force corrections
  \item \textbf{Cosmological imprints}: Early universe may have had different folding configuration, leaving signatures in CMB
\end{itemize}

%==============================================================================
% FOLD-MERGE OPERATOR CONNECTION
%==============================================================================

\paragraph{Fold-Merge Operator:} The Genesis framework defines the fold-merge operator $\mathcal{F}\mathcal{M}$ (Alpha001.06) as:
\begin{equation}
  \mathcal{F}\mathcal{M} = K_{\text{origami-folding}}(x,t) \cdot K_{\text{recursive-fractal}}(x,t)
    \cdot K_{\text{modular-symmetry}}(x)
  \label{eq:genesis:fold-merge-operator}
\end{equation}

The origami folding kernel $K_{\text{origami-folding}}$ is constructed from the dimensional folding formula via:
\begin{equation}
  K_{\text{origami-folding}}(x,t) = \exp\left( -\frac{1}{2} \sum_{i=1}^{N}
    \frac{(\theta_i(x,t) - \theta_{i,0})^2}{\sigma_i^2} \right)
  \label{eq:genesis:origami-kernel}
\end{equation}

where $\theta_i(x,t)$ are spacetime-dependent folding angles, $\theta_{i,0}$ are equilibrium values, and $\sigma_i$ are folding fluctuation widths. This connects the geometric folding mechanism to quantum field kernel formalism.

%==============================================================================
% DEPENDENCIES AND CONNECTIONS
%==============================================================================
% Dependencies: Ch02 (Cayley-Dickson 2048D structure)
%               Ch13 (Genesis origami framework)
%               Ch18 (Dimensional conflict resolution)
%
% Forward references: Ch20 Section 5 (detailed origami folding derivation)
%                     Ch21 (Unified dimensional synthesis)
%                     Ch23 (Experimental dimensional spectroscopy)
%
% See also: eq:unified:origami-folding in eq_dimensional_mapping_unified.tex
%           for simplified single-fold formula
%==============================================================================

%==============================================================================
% Equation: Dimensional mapping between Aether and Genesis frameworks
% Source: FRAMEWORK_CONFLICT_MATRIX_ANALYSIS.md (Section 1)
%         Ch18 conflict resolution analysis (dimensional reconciliation)
% Framework: Unified (maps Aether <-> Genesis)
% Domain: MATH | Status: Theoretical (requires experimental validation)
%==============================================================================
\begin{equation}
  d_{\text{Aether}}(n, \theta) = \left\lfloor 4 + n \cdot \frac{\theta}{2\pi} \right\rfloor
  \eqtag{U}{MATH}{T}
  \label{eq:unified:dimensional_mapping}
\end{equation}
% Notes: Maps Genesis origami folds (n-fold structure with angle theta) to
% Aether integer-dimensional projections. The floor function captures discrete
% dimensional jumps observed in Aether framework while theta/2pi captures
% Genesis continuous fractal parameter.
%
% Physical interpretation:
% - n: Number of recursive origami folds (Genesis nodespace parameter)
% - theta: Folding angle in [0, 2*pi) (Genesis geometric parameter)
% - d_Aether: Effective integer dimension (3D, 4D, 5D, ..., 8D, ...)
%
% Example: n=2 folds at theta=pi gives d_Aether = floor(4 + 2*0.5) = 5D
% This corresponds to Aether's 5D scalar-ZPE well interpretation.
%
% Dependencies: Ch02 (Cayley-Dickson), Ch13 (Genesis origami), Ch18 (resolution)
% Experimental test: Dimensional spectroscopy (Ch24) should reveal resonances
% at energies E ~ hbar*c/(L_fold) where L_fold ~ theta/n determines fold scale.
%==============================================================================



\section{Feasibility Analysis}
\label{sec:feasibility_analysis}

The feasibility of nodespace propulsion is a subject of much debate. While the theoretical framework is sound, there are a number of significant practical challenges that must be overcome before this technology can be realized.

The first challenge is the enormous amount of energy that is required to manipulate nodespace. The energy requirements for even a small-scale demonstration of nodespace propulsion are far beyond our current capabilities.

The second challenge is the fact that nodespace propulsion is constrained by the laws of general relativity. In particular, the speed of light is an absolute limit that cannot be exceeded. This means that even with nodespace propulsion, it would still take many years to travel to the nearest stars.

The third challenge is the fact that nodespace propulsion is also constrained by the laws of quantum field theory. In particular, the uncertainty principle makes it difficult to precisely control the geometry of nodespace.

Despite these challenges, there is still a great deal of interest in nodespace propulsion. This is because it offers the potential for a revolutionary new form of transportation that could one day allow us to explore the galaxy.


\section{Differentiation between Speculative and Empirically Supported Concepts}
\label{sec:differentiation_between_speculative_and_empirically_supported_concepts}

It is important to differentiate between the speculative and empirically supported concepts in the field of spacetime engineering. While the theoretical framework for nodespace propulsion is sound, there is currently no experimental evidence to support it.

The most speculative aspect of nodespace propulsion is the idea that it is possible to create a stable, traversable wormhole. While wormholes are a valid solution to the equations of general relativity, there is no known way to create them.

The most empirically supported aspect of nodespace propulsion is the idea that it is possible to manipulate the geometry of spacetime on a very small scale. This has been demonstrated in a number of experiments, such as the Casimir effect.

The experimental validation pathways for nodespace propulsion are long and challenging. The first step is to develop a better understanding of the fundamental physics of nodespace. This will require a combination of theoretical and experimental work.

Once the fundamental physics is better understood, the next step is to develop a small-scale demonstration of nodespace propulsion. This would be a major breakthrough, and it would open the door to the development of a practical nodespace propulsion system.

The falsifiable predictions of nodespace propulsion are that it should be possible to create a stable, traversable wormhole, and that it should be possible to travel faster than the speed of light. If these predictions are not confirmed by experiment, then the theory of nodespace propulsion will need to be revised.

The risk assessment for nodespace propulsion is that it could be used to create a weapon of mass destruction. It is therefore important to ensure that this technology is developed in a responsible manner.


\section{Pais GEM Force Predictions}
\label{sec:pais_gem_force_predictions}

The Pais GEM (Gravitoelectromagnetism) formalism, as detailed in Chapter~\ref{ch:pais_gem_formalism}, predicts that it is possible to generate a fifth force by coupling the gravitational and electromagnetic fields. This fifth force could be used for propulsion, and it could also be used to create a "warp drive" that would allow for faster-than-light travel.

The coupling mechanism for the fifth force is the scalar field, which is a field that has a single value at every point in spacetime. By manipulating the scalar field, it is possible to create a localized region of negative energy density, which in turn allows for the extraction of energy from the vacuum.

The thrust calculations for the fifth force are based on the following equations:

%==============================================================================
% Equation: Gravitational-electromagnetic coupling (Pais Superforce context)
% Source: draft reply to pais.md (Experimental section)
%==============================================================================
\begin{equation}
  \mathbf{F}_{\mathrm{GEM}}
    = \rho\,\mathbf{g}
    + \frac{1}{c^{2}}\,\mathbf{J} \times \mathbf{B}_{g}
  \eqtag{P}{EM}{proposal}
  \label{eq:pais:gem-coupling}
\end{equation}
% Notes:
%   * $\rho$ effective mass/charge density; $\mathbf{g}$ gravitational field.
%   * $\mathbf{J}$ electric current density; $\mathbf{B}_{g}$ gravitomagnetic field.
%   * Requires experimental constraints; currently a theoretical proposal.
%==============================================================================

%==============================================================================
% Equation: Fifth force Yukawa modification to Newtonian gravity
% Source: Standard scalar-tensor theory (Brans-Dicke, chameleon, etc.)
% Framework: Pais | Domain: GR | Status: Experimental constraints
%==============================================================================
\begin{equation}
  V(r)
    = -\frac{GM}{r}
      \left[ 1 + \alpha \, \ee^{-r/\lambda} \right]
  \eqtag{P}{GR}{E}
  \label{eq:pais:fifth-force}
\end{equation}
% Notes:
%   * $G$ Newton's gravitational constant
%   * $M$ source mass
%   * $r$ distance from source
%   * $\alpha = \beta^{2}$ fifth force strength (dimensionless)
%   * $\lambda = \hbar/(m_{\phi} c)$ Compton wavelength of scalar mediator (range)
%   * Experimental constraints (representative):
%       - $\lambda \sim 1$ micrometer: $\alpha < 10^{-6}$ (Eot-Wash torsion balance)
%       - $\lambda \sim 1$ mm: $\alpha < 10^{-4}$ (atom interferometry)
%       - $\lambda \sim 10^{7}$ m: $\alpha < 10^{-10}$ (GRACE satellite geodesy)
%   * At short range $r \ll \lambda$: $V(r) \approx -GM(1+\alpha)/r$ (constant offset)
%   * At long range $r \gg \lambda$: exponential suppression, recovers Newtonian gravity
%==============================================================================

%==============================================================================
% Equation: Zero-Point Energy Thrust via Casimir-like Effects
% Source: Casimir effect derivation + Alpha003.02 ZPE extraction concepts
% Framework: Aether | Domain: QM | Status: Experimental
%==============================================================================
\begin{equation}
  F_{\text{thrust}} = \frac{\hbar c \pi^2}{240 d^4} A_{\text{plate}} \xi_{\text{geom}}
  \eqtag{S}{QM}{E}
  \label{eq:zpe:thrust-casimir}
\end{equation}
%
% where:
%   F_thrust = thrust force from ZPE extraction (N)
%   hbar     = reduced Planck constant (1.055 x 10^{-34} J*s)
%   c        = speed of light (3 x 10^8 m/s)
%   d        = separation between parallel plates (m)
%   A_plate  = plate area (m^2)
%   xi_geom  = geometry enhancement factor (dimensionless)
%
% Standard Casimir force (xi_geom = 1):
% For parallel conducting plates, attractive force per unit area:
%   P_Casimir = -hbar*c*pi^2 / (240*d^4)
%
% This equation generalizes to thrust generation via asymmetric geometries:
%
% Geometry enhancement factors (examples):
%   xi_geom = 1.0   : Parallel plates (standard Casimir, no net thrust)
%   xi_geom = 1.5   : Tilted plates (angle ~ 10 degrees)
%   xi_geom = 3.0   : Curved surfaces (parabolic reflectors)
%   xi_geom = 10    : Fractal microstructures (resonant cavity modes)
%   xi_geom = 100   : Metamaterial cavities (negative index, slow light)
%
% Thrust scaling examples:
%
% 1. Laboratory demonstration (microscale):
%    - d = 100 nm, A = 1 cm^2 = 10^{-4} m^2, xi_geom = 10
%    - F_thrust ~ (10^{-34} * 3e8 * 10) / (240 * (10^{-7})^4) * 10^{-4} * 10
%    - F_thrust ~ 1.3 x 10^{-15} N (1.3 femtonewtons)
%    - Measurable with AFM cantilevers (force resolution ~ 10^{-15} N)
%
% 2. Small spacecraft (optimistic engineering):
%    - d = 10 nm, A = 1 m^2, xi_geom = 100 (metamaterial cavity array)
%    - F_thrust ~ (10^{-34} * 3e8 * 10) / (240 * (10^{-8})^4) * 1 * 100
%    - F_thrust ~ 1.3 x 10^{-6} N (1.3 micronewtons)
%    - Compare: Ion thruster ~ 10-100 mN (10^4 - 10^5 times higher)
%
% 3. Extreme parameters (highly speculative):
%    - d = 1 nm, A = 100 m^2, xi_geom = 1000 (active ZPE manipulation)
%    - F_thrust ~ 1.3 x 10^{-3} N (1.3 millinewtons)
%
% Efficiency analysis:
% Energy input (to maintain field configuration) vs. momentum output:
%   eta = (F_thrust * v) / P_input
%
% For passive Casimir structures, P_input ~ 0 (no energy consumption after
% fabrication), but thrust is extremely small. For active systems (dynamic
% cavity tuning, field modulation), P_input ~ kW to MW, limiting efficiency.
%
% Specific impulse comparison:
%   I_sp = F_thrust / (mdot * g_0)
%
% ZPE thruster (no propellant mass ejection, mdot = 0):
%   I_sp -> infinity (in principle)
%
% However, practical systems require power supply mass, structure, etc.:
%   I_sp_effective ~ F_thrust / (P_input/c^2) / g_0
%
% For F_thrust = 10^{-6} N, P_input = 1 kW:
%   I_sp_effective ~ 10^{-6} / (1000 / (3e8)^2) / 9.8 ~ 10^7 s
%
% This is 10^4 times higher than chemical rockets (I_sp ~ 300 s) and 100 times
% higher than ion thrusters (I_sp ~ 10^5 s), but thrust level is 10^6 times lower.
%
% Applications:
% - Microspacecraft (CubeSats, chip satellites): acceleration ~ F/m ~ 10^{-6} N / 0.1 kg = 10^{-5} m/s^2
%   - Delta-v accumulation: 1 m/s per ~10^5 s ~ 1 day of continuous operation
% - Attitude control for larger spacecraft (torque generation via differential thrust)
% - Long-duration missions (decades) where cumulative delta-v matters more than instantaneous thrust
%
% Experimental status:
% - Static Casimir force measured to <1% accuracy (Lamoreaux 1997, et al.)
% - Dynamic Casimir effect demonstrated (Wilson et al., Nature 2011)
% - Directional thrust from ZPE: NO confirmed experiments to date
%
% Critical challenges:
% - Thrust-to-weight ratio: F_thrust / (m_structure * g_0) << 1 for all known designs
% - Power supply mass: Solar panels or nuclear batteries add 10-100 kg
% - Thermal management: Waste heat dissipation in vacuum (radiative cooling only)
% - Material stability: Nanoscale gaps require atomic-precision fabrication and vibration isolation
%
% Dependencies: Ch01 (quantum field theory), Ch07-Ch09 (ZPE theory),
%               Ch22 (Casimir force measurements)
% Cross-references: Ch28 (energy extraction), Ch29 (propulsion system integration),
%                   Ch30 (vacuum engineering)
%==============================================================================


The reference to the GEM coupling equation is \eqref{eq:pais:gem-coupling}.


\section{General Relativity Reconciliation}
\label{sec:general_relativity_reconciliation}

The reconciliation of nodespace propulsion with general relativity is a major challenge. This is because nodespace propulsion appears to violate the speed of light, which is a cornerstone of general relativity.

One way to reconcile nodespace propulsion with general relativity is to use a metric perturbation analysis. This involves modifying the spacetime metric to account for the effects of nodespace propulsion.

Another way to reconcile nodespace propulsion with general relativity is to modify the stress-energy tensor. This involves adding a new term to the stress-energy tensor that accounts for the energy and momentum of the nodespace propulsion system.

The equivalence principle is a key principle of general relativity that states that the effects of gravity are indistinguishable from the effects of acceleration. This principle has important implications for nodespace propulsion, as it means that a nodespace propulsion system would need to be able to generate a very large amount of acceleration in order to be effective.

The following equations are used to reconcile nodespace propulsion with general relativity:

\input{modules/equations/eq_aether_effective_metric_wormhole.tex}
\input{modules/equations/eq_metric_perturbation.tex}


\section{Ethical Implications and Risk Assessment}
\label{sec:ethical_implications_and_risk_assessment}

The development of spacetime engineering technologies raises a number of ethical implications and risks. It is important to consider these issues carefully before proceeding with the development of this technology.

One of the main ethical implications of spacetime engineering is the potential for it to be used as a weapon. A spacetime engineering device could be used to create a black hole, which could be used to destroy a planet. It is therefore important to ensure that this technology is developed in a responsible manner.

Another ethical implication of spacetime engineering is the potential for it to be used to create a new form of inequality. If this technology is only available to the wealthy, it could create a new class of "super-humans" who are able to travel through spacetime at will.

The risks of spacetime engineering are also significant. One of the main risks is the potential for a spacetime engineering device to malfunction. A malfunction could have catastrophic consequences, such as the creation of a black hole that destroys the Earth.

It is therefore important to develop a robust regulatory framework for spacetime engineering. This framework should ensure that this technology is developed in a safe and responsible manner.