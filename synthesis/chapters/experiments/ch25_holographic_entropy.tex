%==============================================================================
% Chapter 25: Holographic Entropy and Black Hole Analogues
% Part IV: Experimental Validation
%
% Source: Alpha003.02 (Sections 0.10, 1.13, 7.10, 8.8, 10.7, lines 132-141, 415-432, 1545-1555, 1705-1710, 2053-2070, 2089-2095)
%         Ch18 (experimental distinguishability)
%         Ch21 (unified predictions)
%==============================================================================

\chapter{Holographic Entropy and Black Hole Analogues}\label{ch:exp_holographic}

\section{Introduction: Testing Holographic Principles}

The holographic principle, originating from black hole thermodynamics and formalized through the AdS/CFT correspondence, posits that the information content of a volume of space can be encoded on its boundary. The cornerstone prediction is the Bekenstein-Hawking entropy formula:

\begin{equation}
  S_{\text{BH}} = \frac{A}{4 G \hbar} = \frac{k_B c^3 A}{4 G \hbar}
  \label{eq:holo:bekenstein-hawking}
\end{equation}

where $A$ is the event horizon area, establishing entropy proportional to \emph{area} rather than volume---a profound departure from thermodynamic intuition.

The three frameworks under consideration predict modifications to this baseline formula:

\begin{itemize}
  \item \textbf{\aetherattr}: Scalar-ZPE coupling introduces volumetric corrections to entropy via vacuum coherence within the horizon (Alpha003.02, Section 0.10):
  \begin{equation}
    S = \frac{A}{4 G \hbar} + \alpha \int_{\mathcal{V}} \rho_{\text{ZPE}}(x) d^3 x
    \label{eq:intro:aether-entropy}
  \end{equation}
  where $\alpha \sim 10^{-2}$ to $10^{-1}$ is the predicted coupling strength.

  \item \textbf{\genesisattr}: Nodespace discreteness modifies horizon thermodynamics through fractal boundary corrections:
  \begin{equation}
    S = \frac{A}{4 G \hbar} \left( 1 + \beta \frac{\ell_{\text{node}}^2}{A} \right)
    \label{eq:intro:genesis-entropy}
  \end{equation}
  with $\beta$ determined by nodespace topology (predicted $\beta \sim 1$ to 10).

  \item \textbf{Standard GR}: Unmodified Bekenstein-Hawking formula holds exactly, no volumetric or discrete corrections.
\end{itemize}

Direct observation of astrophysical black holes cannot resolve these subtle differences ($\Delta S / S \sim 10^{-3}$ to $10^{-2}$). However, \textbf{analog black hole systems}---laboratory constructs mimicking event horizon physics in condensed matter or optical media---provide controlled testbeds for holographic entropy with tunable parameters and accessible measurement regimes.

This chapter presents comprehensive protocols for testing holographic entropy modifications using acoustic black holes in Bose-Einstein condensates (BEC) and optical black holes in nonlinear photonic media.

\section{Theoretical Framework}

\subsection{Bekenstein-Hawking Entropy: $S = A / 4G$}

Black hole thermodynamics establishes four laws paralleling classical thermodynamics:

\begin{enumerate}
  \item \textbf{Zeroth law}: Surface gravity $\kappa$ constant on horizon
  \item \textbf{First law}: $dM = \frac{\kappa}{8\pi G} dA + \Omega dJ + \Phi dQ$ (energy balance)
  \item \textbf{Second law}: Horizon area $A$ never decreases ($dA \geq 0$)
  \item \textbf{Third law}: Impossible to reach $\kappa = 0$ in finite operations
\end{enumerate}

Comparing the first law to thermodynamics $dE = T dS + \ldots$ identifies:

\begin{align}
  T_{\text{Hawking}} &= \frac{\hbar \kappa}{2\pi k_B c} \label{eq:theory:hawking-temp} \\
  S_{\text{BH}} &= \frac{k_B c^3 A}{4 G \hbar} = \frac{k_B A}{4 \ell_P^2} \label{eq:theory:bh-entropy}
\end{align}

For a Schwarzschild black hole of mass $M$, this yields:

\begin{equation}
  S_{\text{BH}} = \frac{\pi k_B c^3}{\hbar G} M^2 = k_B \left( \frac{M}{m_P} \right)^2 \approx 10^{77} k_B \left( \frac{M}{M_\odot} \right)^2
  \label{eq:theory:schwarzschild-entropy}
\end{equation}

The enormous entropy ($10^{77}$ to $10^{90}$ for stellar-mass to supermassive black holes) reflects the vast information hidden behind the horizon.

\subsection{Aether Modifications: Scalar-ZPE Contributions}

The \aether~framework introduces scalar field-ZPE interactions that modify near-horizon physics. The unified energy density (Alpha003.02, Section 0.3):

\begin{equation}
  \rho_{\text{total}} = \rho_{\text{ZPE}} + g \phi(t) + \lambda \phi \text{ZPE}^2
  \label{eq:theory:aether-density}
\end{equation}

persists into the black hole interior, creating coherent vacuum structures. The corrected entropy:

%==============================================================================
% Equation: Modified holographic entropy (Aether framework)
% Source: Alpha003.02 (Section 0.10, lines 132-141)
%         Ch21 (unified predictions for holographic corrections)
% Framework: Aether | Domain: GR/QM | Status: Experimental test
%==============================================================================
\begin{equation}
  S_{\text{holo}} = \frac{k_B c^3 A}{4 G \hbar}
  + \alpha \int_{\mathcal{V}} \rho_{\text{ZPE}}(x) \, d^3 x
  \eqtag{S}{GR}{E}
  \label{eq:holo:entropy-modified-aether}
\end{equation}
% Notes:
%   * First term: Standard Bekenstein-Hawking entropy (area law)
%   * Second term: Volumetric correction from scalar-ZPE coupling
%   * $A$ = event horizon area (or effective horizon area in analog systems)
%   * $\rho_{\text{ZPE}}$ = zero-point energy density within horizon volume $\mathcal{V}$
%   * $\alpha$ = coupling constant (predicted $\alpha \sim 10^{-2}$ to $10^{-1}$ in natural units)
%   * Dimensionally: $[S] = [k_B] = \text{energy} / \text{temperature}$
%   * Standard GR: $\alpha = 0$ (no volumetric term)
%   * Genesis framework uses different correction: $S \propto (1 + \beta \ell_{\text{node}}^2 / A)$
% Experimental test:
%   * Vary horizon volume $\mathcal{V}$ at fixed area $A$ in BEC analog systems
%   * Measure ZPE density independently via Casimir force (Ch22)
%   * Expected sensitivity: $\Delta S / S \sim 10^{-2}$ to $10^{-1}$ in analog BH
% Dependencies:
%   * Ch01: Fundamental constants and Planck scale
%   * Ch07: Aether scalar-ZPE coupling formalism
%   * Ch22: Scalar-ZPE experimental protocols (ZPE density measurement)
%   * Ch25: Holographic entropy tests in analog black holes
%==============================================================================


The volumetric term scales as:

\begin{equation}
  \Delta S_{\text{vol}} = \alpha \int_{\mathcal{V}} \rho_{\text{ZPE}} d^3 x
  \sim \alpha \frac{\rho_{\text{ZPE}} \cdot R_s^3}{k_B}
  \sim \alpha \frac{c^6}{G^2 \hbar M}
  \label{eq:theory:aether-correction}
\end{equation}

where $R_s = 2GM/c^2$ is the Schwarzschild radius. For stellar-mass black holes ($M \sim M_\odot$), this predicts:

\begin{equation}
  \frac{\Delta S_{\text{vol}}}{S_{\text{BH}}} \sim \alpha \frac{m_P}{M} \sim 10^{-40} \alpha
  \label{eq:theory:aether-fraction}
\end{equation}

Completely negligible for astrophysical black holes! However, in analog systems with effective Planck mass $m_P^{\text{eff}} \sim 10^{-26}$ kg (atomic mass scale), the correction becomes:

\begin{equation}
  \frac{\Delta S_{\text{vol}}}{S_{\text{BH}}} \sim \alpha \cdot 10^{-2} \text{ to } 10^{-1}
  \label{eq:theory:aether-analog}
\end{equation}

potentially detectable with precision thermometry.

\subsection{Genesis Modifications: Nodespace Discreteness}

The \genesis~framework models spacetime as discrete nodespace with characteristic length $\ell_{\text{node}}$. The event horizon, traditionally a smooth null surface, acquires discrete structure. The horizon area quantization:

\begin{equation}
  A = N_{\text{nodes}} \cdot \ell_{\text{node}}^2
  \label{eq:theory:genesis-area}
\end{equation}

where $N_{\text{nodes}}$ is the number of nodespace cells intersecting the horizon. Standard quantum gravity (loop quantum gravity) predicts similar discreteness with $\ell_{\text{node}} \sim \ell_P$ and quantized area eigenvalues:

\begin{equation}
  A_n = 8\pi \gamma \ell_P^2 \sqrt{n(n+1)}, \quad n = 0, 1, 2, \ldots
  \label{eq:theory:lqg-area}
\end{equation}

The \genesis~framework extends this with fractal boundary corrections. The entropy becomes:

\begin{equation}
  S = \frac{k_B A}{4 \ell_{\text{node}}^2} \left( 1 + \beta \frac{\ell_{\text{node}}^2}{A} + \gamma \frac{\ell_{\text{node}}^4}{A^2} + \ldots \right)
  \label{eq:theory:genesis-series}
\end{equation}

For macroscopic black holes ($A \gg \ell_{\text{node}}^2$), corrections are negligible. But analog systems with effective horizon areas $A_{\text{eff}} \sim 10^{-12}$ m$^2$ (mm-scale BEC) yield:

\begin{equation}
  \frac{\Delta S}{S_{\text{BH}}} \sim \beta \left( \frac{\ell_{\text{node}}}{10^{-6} \text{ m}} \right)^2 \sim 10^{-2} \beta
  \label{eq:theory:genesis-analog}
\end{equation}

for nodespace scales $\ell_{\text{node}} \sim 10^{-6}$ to $10^{-5}$ m (micrometer range).

\section{Analog Black Hole Systems}

Analog systems replicate black hole horizon physics by creating regions where excitation velocities exceed wave propagation speeds---the acoustic/optical equivalent of an event horizon.

\subsection{Acoustic Black Holes in BEC}

\textbf{Physical principle}:

A Bose-Einstein condensate with spatially varying flow velocity $v(x)$ supports phonon excitations with dispersion:

\begin{equation}
  \omega(k) = c_s k + \text{higher-order terms}
  \label{eq:analog:phonon-dispersion}
\end{equation}

where $c_s = \sqrt{gn/m}$ is the sound speed ($g$ = interaction strength, $n$ = density, $m$ = atomic mass). The effective metric for phonons:

\begin{equation}
  ds^2 = \frac{\rho}{c_s^2} \left[ -(c_s^2 - v^2) dt^2 - 2 v_i dx^i dt + dx^i dx^i \right]
  \label{eq:analog:acoustic-metric}
\end{equation}

An \textbf{acoustic horizon} forms where $v(x_h) = c_s$, trapping phonons analogous to light in a black hole.

\textbf{Experimental realization}:

\begin{itemize}
  \item \textbf{Condensate}: $^{87}$Rb or $^{23}$Na atoms, $N \sim 10^5$ to $10^6$, $T < 100$ nK
  \item \textbf{Trap}: Crossed optical dipole trap or magnetic quadrupole, tunable to create flow
  \item \textbf{Flow generation}: "Waterfall" geometry via moving optical potential or density step
  \item \textbf{Horizon properties}:
  \begin{align}
    \text{Effective surface gravity: } & \kappa_{\text{eff}} = \frac{d(v - c_s)}{dx}\bigg|_{x_h} \sim 10^3 \text{ to } 10^5 \text{ s}^{-1} \\
    \text{Hawking temperature: } & T_H^{\text{eff}} = \frac{\hbar \kappa_{\text{eff}}}{2\pi k_B} \sim 1 \text{ to } 100 \text{ nK}
  \end{align}
\end{itemize}

Crucially, $T_H^{\text{eff}}$ is comparable to BEC critical temperature, making thermal Hawking radiation potentially observable.

\subsection{Optical Black Holes in Nonlinear Media}

\textbf{Physical principle}:

Intense laser pulses in nonlinear optical media (e.g., self-focusing Kerr materials) create refractive index variations:

\begin{equation}
  n(I) = n_0 + n_2 I
  \label{eq:analog:kerr-nonlinearity}
\end{equation}

where $I$ is the intensity and $n_2 > 0$ (self-focusing) or $n_2 < 0$ (self-defocusing). The effective metric for probe photons:

\begin{equation}
  ds^2 = \frac{n^2(I)}{c^2} \left[ -c^2 dt^2 + (dx - v_g dt)^2 + dy^2 + dz^2 \right]
  \label{eq:analog:optical-metric}
\end{equation}

where $v_g$ is the group velocity of the pump pulse. An \textbf{optical horizon} forms where $v_g = c/n$, trapping probe photons.

\textbf{Experimental realization}:

\begin{itemize}
  \item \textbf{Material}: Fused silica, photonic crystal fiber, or rubidium vapor
  \item \textbf{Pump laser}: $\lambda = 532$ nm or 1064 nm, $P \sim 1$ W to 100 W, pulse duration $\sim$ ps to ns
  \item \textbf{Probe laser}: Frequency-shifted from pump, low power to avoid back-action
  \item \textbf{Horizon properties}:
  \begin{align}
    \text{Effective temperature: } & T_H^{\text{opt}} \sim \frac{\hbar c \kappa_{\text{eff}}}{2\pi k_B} \sim 10^3 \text{ to } 10^5 \text{ K} \\
    \text{Hawking emission: } & \text{Visible as correlated photon pairs (partner modes)}
  \end{align}
\end{itemize}

Optical systems enable room-temperature operation and sub-nanosecond time resolution, ideal for studying transient horizon dynamics.

\section{Experimental Protocol}

\subsection{System Preparation}

\textbf{BEC acoustic horizon protocol}:

\begin{enumerate}
  \item \textbf{Condensate formation} (Duration: 10-30 seconds per cycle):
  \begin{itemize}
    \item Laser cool $^{87}$Rb atoms to $\sim 100$ $\mu$K
    \item Transfer to magnetic/optical trap, evaporatively cool to $T < T_c \approx 100$ nK
    \item Verify condensate fraction $> 80\%$ via time-of-flight imaging
  \end{itemize}

  \item \textbf{Flow generation} (Duration: $\sim 100$ ms):
  \begin{itemize}
    \item Activate moving optical potential (blue-detuned laser barrier)
    \item Ramp barrier velocity from 0 to $v > c_s$ over 50 ms to avoid heating
    \item Stabilize flow with feedback on barrier position/intensity
  \end{itemize}

  \item \textbf{Horizon characterization}:
  \begin{itemize}
    \item Map density $n(x)$ and velocity $v(x)$ via Bragg spectroscopy
    \item Identify horizon position $x_h$ where $v(x_h) = c_s(x_h)$
    \item Measure surface gravity $\kappa_{\text{eff}}$ from velocity gradient
  \end{itemize}
\end{enumerate}

\textbf{Optical black hole protocol}:

\begin{enumerate}
  \item \textbf{Pump pulse injection}:
  \begin{itemize}
    \item Generate high-power pump pulse (1064 nm, 100 ps duration, $P \sim 10$ W peak)
    \item Focus into photonic crystal fiber or nonlinear crystal
    \item Monitor self-focusing via transverse beam profile imaging
  \end{itemize}

  \item \textbf{Probe detection}:
  \begin{itemize}
    \item Inject weak probe pulse ($P_{\text{probe}} \ll P_{\text{pump}}$) co-propagating with pump
    \item Frequency-resolve probe spectrum to detect Hawking pairs
    \item Cross-correlate pump and probe timing with ps resolution
  \end{itemize}

  \item \textbf{Horizon mapping}:
  \begin{itemize}
    \item Scan probe injection time to probe different horizon positions
    \item Reconstruct effective metric from probe deflection angles
  \end{itemize}
\end{enumerate}

\subsection{Hawking Radiation Detection}

\textbf{Signature in BEC}:

Hawking radiation manifests as correlated density fluctuations (phonon pairs) straddling the horizon. Detection strategy:

\begin{itemize}
  \item \textbf{In-situ imaging}: Phase-contrast or absorption imaging of density $n(x,t)$
  \item \textbf{Correlation analysis}: Compute two-point correlator $\langle \delta n(x, t) \delta n(y, t) \rangle$
  \item \textbf{Hawking signature}: Enhanced correlation at frequency $\omega \sim k_B T_H / \hbar$ across horizon
\end{itemize}

Expected thermal spectrum:

\begin{equation}
  \langle N_\omega \rangle = \frac{1}{e^{\hbar \omega / k_B T_H} - 1}
  \label{eq:protocol:bose-spectrum}
\end{equation}

Compare to non-thermal backgrounds (quantum shot noise, technical noise).

\textbf{Signature in optical system}:

Hawking radiation appears as correlated photon pairs (signal + idler) with frequencies $\omega_s + \omega_i = 2\omega_{\text{pump}}$. Detection via:

\begin{itemize}
  \item \textbf{Spectral measurement}: High-resolution spectrometer resolving $\Delta \omega \sim 1$ GHz
  \item \textbf{Coincidence counting}: Photon detectors with sub-ns timing resolution
  \item \textbf{Hawking signature}: Thermal photon number distribution at $T_H^{\text{opt}}$
\end{itemize}

\subsection{Entropy Measurement via Temperature}

The Bekenstein-Hawking entropy relates directly to measured Hawking temperature through:

\begin{equation}
  S_{\text{BH}} = \frac{2\pi k_B c^3}{\hbar G} \frac{M^2}{T_H}
  \label{eq:protocol:entropy-temperature}
\end{equation}

In analog systems, replace $G \to G_{\text{eff}}$ and $M \to M_{\text{eff}}$ (effective parameters). The experimental procedure:

\begin{enumerate}
  \item \textbf{Measure $T_H$} from Hawking radiation spectrum (Eq.~\ref{eq:protocol:bose-spectrum})

  \item \textbf{Determine effective horizon area} $A_{\text{eff}}$ from spatial extent of horizon

  \item \textbf{Compute baseline entropy}:
  \begin{equation}
    S_{\text{baseline}} = \frac{k_B c_s^2 A_{\text{eff}}}{4 G_{\text{eff}} \hbar}
    \label{eq:protocol:baseline-entropy}
  \end{equation}

  \item \textbf{Test framework corrections}:
  \begin{itemize}
    \item \textbf{\aether}: Measure volumetric ZPE density $\rho_{\text{ZPE}}$ independently (via Casimir force in cavity)
    \item \textbf{\genesis}: Vary $A_{\text{eff}}$ (by tuning BEC geometry) and test for $\propto 1/A$ corrections
  \end{itemize}

  \item \textbf{Compare measured $S_{\text{measured}}$ to predictions}:
  \begin{equation}
    \chi^2 = \sum_i \frac{(S_i^{\text{measured}} - S_i^{\text{model}})^2}{\sigma_i^2}
    \label{eq:protocol:chi-squared}
  \end{equation}
  Reject models with $\chi^2 / \text{dof} > 2$ at 95\% confidence.
\end{enumerate}

\section{Framework-Specific Predictions}

% DUPLICATE REMOVED: eq_holographic_entropy_modified already included at line 87
% %==============================================================================
% Equation: Modified holographic entropy (Aether framework)
% Source: Alpha003.02 (Section 0.10, lines 132-141)
%         Ch21 (unified predictions for holographic corrections)
% Framework: Aether | Domain: GR/QM | Status: Experimental test
%==============================================================================
\begin{equation}
  S_{\text{holo}} = \frac{k_B c^3 A}{4 G \hbar}
  + \alpha \int_{\mathcal{V}} \rho_{\text{ZPE}}(x) \, d^3 x
  \eqtag{S}{GR}{E}
  \label{eq:holo:entropy-modified-aether}
\end{equation}
% Notes:
%   * First term: Standard Bekenstein-Hawking entropy (area law)
%   * Second term: Volumetric correction from scalar-ZPE coupling
%   * $A$ = event horizon area (or effective horizon area in analog systems)
%   * $\rho_{\text{ZPE}}$ = zero-point energy density within horizon volume $\mathcal{V}$
%   * $\alpha$ = coupling constant (predicted $\alpha \sim 10^{-2}$ to $10^{-1}$ in natural units)
%   * Dimensionally: $[S] = [k_B] = \text{energy} / \text{temperature}$
%   * Standard GR: $\alpha = 0$ (no volumetric term)
%   * Genesis framework uses different correction: $S \propto (1 + \beta \ell_{\text{node}}^2 / A)$
% Experimental test:
%   * Vary horizon volume $\mathcal{V}$ at fixed area $A$ in BEC analog systems
%   * Measure ZPE density independently via Casimir force (Ch22)
%   * Expected sensitivity: $\Delta S / S \sim 10^{-2}$ to $10^{-1}$ in analog BH
% Dependencies:
%   * Ch01: Fundamental constants and Planck scale
%   * Ch07: Aether scalar-ZPE coupling formalism
%   * Ch22: Scalar-ZPE experimental protocols (ZPE density measurement)
%   * Ch25: Holographic entropy tests in analog black holes
%==============================================================================


%==============================================================================
% Equation: Modified Hawking temperature (Genesis framework)
% Source: Alpha003.02 (Section 1.13, lines 415-432, black hole scalar coupling)
%         Loop quantum gravity area quantization (Rovelli, Smolin)
% Framework: Genesis | Domain: GR/QM | Status: Theoretical + experimental test
%==============================================================================
\begin{equation}
  T_H = \frac{\hbar \kappa}{2\pi k_B c}
  \left( 1 + \beta \frac{\ell_{\text{node}}^2}{A} + \gamma \frac{\ell_{\text{node}}^4}{A^2} \right)
  \eqtag{G}{GR}{T}
  \label{eq:holo:hawking-temp-modified}
\end{equation}
% Notes:
%   * $\kappa$ = surface gravity at horizon (determines baseline temperature)
%   * $A$ = event horizon area
%   * $\ell_{\text{node}}$ = nodespace characteristic length (discrete spacetime scale)
%   * $\beta, \gamma$ = framework-dependent coefficients (predicted $\beta \sim 1$ to 10, $\gamma \sim 0.1$ to 1)
%   * First term: Standard Hawking temperature $T_H^0 = \hbar \kappa / (2\pi k_B c)$
%   * Second/third terms: Corrections from nodespace discreteness at small $A$
%   * Standard GR: $\beta = \gamma = 0$ (no discrete corrections)
%   * Aether framework: Different correction via ZPE coupling (not discreteness)
% Limiting behavior:
%   * Large $A \gg \ell_{\text{node}}^2$: $T_H \to T_H^0$ (standard GR recovered)
%   * Small $A \sim \ell_{\text{node}}^2$: Discrete corrections become $\mathcal{O}(1)$
% Experimental test:
%   * Measure Hawking radiation spectrum in analog black holes (BEC, optical)
%   * Vary effective horizon area $A_{\text{eff}}$ and test scaling
%   * Expected signature: $\Delta T_H / T_H \sim 10\%$ for $A \sim$ mm$^2$ in BEC
% Relation to entropy:
%   * Via thermodynamic relation: $S = \int (dM / T_H)$ yields modified $S(A)$
%   * Consistent with Genesis entropy formula (Eq.~\ref{eq:theory:genesis-series})
% Dependencies:
%   * Ch01: Fundamental constants ($\hbar$, $k_B$, $c$)
%   * Ch11: Genesis nodespace formulation ($\ell_{\text{node}}$ definition)
%   * Ch18: Framework comparison (distinguishing Hawking temperature predictions)
%   * Ch25: Holographic entropy experimental protocols
%==============================================================================


\begin{table}[htbp]
\centering
\caption{Holographic entropy modifications: Framework predictions for analog systems}
\label{tab:holo:predictions}
\begin{tabular}{@{}llll@{}}
\toprule
\textbf{Property} & \textbf{Standard GR} & \textbf{\aether} & \textbf{\genesis} \\
\midrule
Entropy formula & $S = A / 4G$ & $S = A/4G + \alpha \int \rho_{\text{ZPE}} dV$ & $S = (A/4G)(1 + \beta \ell_{\text{node}}^2 / A)$ \\
Volumetric term & Absent & Present ($\alpha \sim 0.01$ to 0.1) & Absent \\
Discrete corrections & Absent & Absent & Present ($\beta \sim 1$ to 10) \\
$T_H$ modification & None & $\sim 1\%$ shift via ZPE & $\sim 10\%$ shift via discreteness \\
Area scaling & $S \propto A$ & $S \propto A + V$ & $S \propto A (1 + 1/A)$ \\
Analog sensitivity & Baseline & $\Delta S/S \sim 10^{-2}$ & $\Delta S/S \sim 10^{-1}$ \\
\bottomrule
\end{tabular}
\end{table}

Key distinguishing tests:

\begin{enumerate}
  \item \textbf{Area vs. volume scaling}: Vary BEC volume $V$ while holding horizon area $A$ fixed (e.g., pancake vs. cigar geometries). \aether~predicts $S$ changes; others predict no change.

  \item \textbf{Discreteness scaling}: Vary effective Planck scale (by changing BEC parameters $g$, $n$) and test $S \propto 1/A$ term. \genesis~predicts correlation; others predict independence.

  \item \textbf{ZPE coupling}: Introduce external scalar field modulation (as in Ch22 protocols) and measure entropy change. \aether~predicts response; others predict null.
\end{enumerate}

\section{Data Collection and Analysis}

\textbf{Experimental cycle} (BEC system):

\begin{itemize}
  \item \textbf{Repetition rate}: 0.1 Hz to 1 Hz (limited by BEC formation time)
  \item \textbf{Data per cycle}: Density map $n(x,y,z,t)$ with $\sim 10^5$ pixels, 10 time slices
  \item \textbf{Campaign duration}: 100 to 1000 hours for statistical significance
  \item \textbf{Total datasets}: $\sim 10^6$ independent BEC realizations
\end{itemize}

\textbf{Data analysis pipeline}:

\begin{enumerate}
  \item \textbf{Density reconstruction}: Fit $n(x)$ to Thomas-Fermi or Gaussian profiles
  \item \textbf{Velocity extraction}: Compute $v(x) = j(x) / n(x)$ from density current $j$
  \item \textbf{Horizon identification}: Solve $v(x_h) = c_s(x_h)$ for horizon position
  \item \textbf{Correlation functions}: Compute $\langle \delta n \delta n \rangle$ in frequency domain
  \item \textbf{Temperature fit}: Extract $T_H$ from thermal spectrum (Eq.~\ref{eq:protocol:bose-spectrum})
  \item \textbf{Entropy calculation}: Evaluate $S$ from measured $T_H$, $A_{\text{eff}}$, compare to models
\end{enumerate}

\textbf{Systematic uncertainties}:

\begin{itemize}
  \item \textbf{Temperature calibration}: $\pm 5\%$ from imaging resolution and shot noise
  \item \textbf{Horizon position}: $\pm 1$ $\mu$m from density fitting uncertainty
  \item \textbf{ZPE density} (\aether~test): $\pm 10\%$ from Casimir force measurement
  \item \textbf{Effective coupling constants}: $\pm 20\%$ from theoretical modeling uncertainties
\end{itemize}

Total systematic uncertainty on $\Delta S / S$: $\sim 15\%$ to $25\%$, sufficient to distinguish $\sim 10\%$ framework corrections at $> 3\sigma$ confidence.

\section{Interpretation and Validation}

\textbf{Positive detection scenarios}:

\begin{enumerate}
  \item \textbf{\aether~volumetric term observed}:
  \begin{itemize}
    \item Entropy increases with volume at fixed area
    \item Correlation with independently measured $\rho_{\text{ZPE}}$
    \item Magnitude consistent with $\alpha \sim 0.01$ to 0.1
    \item \textbf{Implication}: Scalar-ZPE coupling confirmed, supports crystalline lattice model
  \end{itemize}

  \item \textbf{\genesis~discrete corrections observed}:
  \begin{itemize}
    \item Entropy deviates from $S \propto A$ at small $A$ (sub-mm horizons)
    \item Scaling consistent with $\beta \ell_{\text{node}}^2 / A$ correction
    \item Nodespace scale $\ell_{\text{node}} \sim 10^{-6}$ to $10^{-5}$ m extracted
    \item \textbf{Implication}: Spacetime discreteness at micrometer scales, challenges continuum GR
  \end{itemize}

  \item \textbf{Both corrections observed}:
  \begin{itemize}
    \item Frameworks not mutually exclusive; both may contribute
    \item Requires multi-parameter fit: $S = S_{\text{baseline}} + \Delta S_{\aether} + \Delta S_{\genesis}$
    \item \textbf{Implication}: Hybrid model, unified framework synthesis necessary
  \end{itemize}

  \item \textbf{Null result} (standard GR confirmed):
  \begin{itemize}
    \item Entropy strictly proportional to area with no volume or discrete terms
    \item $|\Delta S| < 0.05 S$ across all tested geometries
    \item \textbf{Implication}: Holographic principle validated, exotic corrections ruled out to $\sim 5\%$
  \end{itemize}
\end{enumerate}

\textbf{Cross-validation with other experiments}:

Results must be consistent with:
\begin{itemize}
  \item Ch22 scalar-ZPE interferometry (if \aether~corrections detected)
  \item Ch24 quantum foam measurements (foam-entropy connection via $\rho_{\text{ZPE}}$)
  \item Ch26 dimensional spectroscopy (if discreteness at $\ell_{\text{node}}$ scales found)
\end{itemize}

Inconsistencies indicate systematic errors or theoretical modeling flaws requiring resolution.

\section{Summary}

This chapter established protocols for testing holographic entropy modifications using analog black hole systems---laboratory-accessible surrogates for astrophysical event horizons. The experimental program distinguishes:

\begin{itemize}
  \item \textbf{Standard GR}: Bekenstein-Hawking entropy $S = A / 4G$ with no corrections
  \item \textbf{\aether}: Volumetric ZPE contributions $\Delta S \sim \alpha \int \rho_{\text{ZPE}} dV$, detectable at $\sim 1\%$ level in BEC systems
  \item \textbf{\genesis}: Discrete nodespace corrections $\Delta S \sim \beta \ell_{\text{node}}^2 / A$, detectable at $\sim 10\%$ level for micrometer-scale horizons
\end{itemize}

The dual-platform approach (BEC acoustic + optical black holes) provides complementary tests:
\begin{itemize}
  \item \textbf{BEC}: Long integration times, precise thermometry, tunable geometries
  \item \textbf{Optical}: High temperatures (easier detection), ultrafast dynamics, room-temperature operation
\end{itemize}

Systematic uncertainties ($\sim 15\%$ to $25\%$) are sufficient to achieve $> 3\sigma$ discrimination between frameworks given predicted $\sim 10\%$ corrections. Integration with scalar-ZPE interferometry (Ch22), quantum foam detection (Ch24), and dimensional spectroscopy (Ch26) will provide comprehensive validation of competing quantum gravity models through independent observational channels.

The next chapter completes the experimental program with dimensional transition spectroscopy, probing the hierarchical structure of hyperdimensional frameworks across energy scales from atomic to collider regimes.

%==============================================================================
% Cross-references:
%   - Ch01: Planck scale, thermodynamic foundations
%   - Ch07: Aether metric perturbations and scalar-ZPE coupling
%   - Ch11: Genesis nodespace formulation
%   - Ch18: Framework experimental distinguishability
%   - Ch21: Unified predictions (holographic entropy)
%   - Ch22: Scalar-ZPE interferometry (ZPE density measurement)
%   - Ch24: Quantum foam detection (foam-entropy connection)
%   - Ch26: Dimensional spectroscopy (discrete scale tests)
%==============================================================================
