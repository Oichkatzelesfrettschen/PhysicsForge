\chapter{Time Crystal Experimental Protocols}\label{ch:time_crystal_protocols}

%==============================================================================
% PROVENANCE:
% Source: Time_Crystal_Experimental_Observations_2016-2025.md (1350 lines)
% Framework integration: Ch17 (framework comparison), Ch19 (unified kernels)
% Theoretical foundation: Ch08 (time crystals), Ch22 (scalar-ZPE protocols)
%==============================================================================

\section{Introduction}

Time crystals represent spontaneous breaking of discrete or continuous time-translation symmetry, manifesting as persistent oscillations in quantum systems\cite{Else2016Floquet,Khemani2016TC}. The \unified{} framework (Ch.~\ref{ch:unified_kernels}) predicts specific couplings between time-crystalline order and scalar-ZPE dynamics through the temporal modulation terms in $\Kunified$:
\begin{equation}
\mathcal{T}_t(t) = \exp\Big[\int_0^t \alpha_{\text{TC}}(s) \phi_{\text{TC}}(s,\omega) \, ds\Big],
\label{eq:ch23:tc_temporal_factor}
\end{equation}
where $\phi_{\text{TC}}(t,\omega)$ denotes the time-crystal order parameter oscillating at drive frequency $\omega$ or its subharmonics. This chapter presents experimental protocols targeting three complementary platforms---trapped ions, superconducting qubits, and nitrogen-vacancy (NV) centers---to validate these predictions and search for scalar-field-mediated effects on time-crystalline coherence.

\paragraph{Experimental Strategy.}
Three apparatus probe different aspects of time-crystal physics:
\begin{itemize}
  \item \textbf{Trapped ion chains}: Discrete time crystals (DTCs) via Floquet driving; long coherence times ($>100$ cycles), controllable disorder and many-body localization\cite{Zhang2017TrappedIons,Kyprianidis2021Prethermal}.
  \item \textbf{Superconducting qubit arrays}: Eigenstate-ordered DTCs with programmable interactions; rapid tuning, large qubit counts (10--100 qubits)\cite{Mi2022TimeCrystal,Patscheider2024DTC}.
  \item \textbf{NV centers in diamond}: Room-temperature operation, ultra-long coherence ($>40$ minutes), solid-state integration\cite{NVtimecrystal2024}, and discrete time quasicrystal phases\cite{DTquasicrystal2025}.
\end{itemize}

\paragraph{Unified Framework Predictions.}
From Ch.~\ref{ch:unified_kernels}, the unified kernel factorization includes a time-crystal contribution at laboratory energy scales:
\begin{equation}
K_{\text{Lab}}(x,y,t) \sim \exp\Big[- \int_0^t \big(g \phi(x,s) \rho_{\text{ZPE}}^2(s) + \alpha_{\text{TC}} \phi_{\text{TC}}(s,\omega)\big) \, ds\Big].
\label{eq:ch23:lab_kernel}
\end{equation}
\textbf{Key testable predictions}:
\begin{enumerate}
  \item Enhanced coherence lifetimes in presence of scalar-ZPE coupling ($\kappa > 0$).
  \item Sideband signatures at frequencies $\omega_{\text{sideband}} = n\omega \pm \omega_{\text{scalar}}$ where $\omega_{\text{scalar}}$ is scalar field resonance.
  \item Modification of subharmonic period from pure integer multiples to fractional values in nodespace-coupled regimes\cite{Patscheider2024DTC}.
  \item Casimir-like force modifications in ion-trap geometries (predicted 5--15\% deviation for fractal electrode surfaces; see Ch.~\ref{ch:scalar_zpe_protocols}).
\end{enumerate}

\section{Theoretical Predictions}\label{sec:ch23:theory}

\subsection{Time Crystal Order Parameter}

For discrete time crystals under periodic drive with period $T$, the order parameter exhibits subharmonic response:
\begin{equation}
\langle \hat{O}(t) \rangle \approx A \cos\Big(\frac{2\pi t}{nT} + \delta\Big),
\quad n = 2, 3, 4, \ldots
\label{eq:ch23:subharmonic_response}
\end{equation}
where $\hat{O}$ is a many-body observable (e.g., collective spin), $A$ is oscillation amplitude, and $\delta$ is phase offset. \textbf{Standard Floquet theory predicts integer $n$}; \unified{} coupling to nodespace origami folds (Ch.~\ref{ch:framework_comparison}) allows \textit{fractional} $n$ in principle, recently observed in Rydberg systems as $n = 2, 3, 4, \ldots, 14$ and fractional values\cite{RydbergFracDTC2024}.

\subsection{Scalar-ZPE Coupling to Time Crystals}

The \aether{} scalar field $\phi(x,t)$ couples nonlinearly to ZPE density $\rho_{\text{ZPE}}$, producing an effective potential well (Ch.~\ref{ch:scalar_zpe_protocols}):
\begin{equation}
V_{\text{eff}}(\phi) = \frac{1}{2} m^2 \phi^2 + \frac{\lambda}{4} \phi^4 + g \phi \rho_{\text{ZPE}}^2.
\label{eq:ch23:scalar_potential}
\end{equation}
In a time-crystal experiment, $\rho_{\text{ZPE}}$ oscillates with the system:
\begin{equation}
\rho_{\text{ZPE}}(t) = \rho_0 \Big[1 + \epsilon \cos\Big(\frac{2\pi t}{nT}\Big)\Big],
\label{eq:ch23:zpe_oscillation}
\end{equation}
where $\epsilon \ll 1$ is modulation depth. This back-action generates sidebands in the power spectrum at $\omega \pm \omega_{\text{TC}}$, with amplitude proportional to coupling $g$ and ZPE fluctuation strength $\epsilon$.

\paragraph{Prediction: Enhanced Coherence.}
If scalar-ZPE coupling stabilizes a quasi-classical potential well (Ch.~\ref{ch:framework_comparison}, 97\% compatibility finding), time-crystal dephasing rate $\Gamma_{\text{dephase}}$ decreases:
\begin{equation}
\Gamma_{\text{dephase}} = \Gamma_0 - \beta \kappa \rho_{\text{ZPE}},
\quad \kappa > 0,
\label{eq:ch23:coherence_enhancement}
\end{equation}
leading to \textit{anomalously long coherence times}---consistent with NV center observations ($>40$ minutes)\cite{NVtimecrystal2024}.

\section{Trapped Ion Platform}\label{sec:ch23:ions}

\subsection{System Specifications}

\paragraph{Ion Chain Configuration.}
\begin{itemize}
  \item \textbf{Species}: $^{171}$Yb$^+$ (hyperfine qubit) or $^{9}$Be$^+$ (Zeeman qubit).
  \item \textbf{Trap type}: Linear Paul trap, RF frequency $\Omega_{\text{RF}} \approx 2\pi \times 10$ MHz.
  \item \textbf{Ion number}: 10--25 ions (scalable to 50+ with segmented traps).
  \item \textbf{Temperature}: Doppler cooling to $\sim 1$ mK, sideband cooling to motional ground state ($\bar{n} < 0.1$ phonons).
  \item \textbf{Motional mode splitting}: Axial center-of-mass mode $\omega_{\text{COM}} \approx 2\pi \times 1$ MHz.
\end{itemize}

\paragraph{Floquet Drive Implementation.}
Periodic drive applied via global microwave or optical pulses:
\begin{equation}
H_{\text{drive}}(t) = \sum_{j=1}^N \Big[\frac{\Omega}{2} \sigma_j^x + \frac{\Delta(t)}{2} \sigma_j^z\Big],
\quad \Delta(t) = \Delta_0 + \Delta_1 \cos(\omega t),
\label{eq:ch23:ion_hamiltonian}
\end{equation}
where $\sigma_j^{x,z}$ are Pauli operators on ion $j$, $\Omega$ is Rabi frequency, and $\Delta(t)$ is time-dependent detuning with period $T = 2\pi/\omega$.

\paragraph{Disorder Engineering.}
Many-body localization requires disorder. Two approaches:
\begin{enumerate}
  \item \textbf{Magnetic field gradients}: Apply inhomogeneous Zeeman shifts $\delta_j$ via external coils, giving $\Delta_j = \Delta_0 + \delta_j$ with $\delta_j$ drawn from Gaussian distribution $\mathcal{N}(0, W^2)$, disorder strength $W \sim 0.5$--$2.0$ in units of $\Omega$.
  \item \textbf{Trap anharmonicity}: Natural disorder from non-uniform ion spacing in anharmonic potential (prethermal regime, no intentional disorder)\cite{Kyprianidis2021Prethermal}.
\end{enumerate}

\subsection{Measurement Protocol}

\paragraph{State Preparation.}
\begin{enumerate}
  \item Initialize all ions in $|\downarrow\rangle^{\otimes N}$ (ground state) via optical pumping.
  \item Apply $\pi/2$ pulse to prepare $|+\rangle^{\otimes N}$ superposition.
  \item Optional: Prepare domain-wall initial state for enhanced DTC signatures\cite{Mi2022TimeCrystal}.
\end{enumerate}

\paragraph{Floquet Evolution.}
Apply $M$ cycles of Floquet drive ($M = 100$--$500$ cycles):
\begin{equation}
U_F = \exp\Big[-i \int_0^T H_{\text{drive}}(t) \, dt\Big].
\label{eq:ch23:floquet_unitary}
\end{equation}
Monitor collective observable $\langle \hat{S}^x \rangle = \frac{1}{N}\sum_j \langle \sigma_j^x \rangle$ every $k$ cycles ($k = 1$ or $2$).

\paragraph{Readout.}
After $M$ cycles, apply $\pi/2$ pulse to rotate to measurement basis, then perform fluorescence detection:
\begin{itemize}
  \item Bright ion ($|\downarrow\rangle$): Scatters photons.
  \item Dark ion ($|\uparrow\rangle$): No fluorescence.
  \item Single-shot fidelity $>99.9\%$.
  \item Repeat 200--500 shots per data point for statistics.
\end{itemize}

\paragraph{Time Crystal Signature.}
Plot $\langle \hat{S}^x(m) \rangle$ vs. cycle number $m$. \textbf{DTC phase}: Observe period-$2T$ oscillations (subharmonic response) persisting for $M > 100$ cycles despite disorder-induced dephasing. \textbf{Thermal phase}: Rapid decay to zero within $\sim 10$ cycles.

\subsection{Coherence Time Measurements}

\paragraph{Rabi Oscillation Baseline.}
Before Floquet driving, measure single-qubit coherence via Rabi sequence:
\begin{equation}
\text{Rabi: } |\downarrow\rangle \xrightarrow{\text{pulse}(\Omega, t_{\text{pulse}})} \text{measure}
\label{eq:ch23:rabi}
\end{equation}
Sweep $t_{\text{pulse}}$ from 0 to $10/\Omega$, fit oscillation envelope to $A e^{-t/T_2^*}$. Typical $T_2^* \approx 1$--$10$ ms (magnetic noise limited).

\paragraph{Ramsey Interferometry.}
Measure dephasing rate via:
\begin{equation}
\text{Ramsey: } |\downarrow\rangle \xrightarrow{\pi/2} \text{free evolution } \tau \xrightarrow{\pi/2} \text{measure}
\label{eq:ch23:ramsey}
\end{equation}
Fit contrast decay vs. $\tau$ to extract $T_2 \approx 10$--$100$ ms (motional heating and magnetic noise).

\paragraph{Spin-Echo Refocusing.}
Apply $\pi$ pulse at $\tau/2$ to cancel low-frequency noise:
\begin{equation}
\text{Echo: } |\downarrow\rangle \xrightarrow{\pi/2} \tau/2 \xrightarrow{\pi} \tau/2 \xrightarrow{\pi/2} \text{measure}
\label{eq:ch23:echo}
\end{equation}
Improved coherence $T_{\text{echo}} \approx 100$--$1000$ ms. Compare DTC lifetime to $T_{\text{echo}}$: if DTC survives $M T > T_{\text{echo}}$, evidence for collective protection.

\paragraph{Predicted Scalar-ZPE Enhancement.}
If Eq.~(\ref{eq:ch23:coherence_enhancement}) holds with $\kappa \rho_{\text{ZPE}} \sim 0.1 \Gamma_0$, expect \textit{10\% increase in DTC lifetime} relative to standard theoretical predictions. Requires comparison runs with varied ZPE environment (e.g., different trap geometries, varying Casimir boundary conditions).

\section{Superconducting Qubit Platform}\label{sec:ch23:qubits}

\subsection{System Specifications}

\paragraph{Processor Architecture.}
\begin{itemize}
  \item \textbf{Qubit type}: Transmon qubits with fixed frequency (capacitively shunted charge qubits).
  \item \textbf{Array size}: 20--100 qubits in 2D grid (Google Sycamore, IBM Quantum, Rigetti).
  \item \textbf{Connectivity}: Tunable CPHASE or iSWAP gates between nearest neighbors.
  \item \textbf{Coherence times}: $T_1 \approx 50$--$150$ $\mu$s (energy relaxation), $T_2 \approx 30$--$100$ $\mu$s (dephasing).
  \item \textbf{Gate fidelity}: 1-qubit $>99.9\%$, 2-qubit $>99\%$.
  \item \textbf{Temperature}: Dilution refrigerator, $T \approx 20$ mK.
\end{itemize}

\paragraph{Floquet Circuit Design.}
Implement Floquet Hamiltonian via gate sequence repeated every period $T$:
\begin{equation}
H_F(t) = \sum_{\langle i,j \rangle} J_{ij} \sigma_i^z \sigma_j^z + \sum_i \big[h_i^x(t) \sigma_i^x + h_i^z(t) \sigma_i^z\big],
\label{eq:ch23:qubit_hamiltonian}
\end{equation}
where $J_{ij}$ is Ising coupling (tunable via CPHASE gate strength), $h_i^x(t)$ is transverse field (single-qubit rotations), and $h_i^z(t)$ is longitudinal disorder (engineered via detuning).

\paragraph{Eigenstate Order Protocol.}
Following Mi et al.\cite{Mi2022TimeCrystal}, initialize qubits in \textit{random} computational basis states (sample entire many-body spectrum), apply Floquet evolution, and measure return probability. \textbf{DTC signature}: All eigenstates exhibit coherent oscillations at period $2T$, demonstrating eigenstate order throughout spectrum.

\subsection{Measurement Protocol}

\paragraph{State Tomography.}
After $M$ Floquet cycles:
\begin{enumerate}
  \item Randomly choose measurement basis: $\{X, Y, Z\}^{\otimes N}$.
  \item Perform simultaneous readout of all qubits (multiplexed resonator readout).
  \item Reconstruct density matrix $\rho(M)$ via maximum likelihood tomography (feasible for $N \lesssim 10$ qubits; partial tomography for larger $N$).
\end{enumerate}

\paragraph{Time-Reversal Test.}
Discriminate thermalization vs. decoherence:
\begin{enumerate}
  \item Evolve forward for $M$ cycles: $|\psi_0\rangle \to |\psi_M\rangle$.
  \item Apply time-reversal unitary $U_F^\dagger$ for $M$ cycles: $|\psi_M\rangle \to |\psi_{\text{rev}}\rangle$.
  \item Measure fidelity $F = |\langle \psi_0 | \psi_{\text{rev}} \rangle|^2$. If $F \to 0$: thermalization. If $F \approx e^{-2M/T_{\text{coh}}}$: pure decoherence.
\end{enumerate}

\paragraph{Correlation Function Measurement.}
Compute two-time correlator:
\begin{equation}
C(t, t+\tau) = \langle \hat{O}(t) \hat{O}(t+\tau) \rangle,
\label{eq:ch23:correlator}
\end{equation}
using ancilla-assisted measurement\cite{Mi2022TimeCrystal}. \textbf{DTC signature}: $C(t, t+2T) \approx C(t,t)$ (rigid period-doubling), whereas thermal phase shows decay $C \sim e^{-\tau/\tau_{\text{th}}}$.

\subsection{Scalar-Field Sideband Search}

\paragraph{Power Spectral Density Analysis.}
Fourier-transform time-series data $\langle \hat{S}^x(m) \rangle$:
\begin{equation}
S(\omega) = \Big|\int_0^{M T} \langle \hat{S}^x(t) \rangle e^{i\omega t} \, dt\Big|^2.
\label{eq:ch23:psd}
\end{equation}
\textbf{Baseline expectation}: Peak at $\omega = \pi/T$ (period-$2T$). \textbf{Scalar-ZPE prediction}: Additional sidebands at $\omega = \pi/T \pm \omega_{\text{scalar}}$, where $\omega_{\text{scalar}} \sim 2\pi \times (1$--$10)$ kHz is scalar field resonance (from Ch.~\ref{ch:scalar_zpe_protocols}, Fabry-Perot measurements).

\paragraph{Cross-Platform Correlation.}
Operate ion trap and qubit processor simultaneously (if feasible at shared facility). Search for correlated sideband frequencies across platforms---strong evidence for environmental scalar field rather than platform-specific artifacts.

\section{NV Center Platform}\label{sec:ch23:nv}

\subsection{System Specifications}

\paragraph{Diamond Sample.}
\begin{itemize}
  \item \textbf{NV density}: $10^{10}$--$10^{14}$ cm$^{-3}$ (trade-off: higher density increases signal but decreases coherence due to dipolar coupling).
  \item \textbf{Isotopic purity}: $^{12}$C enriched ($>99.99\%$) to suppress nuclear spin bath (natural $^{13}$C concentration $1.1\%$ limits $T_2 \approx 600$ $\mu$s).
  \item \textbf{Sample geometry}: Bulk diamond (3 mm $\times$ 3 mm $\times$ 0.5 mm) or nanopillar arrays.
\end{itemize}

\paragraph{Control and Readout.}
\begin{itemize}
  \item \textbf{Optical initialization}: 532 nm laser pumps NV$^-$ into $m_s = 0$ ground state ($>90\%$ fidelity).
  \item \textbf{Microwave control}: Resonant MW pulses at $D_{gs} \approx 2.87$ GHz (zero-field splitting) for spin rotations.
  \item \textbf{Optical readout}: Spin-dependent fluorescence (photoluminescence at 637--800 nm); $m_s = 0$ bright, $m_s = \pm 1$ dark.
  \item \textbf{Temperature}: Room temperature (major advantage vs. cryogenic platforms).
\end{itemize}

\paragraph{Time Quasicrystal Phase.}
Recent observation\cite{DTquasicrystal2025} of discrete time quasicrystals in NV ensembles: multiple incommensurate subharmonic frequencies. Protocol:
\begin{enumerate}
  \item Apply two-tone Floquet drive: $H(t) = H_1 \cos(\omega_1 t) + H_2 \cos(\omega_2 t)$, with $\omega_1/\omega_2$ irrational (e.g., golden ratio $\phi = (1+\sqrt{5})/2$).
  \item Measure $\langle \hat{S}^z(t) \rangle$ over long times ($10^4$--$10^5$ cycles).
  \item Fourier analysis reveals peaks at $\omega = m\omega_1 + n\omega_2$ with $m, n \in \mathbb{Z}$ (quasiperiodic structure).
\end{enumerate}

\subsection{Measurement Protocol}

\paragraph{Pulsed ODMR Sequence.}
Optically detected magnetic resonance (ODMR) with pulsed MW:
\begin{enumerate}
  \item Laser pulse (532 nm, 1 $\mu$s) initializes to $m_s = 0$.
  \item Wait 300 ns (excited state decay).
  \item Apply MW pulse sequence (Floquet drive or dynamical decoupling).
  \item Readout laser pulse (532 nm, 300 ns).
  \item Collect fluorescence photons via APD (single-photon counting).
\end{enumerate}
Repeat $10^4$--$10^6$ times per data point.

\paragraph{Dynamical Decoupling Baseline.}
Before Floquet experiments, characterize intrinsic coherence via XY-8 sequence:
\begin{equation}
\text{XY-8: } \pi/2 \text{--} [\tau \text{--} \pi_x \text{--} 2\tau \text{--} \pi_y \text{--} 2\tau \text{--} \cdots]_8 \text{--} \pi/2
\label{eq:ch23:xy8}
\end{equation}
Extract $T_2$ by fitting decay vs. total evolution time $T_{\text{total}} = 16\tau$. Isotopically pure diamond: $T_2 \approx 1$--$10$ ms. Nuclear-spin-free environment: $T_2 > 1$ s.

\paragraph{Time Crystal Protocol.}
\begin{enumerate}
  \item Initialize $m_s = 0$.
  \item Apply periodic MW drive (period $T = 2$--$20$ $\mu$s) for $M$ cycles.
  \item Readout $\langle \hat{S}^z \rangle$.
  \item Vary $M$ from 1 to 500 to map out oscillation envelope.
\end{enumerate}
\textbf{DTC signature}: Period-$2T$ oscillations persist for $M > 100$ despite $T \ll T_2$ (drive faster than intrinsic decoherence).

\subsection{Scalar-ZPE Coupling Search}

\paragraph{Enhanced Coherence Signature.}
Compare $T_2^{\text{DTC}}$ (effective coherence time of DTC oscillations) to baseline $T_2$ (XY-8 or Ramsey). Prediction: $T_2^{\text{DTC}}/T_2 > 1$ if scalar-ZPE coupling stabilizes collective time-crystal state. Room-temperature operation and long intrinsic $T_2$ make NV centers ideal for detecting \textit{anomalous coherence extension} (consistent with 40-minute record\cite{NVtimecrystal2024}).

\paragraph{Casimir Geometry Modulation.}
Embed diamond sample between fractal-patterned metal plates (similar to Ch.~\ref{ch:scalar_zpe_protocols}, Casimir experiment). Vary plate separation $d = 100$ nm--$1$ $\mu$m and measure DTC coherence vs. $d$. \textbf{Prediction}: If ZPE-scalar coupling affects time crystals, observe modulation $T_2^{\text{DTC}}(d)$ correlated with Casimir force $F(d)$ (both probe vacuum fluctuations).

\section{Data Analysis and Validation}\label{sec:ch23:data}

\subsection{Statistical Framework}

\paragraph{Fitting Subharmonic Oscillations.}
Model time-crystal signal as damped sinusoid:
\begin{equation}
\langle \hat{O}(m) \rangle = A e^{-m/\tau_{\text{DTC}}} \cos\Big(\frac{2\pi m}{n} + \delta\Big) + O_0,
\label{eq:ch23:fit_function}
\end{equation}
where $m$ is cycle number, $A$ is initial amplitude, $\tau_{\text{DTC}}$ is DTC lifetime (in cycles), $n$ is subharmonic order, $\delta$ is phase offset, and $O_0$ is thermal baseline. Fit via nonlinear least-squares (Levenberg-Marquardt).

\paragraph{Uncertainty Quantification.}
\begin{itemize}
  \item Bootstrapping: Resample data with replacement ($10^3$ bootstrap samples), refit each time, compute 95\% confidence intervals on $\tau_{\text{DTC}}$ and $n$.
  \item Bayesian inference: Prior on $n$ (integer or fractional), likelihood from Gaussian measurement noise, compute posterior $p(n, \tau_{\text{DTC}} | \text{data})$ via MCMC\cite{Mi2022TimeCrystal}.
\end{itemize}

\paragraph{Sideband Detection Threshold.}
For power spectral density $S(\omega)$, define sideband peak as statistically significant if:
\begin{equation}
\frac{S(\omega_{\text{peak}})}{S_{\text{noise}}} > 5,
\label{eq:ch23:sideband_threshold}
\end{equation}
where $S_{\text{noise}}$ is median spectral density in off-resonance region. Requires signal integration time sufficient for $\text{SNR} > 5$ (typically $10^4$--$10^5$ measurements per frequency bin).

\subsection{Cross-Platform Consistency Checks}

\paragraph{Coherence Time Scaling.}
Plot $\tau_{\text{DTC}}$ vs. intrinsic $T_2$ for all three platforms (ions, qubits, NV centers). Standard Floquet theory predicts $\tau_{\text{DTC}} \propto T_2$. \textbf{Scalar-ZPE hypothesis}: Systematic \textit{upward deviation} from linear scaling if environmental scalar field provides collective stabilization (Ch.~\ref{ch:framework_comparison}, 97\% compatibility finding).

\paragraph{Sideband Frequency Universality.}
If sidebands at $\omega_{\text{scalar}}$ appear across all platforms, strong evidence for external scalar field rather than platform-specific systematics. Require $|\omega_{\text{ion}} - \omega_{\text{qubit}}|/\omega_{\text{ion}} < 5\%$.

\paragraph{Environmental Dependence.}
Vary experimental conditions and check consistency:
\begin{itemize}
  \item \textbf{Magnetic field}: Sweep $B = 0$--$500$ G, verify DTC signature independent of Zeeman shifts.
  \item \textbf{Drive frequency}: Vary $\omega$ over factor of 2--5, check $\tau_{\text{DTC}}$ scaling.
  \item \textbf{Temperature} (where applicable): NV centers operate room temperature; verify DTC at 77 K (liquid N$_2$) and 4 K (liquid He) as control.
\end{itemize}

\section{Experimental Roadmap}\label{sec:ch23:roadmap}

\paragraph{Phase 1 (Months 1--4): Platform Commissioning.}
\begin{itemize}
  \item Commission ion trap: Achieve $T_2 > 10$ ms, demonstrate Floquet DTC with $\tau_{\text{DTC}} > 100$ cycles baseline.
  \item Commission qubit processor: Calibrate 20-qubit array, measure $T_1, T_2$, implement eigenstate order protocol.
  \item Commission NV diamond: Characterize $T_2$ via XY-8, verify ODMR contrast $>30\%$.
\end{itemize}

\paragraph{Phase 2 (Months 5--8): Baseline DTC Characterization.}
\begin{itemize}
  \item Map DTC phase diagram: Vary disorder strength $W$, drive amplitude $\Omega$, frequency $\omega$.
  \item Measure coherence times $\tau_{\text{DTC}}$ and compare to theoretical predictions (Floquet MBL, prethermalization).
  \item Document all control systematics (laser intensity noise, MW phase drift, magnetic field fluctuations).
\end{itemize}

\paragraph{Phase 3 (Months 9--12): Scalar-ZPE Coupling Tests.}
\begin{itemize}
  \item \textbf{Sideband search}: Collect high-statistics time-series data ($10^5$ shots/point), compute PSD, search for peaks at $\omega_{\text{scalar}}$.
  \item \textbf{Coherence enhancement}: Compare $\tau_{\text{DTC}}$ in environments with modified ZPE (e.g., fractal Casimir plates, varied trap geometries).
  \item \textbf{Cross-platform correlation}: Operate ion trap and NV setup simultaneously, check for correlated fluctuations in $\langle \hat{O}(t) \rangle$.
  \item \textbf{Time quasicrystal validation}: Implement two-tone drive on NV centers, verify quasiperiodic spectral structure; test for nodespace-origami signatures predicted by \genesis{} framework\cite{DTquasicrystal2025}.
\end{itemize}

\paragraph{Success Criteria.}
\begin{enumerate}
  \item Observation of DTC oscillations with $\tau_{\text{DTC}} > 100$ cycles on all three platforms ($5\sigma$ significance).
  \item Detection of sidebands at frequency $\omega_{\text{scalar}} \pm 5\%$ consistent across platforms (if present; null result also informative).
  \item Measurement of $\tau_{\text{DTC}}$ vs. $T_2$ scaling; statistical test for deviation from linearity at 95\% confidence.
  \item Documentation of environmental dependencies sufficient to distinguish scalar-ZPE effects from systematics.
\end{enumerate}

\section{Summary and Forward References}\label{sec:ch23:summary}

This chapter presented comprehensive experimental protocols for time-crystal validation across three complementary platforms: trapped ions (long coherence, controllable disorder), superconducting qubits (programmable interactions, eigenstate order), and NV centers (room temperature, ultra-long $T_2$). The \unified{} framework (Ch.~\ref{ch:unified_kernels}) predicts:
\begin{enumerate}
  \item Enhanced coherence lifetimes via scalar-ZPE stabilization (Eq.~\ref{eq:ch23:coherence_enhancement}).
  \item Sideband signatures at $\omega_{\text{TC}} \pm \omega_{\text{scalar}}$ from back-action coupling.
  \item Fractional subharmonic orders in nodespace-coupled regimes.
  \item Casimir-geometry-dependent DTC properties.
\end{enumerate}

\textbf{Connection to scalar-ZPE protocols}: Ch.~\ref{ch:scalar_zpe_protocols} Fabry-Perot measurements yield $\omega_{\text{scalar}}$, which sets target frequency for time-crystal sideband search. Combined scalar-interferometry + time-crystal experiments provide multi-observable validation.

\textbf{Outstanding tasks}:
\begin{itemize}
  \item Attach primary literature references (Zhang et al. 2017, Mi et al. 2022, Kyprianidis et al. 2021, DTquasicrystal 2025)\cite{Zhang2017TrappedIons,Mi2022TimeCrystal,Kyprianidis2021Prethermal,DTquasicrystal2025,NVtimecrystal2024}.
  \item Derive expected sideband amplitudes from $\Kunified$ perturbative expansion.
  \item Cross-reference with Ch.~\ref{ch:scalar_zpe_protocols} (ZPE coherence detection) for gradiometry correlations.
\end{itemize}

Chapter~\ref{ch:scalar_zpe_protocols} extends these protocols to ZPE coherence detection via superconducting gradiometry, completing the suite of experimental tests for the \unified{} framework.
