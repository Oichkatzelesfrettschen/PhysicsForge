%==============================================================================
% COMMON PREAMBLE: Shared Packages and Settings
% Used by all papers in the PhysicsForge series
%==============================================================================

% Essential Encoding and Fonts
\usepackage[utf8]{inputenc}
\usepackage[T1]{fontenc}
\usepackage{lmodern}

% Page Geometry (with wide margins for marginal notes)
\usepackage[margin=1in,marginparwidth=1.5in]{geometry}

%------------------------------------------------------------------------------
% Mathematics Packages
%------------------------------------------------------------------------------
\usepackage{amsmath}        % AMS mathematical facilities
\usepackage{amssymb}        % AMS mathematical symbols
\usepackage{amsfonts}       % AMS fonts
\usepackage{amsthm}         % Theorem environments
\usepackage{mathtools}      % Extensions to amsmath
\usepackage{physics}        % Physics notation (bra-ket, derivatives, etc.)

%------------------------------------------------------------------------------
% Graphics and Visualization
%------------------------------------------------------------------------------
\usepackage{graphicx}       % Include external graphics
\usepackage{xcolor}         % Color support

% TikZ and PGFPlots for high-quality diagrams
\usepackage{tikz}
\usetikzlibrary{calc,arrows.meta,patterns,shapes,decorations.pathreplacing,positioning,3d}
\usepackage{pgfplots}
\pgfplotsset{compat=1.18}

%------------------------------------------------------------------------------
% Tables and Tabular Environments
%------------------------------------------------------------------------------
\usepackage{booktabs}       % Professional-quality tables
\usepackage{array}          % Extended array/tabular
\usepackage{tabularx}       % Auto-sizing table columns
\usepackage{longtable}      % Multi-page tables
\usepackage{multirow}       % Multi-row cells

%------------------------------------------------------------------------------
% Marginal Notes (Lions Commentary Style)
%------------------------------------------------------------------------------
\usepackage{marginnote}
\renewcommand*{\marginfont}{\footnotesize\sffamily}
\usepackage{sidenotes}

%------------------------------------------------------------------------------
% Cross-References and Hyperlinks
%------------------------------------------------------------------------------
\usepackage[colorlinks=true,linkcolor=blue,citecolor=blue,urlcolor=blue]{hyperref}
\usepackage{cleveref}       % Intelligent cross-referencing

%------------------------------------------------------------------------------
% Bibliography
%------------------------------------------------------------------------------
\usepackage{natbib}
\bibliographystyle{plainnat}

%------------------------------------------------------------------------------
% Index
%------------------------------------------------------------------------------
\usepackage{makeidx}
\makeindex

%------------------------------------------------------------------------------
% Code Listings (for algorithms and pseudocode)
%------------------------------------------------------------------------------
\usepackage{listings}
\lstset{
  basicstyle=\ttfamily\small,
  keywordstyle=\color{blue},
  commentstyle=\color{green!60!black},
  stringstyle=\color{red},
  showstringspaces=false,
  breaklines=true,
  frame=single,
  numbers=left,
  numberstyle=\tiny\color{gray}
}

%------------------------------------------------------------------------------
% Units and Physical Quantities
%------------------------------------------------------------------------------
\usepackage{siunitx}
\sisetup{
  detect-all,
  separate-uncertainty = true,
  multi-part-units = single
}

%------------------------------------------------------------------------------
% Theorem-like Environments
%------------------------------------------------------------------------------
\theoremstyle{plain}
\newtheorem{theorem}{Theorem}[chapter]
\newtheorem{lemma}[theorem]{Lemma}
\newtheorem{proposition}[theorem]{Proposition}
\newtheorem{corollary}[theorem]{Corollary}

\theoremstyle{definition}
\newtheorem{definition}[theorem]{Definition}
\newtheorem{example}[theorem]{Example}
\newtheorem{remark}[theorem]{Remark}

%------------------------------------------------------------------------------
% Caption Formatting
%------------------------------------------------------------------------------
\usepackage{caption}
\usepackage{subcaption}
\captionsetup{
  font=small,
  labelfont=bf,
  format=plain,
  justification=justified,
  singlelinecheck=false
}

%------------------------------------------------------------------------------
% Headers and Footers
%------------------------------------------------------------------------------
\usepackage{fancyhdr}
\pagestyle{fancy}
\fancyhf{}
\fancyhead[LE,RO]{\thepage}
\fancyhead[RE]{\nouppercase{\leftmark}}
\fancyhead[LO]{\nouppercase{\rightmark}}
\renewcommand{\headrulewidth}{0.4pt}

%------------------------------------------------------------------------------
% Section Formatting
%------------------------------------------------------------------------------
\usepackage{titlesec}
\titleformat{\chapter}[display]
  {\normalfont\huge\bfseries}{\chaptertitlename\ \thechapter}{20pt}{\Huge}
\titlespacing*{\chapter}{0pt}{-20pt}{40pt}

%------------------------------------------------------------------------------
% Miscellaneous
%------------------------------------------------------------------------------
\usepackage{lipsum}         % Dummy text (for development)
\usepackage{enumitem}       % Customizable lists
\setlist{nosep}             % Compact lists

%------------------------------------------------------------------------------
% PDF Metadata (to be customized per paper)
%------------------------------------------------------------------------------
\hypersetup{
  pdfauthor={PhysicsForge Collaboration},
  pdfsubject={Unified Field Theories and Advanced Physics},
  pdfkeywords={scalar fields, quantum vacuum, E8, exceptional algebras, unification},
  pdfcreator={LaTeX with hyperref}
}

%==============================================================================
% End of Common Preamble
%==============================================================================
