%==============================================================================
% Figure: RCA Problem-Solution Flowchart Template
% Purpose: Reusable Root Cause Analysis flowchart for chapter openings
% Chapter: Universal template for all chapters
% Type: Flowchart / Process diagram
%==============================================================================

\begin{figure}[htbp]
  \centering
  \begin{tikzpicture}[
    node distance=2.5cm,
    auto,
    process/.style={
      rectangle,
      rounded corners,
      minimum width=3.5cm,
      minimum height=1.2cm,
      text centered,
      draw=black,
      line width=0.8pt,
      fill=blue!20,
      font=\large
    },
    decision/.style={
      diamond,
      minimum width=3cm,
      minimum height=1.2cm,
      text centered,
      draw=black,
      line width=0.8pt,
      fill=yellow!20,
      font=\normalsize
    },
    arrow/.style={
      thick,
      ->,
      >=stealth,
      line width=1pt
    },
    annotation/.style={
      font=\small,
      text width=2.5cm,
      align=center
    }
  ]

    % Main process nodes
    \node (problem) [process, fill=red!20] {Problem Statement};
    \node (analysis) [process, below of=problem] {Data Analysis};
    \node (root) [process, below of=analysis, fill=orange!20] {Root Cause Identification};
    \node (solution) [process, below of=root, fill=green!20] {Proposed Solution};
    \node (validation) [process, below of=solution, fill=blue!30] {Validation Protocol};

    % Decision point
    \node (check) [decision, right of=validation, node distance=5cm] {Validated?};

    % Success node
    \node (success) [process, below of=validation, fill=green!40] {Implementation};

    % Arrows connecting main flow
    \draw [arrow] (problem) -- node[right, annotation] {Observe symptoms} (analysis);
    \draw [arrow] (analysis) -- node[right, annotation] {Investigate mechanisms} (root);
    \draw [arrow] (root) -- node[right, annotation] {Design intervention} (solution);
    \draw [arrow] (solution) -- node[right, annotation] {Test predictions} (validation);
    \draw [arrow] (validation) -- (check);
    \draw [arrow] (check) -- node[above] {Yes} (success);

    % Feedback loop
    \draw [arrow, dashed] (check) -- node[above, annotation] {No: Refine} ++(0,3.5) -| (root);

    % Side annotations
    \node[annotation, left of=problem, node distance=5cm, text=blue!60!black]
      {RCA Phase 1:\\Define scope};
    \node[annotation, left of=analysis, node distance=5cm, text=blue!60!black]
      {RCA Phase 2:\\Gather evidence};
    \node[annotation, left of=root, node distance=5cm, text=blue!60!black]
      {RCA Phase 3:\\Identify cause};
    \node[annotation, left of=solution, node distance=5cm, text=blue!60!black]
      {RCA Phase 4:\\Develop fix};
    \node[annotation, left of=validation, node distance=5cm, text=blue!60!black]
      {RCA Phase 5:\\Verify outcome};

  \end{tikzpicture}

  \caption{Root Cause Analysis (RCA) flowchart template. This systematic approach guides problem-solving throughout the monograph: identify the problem, analyze data, determine root causes, propose solutions, and validate through experimental protocols. Dashed feedback loop represents iterative refinement when initial validation fails.}
  \label{fig:rca-template}
\end{figure}
