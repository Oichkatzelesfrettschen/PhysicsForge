% Riemann Tensor: Geometric Meaning via Parallel Transport Loop
% Shows how curvature manifests as rotation after closed path

\begin{tikzpicture}[scale=3]

% Draw curved surface (2D projection of curved space)
\draw[thick, fill=blue!5] plot[smooth cycle, tension=0.8] coordinates {
    (0,0) (1.5,0.2) (2,1) (1.5,1.8) (0.5,2) (-0.3,1) (0,0)
};

% Grid on surface to show curvature
\foreach \x in {0.2,0.4,...,1.6} {
    \draw[gray!40, thin] plot[smooth, tension=0.7] coordinates {
        (\x-0.2,0.1) (\x,0.5) (\x,1.5) (\x+0.1,1.9)
    };
}
\foreach \y in {0.3,0.6,...,1.5} {
    \draw[gray!40, thin] plot[smooth, tension=0.7] coordinates {
        (0,\y) (0.7,\y) (1.4,\y) (2,\y+0.1)
    };
}

% Closed path (small loop)
\coordinate (P1) at (0.8,0.8);
\coordinate (P2) at (1.2,0.8);
\coordinate (P3) at (1.2,1.2);
\coordinate (P4) at (0.8,1.2);

% Draw the loop with directional arrows
\draw[->, very thick, red] (P1) -- node[below, font=\tiny] {$\Delta x$} (P2);
\draw[->, very thick, red] (P2) -- node[right, font=\tiny] {$\Delta y$} (P3);
\draw[->, very thick, red] (P3) -- node[above, font=\tiny] {$-\Delta x$} (P4);
\draw[->, very thick, red] (P4) -- node[left, font=\tiny] {$-\Delta y$} (P1);

% Vector at start
\draw[->, very thick, blue, line width=1.5pt] (P1) -- ++(0,0.3);
\node[left, font=\small, text=blue] at (P1) {$V_0$};

% Vector after loop (rotated slightly due to curvature)
\draw[->, very thick, purple, line width=1.5pt] (P1) -- ++(-0.08,0.29);
\node[right, font=\small, text=purple] at ($(P1)+(0.1,0.15)$) {$V_0 + \Delta V$};

% Rotation angle annotation
\draw[<->, thick, orange] ($(P1)+(0,0.3)$) arc (90:105:0.3);
\node[above, font=\tiny, text=orange] at ($(P1)+(0,0.35)$) {$\Delta\theta$};

% Formula box showing the relationship
\node[draw, thick, fill=white, align=left, font=\small] at (2.8,1.5) {
    \textbf{Riemann Curvature:}\\[2pt]
    $\Delta V^\rho = R^\rho_{\sigma\mu\nu} V^\sigma \Delta x^\mu \Delta x^\nu$\\[4pt]
    $\Delta\theta \propto R^\rho_{\sigma\mu\nu} \cdot \text{Area}$\\[4pt]
    For small loop:\\
    $\text{Area} = \Delta x^\mu \Delta x^\nu$
};

% Physical interpretation box
\node[draw, thick, fill=yellow!10, align=left, font=\small] at (2.8,0.5) {
    \textbf{Physical Meaning:}\\[2pt]
    • Parallel transport vector\\
    \quad around closed loop\\
    • Returns rotated/changed\\
    • Rotation $\propto$ curvature\\
    • Riemann tensor $R^\rho_{\sigma\mu\nu}$\\
    \quad measures this effect
};

% Symmetries annotation
\node[draw, thick, fill=green!10, align=left, font=\tiny] at (2.8,-0.4) {
    \textbf{Riemann Tensor Properties:}\\
    • 20 independent components (4D)\\
    • Antisymmetric: $R^\rho_{\sigma\mu\nu} = -R^\rho_{\sigma\nu\mu}$\\
    • Bianchi identity: $\nabla_{[\lambda}R_{\rho\sigma]\mu\nu} = 0$\\
    • Symmetry: $R_{\mu\nu\rho\sigma} = R_{\rho\sigma\mu\nu}$\\
    • Units: $[\text{length}]^{-2}$
};

% Coordinate labels
\node[font=\small] at (0.2,0.2) {$x^\mu$};
\node[font=\small] at (1.9,0.3) {$x^\mu + \Delta x$};

% Title
\node[font=\large\bfseries] at (1,-0.6) {Riemann Tensor: Geometric Meaning};
\node[font=\small] at (1,-0.85) {Curvature = Holonomy Around Infinitesimal Loop};

\end{tikzpicture}
