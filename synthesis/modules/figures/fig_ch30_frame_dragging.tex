% fig_ch30_frame_dragging.tex
% Author: Computernonymouse
% Description: 3D visualization of frame-dragging (Lense-Thirring effect)
% Shows gravitomagnetic field around rotating mass with vector field
% Mathematical basis: B_g = (2G/c^2*r^3) J x r

\documentclass[tikz,border=10pt]{standalone}
\usepackage{pgfplots}
\usepackage{amsmath,amssymb}
\usepackage{xcolor}
\usepackage{tikz-3dplot}
\pgfplotsset{compat=1.18}
\usetikzlibrary{arrows.meta,decorations.pathreplacing,calc,patterns,backgrounds,shapes.geometric,positioning}

% Define custom colors
\definecolor{earth}{RGB}{100,150,200}
\definecolor{fieldline}{RGB}{200,100,50}
\definecolor{gyro}{RGB}{150,50,150}
\definecolor{satellite}{RGB}{50,150,50}

% Set 3D perspective
\tdplotsetmaincoords{70}{45}

\begin{document}
\begin{tikzpicture}[tdplot_main_coords,scale=1.2]

% Title
\node[font=\Large\bfseries] at (0,8,0) {Frame-Dragging (Lense-Thirring Effect)};
\node[font=\normalsize] at (0,7.3,0) {$\vec{B}_g = \frac{2G}{c^2 r^3}(\vec{J} \times \vec{r})$};

% Draw Earth as central rotating sphere
\shade[ball color=earth,opacity=0.8] (0,0,0) circle (1.5cm);

% Add rotation axis arrow
\draw[-{Latex[length=4mm,width=3mm]},ultra thick,blue!80!black]
    (0,0,-2.5) -- (0,0,2.5)
    node[above] {\bfseries Rotation axis $\vec{\omega}$};

% Add angular momentum vector
\draw[-{Latex[length=3mm,width=2mm]},thick,blue!60!black]
    (0,0,0) -- (0,0,1.8)
    node[right,midway] {$\vec{J}$};

% Draw equatorial plane (faint)
\begin{scope}[canvas is xy plane at z=0]
    \draw[gray!30,dashed] (0,0) circle (4cm);
    \node[gray,font=\small] at (4.3,0) {Equatorial plane};
\end{scope}

% Draw field lines (circular around rotation axis)
\foreach \z in {-1.5,-0.75,0,0.75,1.5} {
    \begin{scope}[canvas is xy plane at z=\z]
        \draw[fieldline,thick,decoration={markings,
            mark=at position 0.125 with {\arrow{Latex[length=2mm]}},
            mark=at position 0.375 with {\arrow{Latex[length=2mm]}},
            mark=at position 0.625 with {\arrow{Latex[length=2mm]}},
            mark=at position 0.875 with {\arrow{Latex[length=2mm]}}
        },postaction={decorate}] (0,0) circle ({2.5+0.3*abs(\z)});
    \end{scope}
}

% Draw velocity field vectors in a grid pattern
\foreach \theta in {0,45,90,135,180,225,270,315} {
    \foreach \r in {2.5,3.5} {
        % Calculate position
        \pgfmathsetmacro{\px}{\r*cos(\theta)}
        \pgfmathsetmacro{\py}{\r*sin(\theta)}
        % Calculate field direction (perpendicular to radial)
        \pgfmathsetmacro{\vx}{-0.4*sin(\theta)}
        \pgfmathsetmacro{\vy}{0.4*cos(\theta)}

        % Upper hemisphere vectors
        \draw[-{Latex[length=2mm]},fieldline!70,thick]
            (\px,\py,0.7) -- ({\px+\vx},{\py+\vy},0.7);

        % Lower hemisphere vectors
        \draw[-{Latex[length=2mm]},fieldline!70,thick]
            (\px,\py,-0.7) -- ({\px+\vx},{\py+\vy},-0.7);
    }
}

% Draw Gravity Probe B satellite orbit
\begin{scope}[rotate around z=30]
    \draw[satellite,thick,dashed] (0,0,0) circle (3.5cm);

    % Draw satellite
    \node[circle,fill=satellite,inner sep=3pt] (sat) at (3.5,0,0) {};
    \node[anchor=west,font=\small\bfseries,satellite!80!black] at (3.7,0,0) {Gravity Probe B};

    % Draw gyroscope precession
    \draw[-{Latex[length=2mm]},gyro,ultra thick] (3.5,0,0) -- (3.5,0,0.8)
        node[above,font=\small] {Gyroscope};

    % Precession cone
    \draw[gyro,dashed] (3.5,0,0) -- (3.3,0.2,0.8);
    \draw[gyro,dashed] (3.5,0,0) -- (3.7,-0.2,0.8);

    % Precession arc
    \draw[gyro,thick,->] (3.4,0.1,0.8) arc (150:210:0.2);
    \node[gyro,font=\tiny] at (3.5,-0.3,0.8) {Precession};
\end{scope}

% Add electromagnetic analogy diagram (bottom right)
\begin{scope}[shift={(6,-3,0)},scale=0.8]
    \node[font=\small\bfseries] at (0,2.5,0) {EM Analogy};

    % Current-carrying wire
    \draw[ultra thick,gray] (0,0,-1.5) -- (0,0,1.5);
    \draw[-{Latex[length=2mm]},thick,blue!80!black] (0,0,-1) -- (0,0,1)
        node[right,midway,font=\tiny] {Current $I$};

    % Magnetic field circles
    \foreach \z in {-0.5,0,0.5} {
        \begin{scope}[canvas is xy plane at z=\z]
            \draw[red!60,thick,decoration={markings,
                mark=at position 0.25 with {\arrow{Latex[length=1.5mm]}},
                mark=at position 0.75 with {\arrow{Latex[length=1.5mm]}}
            },postaction={decorate}] (0,0) circle (1cm);
        \end{scope}
    }

    \node[font=\tiny,align=center] at (0,-2,0) {
        Magnetic field $\vec{B}$\\
        around current
    };
\end{scope}

% Add parameter box
\node[anchor=north west,draw=black!40,fill=white,rounded corners,text width=5cm,align=left,font=\footnotesize]
    at (-7,5,0) {
    \textbf{Frame-Dragging Parameters:}\\[2mm]
    \textbf{Earth:}\\
    $\bullet$ $M = 5.97 \times 10^{24}$ kg\\
    $\bullet$ $R = 6.37 \times 10^6$ m\\
    $\bullet$ $J = 2MR^2\omega/5$\\
    $\bullet$ $\omega = 7.3 \times 10^{-5}$ rad/s\\[2mm]
    \textbf{Effect magnitude:}\\
    $\bullet$ $|\vec{B}_g| \sim 10^{-11}$ (Earth surface)\\
    $\bullet$ Precession: 6.6 arcsec/year\\
    $\bullet$ Measured by GP-B: 2004-2011\\[2mm]
    \textbf{Gravitomagnetic field:}\\
    $\vec{B}_g = \frac{2G}{c^2 r^3}(\vec{J} \times \vec{r})$\\[1mm]
    Analogous to $\vec{B} = \frac{\mu_0}{4\pi r^3}(\vec{\mu} \times \vec{r})$
};

% Add effect strength indicator
\begin{scope}[shift={(0,-5,0)}]
    \draw[thick,->] (-4,0,0) -- (4,0,0) node[right,font=\small] {Distance from center};
    \draw[thick,->] (-4,0,0) -- (-4,2,0) node[above,font=\small] {Effect strength};

    % Plot 1/r^3 falloff
    \draw[ultra thick,orange!80!black,domain=-3.5:3.5,samples=50,smooth]
        plot (\x,{0.8/(0.5+abs(\x))^1.5});

    \node[font=\small] at (0,-0.5,0) {$|\vec{B}_g| \propto r^{-3}$ falloff};
\end{scope}

% Add detection methods box
\node[anchor=south east,draw=black!40,fill=white,rounded corners,text width=4.5cm,align=left,font=\footnotesize]
    at (7,-1,0) {
    \textbf{Detection Methods:}\\[2mm]
    1. \textbf{Gyroscope precession}\\
    \phantom{1.} (Gravity Probe B)\\[2mm]
    2. \textbf{Satellite orbit shifts}\\
    \phantom{2.} (LAGEOS, LARES)\\[2mm]
    3. \textbf{Pulsar timing}\\
    \phantom{3.} (Binary systems)\\[2mm]
    4. \textbf{Ring laser gyros}\\
    \phantom{4.} (Future terrestrial)
};

% Bottom caption
\node[below=6cm of current bounding box.center,text width=14cm,align=center,font=\small] {
    \textbf{Figure:} Visualization of frame-dragging around a rotating mass (Earth). The gravitomagnetic
    field $\vec{B}_g$ forms circular patterns around the rotation axis, analogous to the magnetic field
    around a current-carrying wire. The Gravity Probe B satellite measured gyroscope precession of
    6.6 arcseconds per year, confirming general relativity predictions to within 0.3\%.
};

\end{tikzpicture}
\end{document}