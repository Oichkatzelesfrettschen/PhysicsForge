% fig_ch30_warp_drive_spacetime.tex
% Author: Computernonymouse
% Description: 3D visualization of Alcubierre warp drive spacetime metric
% Shows spacetime contraction ahead and expansion behind the warp bubble
% Mathematical basis: ds^2 = -c^2 dt^2 + [dx - v_s f(r) dt]^2 + dy^2 + dz^2

\documentclass[tikz,border=10pt]{standalone}
\usepackage{pgfplots}
\usepackage{amsmath,amssymb}
\usepackage{xcolor}
\pgfplotsset{compat=1.18}
\usepgfplotslibrary{colorbrewer}
\usetikzlibrary{arrows.meta,decorations.pathreplacing,calc,patterns,backgrounds}

% Define custom colormap for spacetime curvature
\pgfplotsset{
    colormap={warpspace}{
        rgb255(0cm)=(26,26,120);     % Deep blue (contracted space)
        rgb255(1cm)=(40,60,190);     % Blue
        rgb255(2cm)=(70,130,255);    % Light blue
        rgb255(3cm)=(255,255,255);   % White (flat space)
        rgb255(4cm)=(255,200,100);   % Light orange
        rgb255(5cm)=(255,100,50);    % Orange
        rgb255(6cm)=(200,30,30)      % Red (expanded space)
    }
}

\begin{document}
\begin{tikzpicture}

% Main 3D plot showing warp bubble geometry
\begin{axis}[
    view={30}{25},
    width=16cm,
    height=12cm,
    xlabel={$x$ [bubble radii]},
    ylabel={$y$ [bubble radii]},
    zlabel={$g_{tt}$ [metric component]},
    xmin=-4,xmax=4,
    ymin=-3,ymax=3,
    zmin=-1.5,zmax=1.5,
    xtick={-4,-3,-2,-1,0,1,2,3,4},
    ytick={-3,-2,-1,0,1,2,3},
    ztick={-1.5,-1,-0.5,0,0.5,1,1.5},
    grid=major,
    grid style={gray!20},
    colormap name=warpspace,
    colorbar,
    colorbar style={
        title={Spacetime\\curvature},
        ylabel={},
        ytick={-1.5,-1,-0.5,0,0.5,1,1.5},
        yticklabels={Contracted,,,Flat,,,Expanded},
        width=8pt,
        at={(1.05,0.5)},
        anchor=west
    },
    axis lines=center,
    axis line style={-Latex,thick},
    mesh/ordering=y varies,
    mesh/cols=60,
    samples=60,
    samples y=45,
    shader=interp,
    title={{\Large\bfseries Alcubierre Warp Drive Metric}\\[2mm]
           {\normalsize $ds^2 = -c^2dt^2 + [dx - v_s f(r_s)dt]^2 + dy^2 + dz^2$}},
    title style={at={(0.5,1.08)}}
]

% Define warp bubble shape function
% f(r) = tanh(sigma*(r + R)) - tanh(sigma*(r - R))
% where r = sqrt(x^2 + y^2), R = bubble radius, sigma = sharpness
\addplot3[
    surf,
    domain=-4:4,
    domain y=-3:3,
    point meta=z,
]
{
    % Alcubierre metric component g_tt
    % Creates contraction ahead (negative z) and expansion behind (positive z)
    0.8 * (tanh(2*(sqrt((x+1.5)^2 + y^2) - 1.2)) - tanh(2*(sqrt((x+1.5)^2 + y^2) + 1.2)))
    * exp(-0.15*(x^2 + y^2))
};

% Add contour lines at key metric values
\addplot3[
    contour gnuplot={
        levels={-1,-0.5,0,0.5,1},
        draw color=black!40,
        labels=false,
    },
    domain=-4:4,
    domain y=-3:3,
    samples=40,
]
{
    0.8 * (tanh(2*(sqrt((x+1.5)^2 + y^2) - 1.2)) - tanh(2*(sqrt((x+1.5)^2 + y^2) + 1.2)))
    * exp(-0.15*(x^2 + y^2))
};

% Mark bubble center
\addplot3[
    mark=*,
    mark size=3pt,
    color=black,
    only marks,
] coordinates {(-1.5,0,0)};

% Add annotations
\node[anchor=south] at (axis cs:-1.5,0,0.2) {\small\bfseries Bubble center};

% Direction of motion arrow
\draw[-{Latex[length=4mm,width=3mm]},ultra thick,red!80!black]
    (axis cs:2,0,0) -- (axis cs:-3,0,0)
    node[midway,below=3mm] {\small\bfseries Direction of motion};

\end{axis}

% Add side annotations with parameter explanations
\begin{scope}[on background layer]
    % Bubble radius indicator
    \draw[<->,thick,blue!60!black] ([xshift=-5mm]current axis.south west) --
        node[left,align=right] {Bubble\\radius\\$r_s$} ([xshift=-5mm]current axis.north west);

    % Sharpness parameter indicator
    \node[anchor=north west,align=left,draw=black!40,fill=white,rounded corners]
        at ([xshift=5mm,yshift=-5mm]current axis.north east) {
        \textbf{Parameters:}\\[1mm]
        $r_s = 1.2$ (bubble radius)\\
        $\sigma = 2$ (sharpness)\\
        $v_s = $ warp velocity\\[1mm]
        \textbf{Regions:}\\[1mm]
        {\color{blue!70!black}Blue}: Contracted space\\
        {\color{red!70!black}Red}: Expanded space\\
        White: Flat spacetime
    };
\end{scope}

% Add detailed caption-like description
\node[below=8mm of current axis.south,text width=14cm,align=justify,font=\small] {
    \textbf{Figure:} Spacetime geometry of an Alcubierre warp drive showing the metric
    component $g_{tt}$ that creates spacetime contraction ahead of the bubble (blue region)
    and expansion behind it (red region). The warp bubble moves in the negative $x$ direction
    while the interior remains in flat spacetime. The shape function creates a smooth
    transition controlled by the sharpness parameter $\sigma$.
};

% Add small inset showing cross-section
\begin{scope}[shift={(11.5,-2)}]
    \begin{axis}[
        width=5cm,
        height=3.5cm,
        xlabel={$x$ [bubble radii]},
        ylabel={$g_{tt}$},
        xmin=-4,xmax=2,
        ymin=-1.5,ymax=1.5,
        grid=major,
        grid style={gray!20},
        title={\small Cross-section at $y=0$},
        title style={at={(0.5,1.05)}},
        axis lines=left,
        xtick={-4,-3,-2,-1,0,1,2},
        ytick={-1.5,-1,-0.5,0,0.5,1,1.5},
        yticklabels={-1.5,-1,-0.5,0,0.5,1,1.5},
        tick label style={font=\tiny},
        label style={font=\small},
        legend pos=south east,
        legend style={font=\tiny,draw=none,fill=white,fill opacity=0.8}
    ]

    % Plot cross-section
    \addplot[
        thick,
        blue!60!red,
        domain=-4:2,
        samples=100,
    ]
    {
        0.8 * (tanh(2*(abs(x+1.5) - 1.2)) - tanh(2*(abs(x+1.5) + 1.2)))
        * exp(-0.15*x^2)
    };

    % Mark key points
    \addplot[mark=*,only marks,red] coordinates {(-2.7,{0.8 * (tanh(2*(abs(-2.7+1.5) - 1.2)) - tanh(2*(abs(-2.7+1.5) + 1.2))) * exp(-0.15*(-2.7)^2)})};
    \addplot[mark=*,only marks,blue] coordinates {(-0.3,{0.8 * (tanh(2*(abs(-0.3+1.5) - 1.2)) - tanh(2*(abs(-0.3+1.5) + 1.2))) * exp(-0.15*(-0.3)^2)})};

    \legend{Metric,$g_{tt} < 0$,$g_{tt} > 0$}
    \end{axis}
\end{scope}

\end{tikzpicture}
\end{document}