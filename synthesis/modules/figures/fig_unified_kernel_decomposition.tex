%==============================================================================
% Figure: Unified Kernel Decomposition
% Purpose: Genesis kernel as central hub with component inputs/outputs
% Chapter: Ch21 - Unified Mathematical Kernel
% Type: Mathematical/Schematic
%==============================================================================

\begin{figure}[htbp]
  \centering
  \begin{tikzpicture}[
    scale=1.0,
    kernel/.style={ellipse, draw=black, fill=purple!20, ultra thick, minimum width=5cm, minimum height=3cm, align=center, font=\Large\bfseries},
    component/.style={rectangle, draw=black, fill=blue!15, thick, minimum width=3.5cm, minimum height=1.2cm, align=center, font=\small},
    limit/.style={rectangle, draw=black, fill=green!15, thick, minimum width=3cm, minimum height=1.8cm, align=center, font=\small},
    arrow_in/.style={->, >=stealth, ultra thick, blue!70},
    arrow_out/.style={->, >=stealth, ultra thick, green!70}
  ]

    % Central kernel
    \node[kernel] (kernel) at (0, 0) {
      Unified Kernel \\[0.3cm]
      $K_{\text{total}}$
    };

    % Input components (arranged in a circle around kernel)
    \node[component] (base) at (0, 4.5) {
      \textbf{Base Kernel} \\
      $K_{\text{base}}(x,y,t)$ \\
      Spacetime geometry
    };

    \node[component] (scalar) at (4.5, 2.5) {
      \textbf{Scalar-ZPE} \\
      $K_{\text{scalar}}(x,t)$ \\
      Quantum vacuum
    };

    \node[component] (force) at (4.5, -2.5) {
      \textbf{Force Matrix} \\
      $F_M^{\text{extended}}$ \\
      Interactions
    };

    \node[component] (metric) at (0, -4.5) {
      \textbf{Metric Modulation} \\
      $M_n(x)$ \\
      Curvature
    };

    \node[component] (field) at (-4.5, -2.5) {
      \textbf{Field Potential} \\
      $\Phi_{\text{total}}(x,y,z,t)$ \\
      All fields
    };

    % Additional input (meta-principle)
    \node[component] (meta) at (-4.5, 2.5) {
      \textbf{Meta-Principle} \\
      $\mathcal{M}[\cdot]$ \\
      Selection rule
    };

    % Arrows from components to kernel
    \draw[arrow_in] (base) -- (kernel);
    \draw[arrow_in] (scalar) -- (kernel);
    \draw[arrow_in] (force) -- (kernel);
    \draw[arrow_in] (metric) -- (kernel);
    \draw[arrow_in] (field) -- (kernel);
    \draw[arrow_in] (meta) -- (kernel);

    % Output limiting cases (frameworks emerge)
    \node[limit] (aether_out) at (7, 0) {
      \textbf{Aether Limit} \\
      $K_{\text{scalar}} \to \max$ \\
      $M_n \to$ lattice \\
      \\
      Crystalline + \\
      Scalar-ZPE
    };

    \node[limit] (genesis_out) at (3.5, -6.5) {
      \textbf{Genesis Limit} \\
      $\mathcal{M} \to$ active \\
      $K_{\text{base}} \to$ graph \\
      \\
      Nodespace + \\
      Meta-principle
    };

    \node[limit] (pais_out) at (-3.5, -6.5) {
      \textbf{Pais Limit} \\
      $F_M \to$ GEM \\
      $\Phi \to$ EM+Grav \\
      \\
      Superforce \\
      coupling
    };

    % Arrows from kernel to limiting cases
    \draw[arrow_out] (kernel) -- (aether_out) node[midway, above, font=\footnotesize, sloped] {Specialize};
    \draw[arrow_out] (kernel) -- (genesis_out) node[midway, below, font=\footnotesize, sloped] {Specialize};
    \draw[arrow_out] (kernel) -- (pais_out) node[midway, below, font=\footnotesize, sloped] {Specialize};

    % Central kernel equation
    \node[anchor=north, align=center, font=\footnotesize, draw=purple!70, fill=purple!5, rounded corners, thick]
      at (0, -1.8) {
      $K_{\text{total}} = K_{\text{base}} \cdot K_{\text{scalar}} \cdot F_M \cdot M_n \cdot \Phi \cdot \mathcal{M}[\cdot]$
    };

    % Component details box
    \node[anchor=north west, align=left, font=\footnotesize, draw=black, fill=yellow!10, rounded corners, thick]
      at (-10, 5) {
      \textbf{Component Descriptions:} \\
      \\
      \textbf{Base Kernel} $K_{\text{base}}(x,y,t)$: \\
      $\quad$ Fundamental spacetime structure \\
      $\quad$ Can be continuum (metric) or discrete (lattice/graph) \\
      \\
      \textbf{Scalar-ZPE} $K_{\text{scalar}}(x,t)$: \\
      $\quad$ Zero-point energy density and scalar field \\
      $\quad$ Coupling: $g \phi^2 \rho_{\text{ZPE}}$ \\
      \\
      \textbf{Force Matrix} $F_M^{\text{extended}}$: \\
      $\quad$ Interaction matrix for all forces \\
      $\quad$ Includes EM, weak, strong, gravity, exotic \\
      \\
      \textbf{Metric Modulation} $M_n(x)$: \\
      $\quad$ Curvature and geometric perturbations \\
      $\quad$ $g_{\mu\nu}(x) = \eta_{\mu\nu} + h_{\mu\nu}(x)$ \\
      \\
      \textbf{Field Potential} $\Phi_{\text{total}}$: \\
      $\quad$ Sum of all field contributions \\
      $\quad$ Scalar, vector, tensor, spinor \\
      \\
      \textbf{Meta-Principle} $\mathcal{M}[\cdot]$: \\
      $\quad$ Variational principle or selection rule \\
      $\quad$ Determines which configurations are physical
    };

    % Limiting case details
    \node[anchor=north east, align=left, font=\footnotesize, draw=black, fill=green!5, rounded corners, thick]
      at (10, 5) {
      \textbf{Limiting Cases (Frameworks):} \\
      \\
      \textbf{Aether Limit:} \\
      $\quad$ Maximize $K_{\text{scalar}}$: ZPE coupling dominant \\
      $\quad$ $M_n \to$ discrete lattice at Planck scale \\
      $\quad$ $\mathcal{M} \to$ elasticity action \\
      $\quad$ Reproduce: Casimir modifications, time crystals \\
      \\
      \textbf{Genesis Limit:} \\
      $\quad$ Activate $\mathcal{M}$: meta-principle selects topology \\
      $\quad$ $K_{\text{base}} \to$ nodespace graph \\
      $\quad$ $M_n \to$ connectivity-based metric \\
      $\quad$ Reproduce: origami dimensions, multiverse \\
      \\
      \textbf{Pais Limit:} \\
      $\quad$ $F_M \to$ GEM coupling matrix \\
      $\quad$ $\Phi \to$ combined EM+Gravity potential \\
      $\quad$ $\kappa \sim 10^{-50}$ coupling constant \\
      $\quad$ Reproduce: Lense-Thirring, GEM waves \\
      \\
      \textbf{Full Kernel:} \\
      $\quad$ All components active simultaneously \\
      $\quad$ No framework preferred $a$ $priori$ \\
      $\quad$ Experimental data selects best description
    };

    % Title
    \node[anchor=south, font=\huge\bfseries] at (0, 6.5) {Unified Mathematical Kernel Decomposition};

    % Subtitle
    \node[anchor=north, font=\large\itshape] at (0, -8.5) {
      All frameworks as limiting cases of unified structure
    };

  \end{tikzpicture}
  \caption{Decomposition of the unified mathematical kernel $K_{\text{total}}$ as a product of six
    fundamental components (blue boxes, input arrows). The central purple ellipse represents the
    full kernel, which can be specialized to recover each framework as a limiting case (green boxes,
    output arrows). The six components are: (1) Base Kernel $K_{\text{base}}(x,y,t)$ encoding
    fundamental spacetime structure (continuum metric or discrete lattice/graph), (2) Scalar-ZPE
    $K_{\text{scalar}}(x,t)$ for zero-point energy and scalar field coupling, (3) Force Matrix
    $F_M^{\text{extended}}$ for all interactions (EM, weak, strong, gravity, exotic), (4) Metric
    Modulation $M_n(x)$ for curvature perturbations $g_{\mu\nu} = \eta_{\mu\nu} + h_{\mu\nu}$,
    (5) Field Potential $\Phi_{\text{total}}$ summing all field contributions, and (6) Meta-Principle
    $\mathcal{M}[\cdot]$ providing variational selection of physical configurations. Specialization
    yields frameworks: \textbf{Aether} by maximizing $K_{\text{scalar}}$ (ZPE dominant) and setting
    $M_n$ to discrete lattice; \textbf{Genesis} by activating $\mathcal{M}$ (meta-principle selection)
    and $K_{\text{base}} \to$ nodespace graph; \textbf{Pais} by setting $F_M \to$ GEM coupling and
    $\Phi \to$ EM+Gravity with $\kappa \sim 10^{-50}$. The full kernel with all components active
    represents the most general unified theory, with experimental data selecting the appropriate
    limiting case for each physical regime. This structure, developed in Ch20-21, provides the
    mathematical foundation for reconciling all three frameworks within a single formalism.}
  \label{fig:unified-kernel-decomposition}
\end{figure}
