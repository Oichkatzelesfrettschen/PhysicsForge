% Figure 7: Comparative Efficiency Chart
% Author: Dr. Computernonymouse
% Description: Logarithmic bar chart comparing power densities across
%              different energy technologies including ZPE harvesting

\documentclass[tikz,border=10pt]{standalone}
\usepackage{pgfplots}
\usepackage{pgfplotstable}
\usepackage{amsmath}
\usepackage{amssymb}
\usepackage{siunitx}

\pgfplotsset{compat=1.18}
\usetikzlibrary{patterns,arrows.meta}

% Define colors by category
\definecolor{chemical}{RGB}{255,127,14}
\definecolor{nuclear}{RGB}{214,39,40}
\definecolor{renewable}{RGB}{44,160,44}
\definecolor{experimental}{RGB}{148,103,189}
\definecolor{threshold}{RGB}{0,0,0}
\definecolor{gridcolor}{RGB}{128,128,128}

% Create data table
\pgfplotstableread{
Technology      PowerDensity    Category
Li-ion          1e6            1
Gasoline        1.3e10         1
Nuclear         3.8e11         2
Solar           1e2            3
Wind            3e1            3
ZPE             4e-5           4
}\datatable

\begin{document}

\begin{tikzpicture}

\begin{semilogyaxis}[
    ybar,
    bar width=1.2cm,
    width=14cm,
    height=10cm,
    ylabel={Power Density (W/m$^3$)},
    xlabel={Energy Technology},
    ymin=1e-6,
    ymax=1e13,
    ytick={1e-6,1e-4,1e-2,1e0,1e2,1e4,1e6,1e8,1e10,1e12},
    yticklabels={
        $10^{-6}$,
        $10^{-4}$,
        $10^{-2}$,
        $10^{0}$,
        $10^{2}$,
        $10^{4}$,
        $10^{6}$,
        $10^{8}$,
        $10^{10}$,
        $10^{12}$
    },
    xtick=data,
    xticklabels={Li-ion Battery,Gasoline,Nuclear Fission,Solar PV,Wind Turbine,ZPE Harvesting},
    x tick label style={rotate=45,anchor=east,font=\normalsize},
    ylabel style={font=\large},
    xlabel style={font=\large,yshift=-20pt},
    grid=y,
    major grid style={line width=0.5pt,draw=gridcolor!50},
    minor grid style={line width=0.2pt,draw=gridcolor!20},
    minor y tick num=9,
    legend pos=north west,
    legend style={
        font=\normalsize,
        fill=white,
        fill opacity=0.95,
        draw=black,
        rounded corners=2mm,
        row sep=2pt
    },
    title={Power Density Comparison Across Energy Technologies},
    title style={font=\Large\bfseries,yshift=5pt},
    enlarge x limits=0.15,
    nodes near coords align={vertical},
    point meta=explicit symbolic
]

% Li-ion Battery (Chemical)
\addplot[
    fill=chemical,
    draw=black,
    line width=0.5pt
] coordinates {(0,1e6)};

% Gasoline (Chemical)
\addplot[
    fill=chemical,
    draw=black,
    line width=0.5pt
] coordinates {(1,1.3e10)};

% Nuclear Fission (Nuclear)
\addplot[
    fill=nuclear,
    draw=black,
    line width=0.5pt
] coordinates {(2,3.8e11)};

% Solar PV (Renewable)
\addplot[
    fill=renewable,
    draw=black,
    line width=0.5pt
] coordinates {(3,1e2)};

% Wind Turbine (Renewable)
\addplot[
    fill=renewable,
    draw=black,
    line width=0.5pt
] coordinates {(4,3e1)};

% ZPE Harvesting (Experimental) - Highlighted
\addplot[
    fill=experimental,
    draw=red,
    line width=2pt,
    pattern=crosshatch,
    pattern color=experimental!50
] coordinates {(5,4e-5)};

% Viability threshold line at 1 W/m³
\draw[threshold,ultra thick,dashed] (axis cs:-0.5,1) -- (axis cs:5.5,1)
    node[right,font=\normalsize] {Viability Threshold};

% Category labels in legend
\addlegendimage{fill=chemical,draw=black}
\addlegendentry{Chemical}
\addlegendimage{fill=nuclear,draw=black}
\addlegendentry{Nuclear}
\addlegendimage{fill=renewable,draw=black}
\addlegendentry{Renewable}
\addlegendimage{fill=experimental,pattern=crosshatch,pattern color=experimental!50,draw=red,line width=2pt}
\addlegendentry{Experimental}

% Annotations with power density values
\node[above,font=\footnotesize] at (axis cs:0,2e6) {$10^6$};
\node[above,font=\footnotesize] at (axis cs:1,2.6e10) {$1.3 \times 10^{10}$};
\node[above,font=\footnotesize] at (axis cs:2,7.6e11) {$3.8 \times 10^{11}$};
\node[above,font=\footnotesize] at (axis cs:3,2e2) {$10^2$};
\node[above,font=\footnotesize] at (axis cs:4,6e1) {$3 \times 10^1$};
\node[above,font=\footnotesize,red,draw=red,rounded corners,fill=white] at (axis cs:5,8e-5) {$4 \times 10^{-5}$};

% Comparison annotations
\draw[<->,thick,red] (axis cs:5,1e-4) -- (axis cs:5,1)
    node[midway,right,font=\small,red] {$10^5 \times$ below threshold};

\draw[<->,thick,blue] (axis cs:3.5,1e2) -- (axis cs:5.5,4e-5)
    node[midway,below right,font=\small,blue,align=center] {
        $10^7 \times$ less than\\photovoltaics
    };

% Efficiency indicators
\node[align=center,font=\footnotesize] at (axis cs:0,5e4) {
    Mature\\Tech
};

\node[align=center,font=\footnotesize] at (axis cs:2,2e10) {
    Highest\\Density
};

\node[align=center,font=\footnotesize,red] at (axis cs:5,1e-7) {
    Theoretical\\Limit
};

\end{semilogyaxis}

% Additional information boxes
\node[draw=black,thick,rounded corners=2mm,fill=white!95,
      align=left,font=\footnotesize] at (3,2) {
    \textbf{ZPE Characteristics:}\\
    $\bullet$ Volume: 1 m$^3$ $\rightarrow$ 40 $\mu$W\\
    $\bullet$ Football field cavity $\rightarrow$ 0.4 W\\
    $\bullet$ Requires cryogenic cooling\\
    $\bullet$ Limited to niche applications
};

\node[draw=black,thick,rounded corners=2mm,fill=green!10,
      align=left,font=\footnotesize] at (10,8) {
    \textbf{Viable Technologies:}\\
    Power density $>$ 1 W/m$^3$\\
    Room temperature operation\\
    Established infrastructure
};

% Bottom summary
\node[below,font=\normalsize,align=center] at (7,-1.5) {
    \textbf{Conclusion:} ZPE harvesting faces fundamental physical limitations,\\
    with power density $10^{16}$ times lower than nuclear and $10^7$ times lower than solar PV.
};

\end{tikzpicture}

\end{document}