%==============================================================================
% Figure: Conflict Resolution Decision Tree
% Purpose: Decision tree for resolving framework conflicts
% Chapter: Ch18 - Conflict Resolution
% Type: Conceptual
%==============================================================================

\begin{figure}[htbp]
  \centering
  \begin{tikzpicture}[
    scale=1.0,
    level 1/.style={sibling distance=8cm, level distance=2.5cm},
    level 2/.style={sibling distance=4cm, level distance=2.5cm},
    level 3/.style={sibling distance=3cm, level distance=2.5cm},
    decision/.style={rectangle, draw=black, fill=yellow!20, thick, minimum width=3cm, minimum height=1cm, align=center, font=\small\bfseries},
    outcome/.style={rectangle, draw=black, fill=green!20, thick, minimum width=2.8cm, minimum height=1.2cm, align=center, font=\footnotesize},
    conflict/.style={ellipse, draw=red, fill=red!10, thick, minimum width=2.5cm, minimum height=0.8cm, align=center, font=\small},
    arrow/.style={->, >=stealth, thick}
  ]

    % Root: Conflict identified
    \node[conflict] (root) at (0, 0) {Conflict\\Identified};

    % Level 1: Scale question
    \node[decision] (scale) at (0, -2.5) {Do frameworks\\apply at\\different scales?};

    \draw[arrow] (root) -- (scale);

    % Level 2: Scale branches
    \node[outcome] (scale_yes) at (-6, -5.5) {
      \textbf{Scale Separation} \\
      Different domains \\
      \\
      \textit{Example:} \\
      Aether (Planck) \\
      vs Genesis (cosmic) \\
      \\
      \textcolor{blue}{$\Rightarrow$ Complementary}
    };

    \node[decision] (scale_no) at (6, -5.5) {Are predictions\\mathematically\\equivalent?};

    \draw[arrow] (scale) -- (scale_yes) node[midway, above, sloped, font=\tiny] {YES};
    \draw[arrow] (scale) -- (scale_no) node[midway, above, sloped, font=\tiny] {NO};

    % Level 3: Equivalence branches
    \node[outcome] (equiv_yes) at (3, -8.5) {
      \textbf{Equivalence} \\
      Different formulations \\
      same physics \\
      \\
      \textit{Example:} \\
      Discrete lattice \\
      $\leftrightarrow$ continuum limit \\
      \\
      \textcolor{blue}{$\Rightarrow$ Dual descriptions}
    };

    \node[decision] (equiv_no) at (9, -8.5) {Can experiment\\distinguish?};

    \draw[arrow] (scale_no) -- (equiv_yes) node[midway, above, sloped, font=\tiny] {YES};
    \draw[arrow] (scale_no) -- (equiv_no) node[midway, above, sloped, font=\tiny] {NO};

    % Level 4: Experimental test branches
    \node[outcome] (exp_yes) at (7, -11.5) {
      \textbf{Experimental Test} \\
      Design protocol \\
      measure difference \\
      \\
      \textit{Example:} \\
      Casimir geometry \\
      (Aether vs Pais) \\
      \\
      \textcolor{orange}{$\Rightarrow$ Ch22 protocols}
    };

    \node[outcome] (exp_no) at (11, -11.5) {
      \textbf{Meta-Analysis} \\
      Philosophical choice \\
      or pending data \\
      \\
      \textit{Example:} \\
      Ontology: \\
      medium vs emergent \\
      \\
      \textcolor{purple}{$\Rightarrow$ Document both}
    };

    \draw[arrow] (equiv_no) -- (exp_yes) node[midway, above, sloped, font=\tiny] {YES};
    \draw[arrow] (equiv_no) -- (exp_no) node[midway, above, sloped, font=\tiny] {NO};

    % Additional outcome: Reconciliation pathway
    \node[outcome] (reconcile) at (-2, -8.5) {
      \textbf{Reconciliation} \\
      Map between \\
      frameworks \\
      \\
      \textit{Example:} \\
      Aether 2048D \\
      $\leftrightarrow$ Genesis 8D \\
      via dimensional map \\
      \\
      \textcolor{blue}{$\Rightarrow$ Ch20 unified kernel}
    };

    \draw[arrow, dashed] (scale_yes) -- (reconcile) node[midway, above, sloped, font=\tiny] {IF RELATED};

    % Legend
    \node[anchor=north west, align=left, font=\small, draw=black, fill=yellow!10, rounded corners, thick]
      at (-13, 1) {
      \textbf{Decision Tree Legend:} \\
      \\
      \textcolor{red}{\textbullet} Conflict (red ellipse): \\
      \quad Incompatible claims identified \\
      \\
      \textcolor{orange!80}{\textbullet} Decision (yellow box): \\
      \quad Question to ask \\
      \\
      \textcolor{green!60!black}{\textbullet} Outcome (green box): \\
      \quad Resolution strategy \\
      \\
      \textbf{Colors in outcomes:} \\
      \textcolor{blue}{Blue}: Frameworks compatible \\
      \textcolor{orange}{Orange}: Experimental test needed \\
      \textcolor{purple}{Purple}: Undecidable (currently)
    };

    % Examples box
    \node[anchor=north east, align=left, font=\small, draw=black, fill=blue!5, rounded corners, thick]
      at (13, 1) {
      \textbf{Specific Conflicts Resolved:} \\
      \\
      \textbf{1. Dimensionality (2048D vs 8D):} \\
      $\Rightarrow$ Reconciliation via Ch20 map \\
      \\
      \textbf{2. ZPE cutoff (Planck vs dynamic):} \\
      $\Rightarrow$ Scale separation (Aether micro, Genesis macro) \\
      \\
      \textbf{3. Unification (scalar vs GEM):} \\
      $\Rightarrow$ Experimental test (Casimir vs GW polarization) \\
      \\
      \textbf{4. Discrete vs continuous:} \\
      $\Rightarrow$ Equivalence (lattice $\to$ continuum limit) \\
      \\
      \textbf{5. Meta-principle vs mechanism:} \\
      $\Rightarrow$ Meta-analysis (ontological difference)
    };

    % Title
    \node[anchor=south, font=\huge\bfseries] at (0, 1.5) {Conflict Resolution Decision Tree};

  \end{tikzpicture}
  \caption{Decision tree for systematically resolving conflicts between the Aether, Genesis, and
    Pais frameworks. Starting from an identified conflict (red ellipse), the tree branches through
    a series of diagnostic questions (yellow boxes) to reach resolution strategies (green boxes).
    The first question asks whether frameworks apply at different scales: if yes, conflicts may
    resolve via scale separation (e.g., Aether at Planck scale, Genesis at cosmological scale) or
    reconciliation through mapping (Ch20 dimensional correspondence). If scales overlap, the next
    question asks whether predictions are mathematically equivalent despite different formalisms:
    if yes, frameworks are dual descriptions of the same physics (e.g., discrete lattice versus
    continuum limit). If not equivalent, the final question asks whether experiments can distinguish:
    if yes, design experimental protocols (orange, Ch22); if no, document both as meta-analysis
    pending future data (purple, philosophical differences like ontology). Color coding in outcomes
    indicates: blue (frameworks compatible/complementary), orange (experimental resolution needed),
    purple (currently undecidable). Specific examples include: dimensionality conflict (2048D vs
    8D) resolved via Ch20 reconciliation, ZPE cutoff via scale separation, unification mechanisms
    via Casimir/GW experiments, and discrete vs continuous via equivalence. This systematic approach
    ensures all conflicts are addressed rigorously in Ch18.}
  \label{fig:conflict-resolution-tree}
\end{figure>
