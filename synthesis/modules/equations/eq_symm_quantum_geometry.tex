%==============================================================================
% Equation Module: Genesis Quantum Geometry
% Source: Chapter 14 - Genesis Superforce Applications
% Date: 2025-10-23
%==============================================================================

\begin{equation}
  \boxed{
  \hat{A}_S = 8\pi\gamma l_P^2 \sum_{e \cap S} \sqrt{\hat{j}_e(\hat{j}_e+1)}
  }
  \eqtag{X}{QG}{AS}
  \label{eq:module:genesis-area-spectrum}
\end{equation}

\noindent where:
\begin{itemize}[noitemsep]
  \item $\hat{A}_S$: Area operator for surface $S$
  \item $\gamma$: Immirzi parameter
  \item $l_P = \sqrt{\hbar G / c^3}$: Planck length
  \item $\hat{j}_e$: Spin operator on edge $e$
  \item $e \cap S$: Edges intersecting surface $S$
\end{itemize}

\vspace{0.5em}
\noindent\textbf{Volume Operator:}
\begin{equation}
  \hat{V}_R = \sqrt{\frac{2}{3}} l_P^3 \sum_{v \in R} \sqrt{|\hat{q}_v|}
  \eqtag{X}{QG}{VS}
\end{equation}

\vspace{0.5em}
\noindent\textbf{Hamiltonian Constraint:}
\begin{equation}
  \hat{H}_{\text{Genesis}} = \frac{1}{16\pi G} \sum_{v \in V} \epsilon_{ijk} \text{Tr}\left(F_{ij}^v \left[\hat{V}_v, F_{jk}^v\right]\right)
  \eqtag{X}{QG}{HC}
\end{equation}

\vspace{0.5em}
\noindent\textbf{Black Hole Entropy:}
\begin{equation}
  S_{\text{BH}} = \frac{A}{4l_P^2} - \frac{1}{2}\ln\left(\frac{A}{l_P^2}\right) + \sum_{k=1}^{\infty} \frac{c_k}{\varphi^k} \left(\frac{l_P^2}{A}\right)^k
  \eqtag{X}{QG}{BH}
\end{equation}

\vspace{0.5em}
\noindent\textbf{Eigenvalue Spectra:}
\begin{equation}
  A_n = 8\pi\gamma l_P^2 \sqrt{n(n+1)}, \quad V_m = l_P^3 \varphi^m
  \eqtag{X}{QG}{EV}
\end{equation}

\noindent\textbf{Physical Significance:} Quantum geometry emerges from nodespace structure. Area and volume have discrete spectra, resolving singularities. The correspondence with loop quantum gravity validates the nodespace approach. Black hole entropy matches Bekenstein-Hawking formula with quantum corrections scaling as golden ratio powers.

%==============================================================================
% End of Equation Module
%==============================================================================