%==============================================================================
% Equation Module: Genesis Folding Topology
% Source: Chapter 13 - Genesis Origami Dimensions
% Date: 2025-10-23
%==============================================================================

\begin{equation}
  \boxed{
  H^k_{\text{fold}}(E) = \frac{\text{Ker}(d^k_{\text{fold}})}{\text{Im}(d^{k-1}_{\text{fold}})}, \quad d_{\text{fold}} = d + \mathcal{M}_{\text{fold}}
  }
  \eqtag{G}{TP}{CH}
  \label{eq:module:genesis-folding-cohomology}
\end{equation}

\noindent where:
\begin{itemize}[noitemsep]
  \item $H^k_{\text{fold}}$: $k$-th folding cohomology group
  \item $d$: Standard exterior derivative
  \item $\mathcal{M}_{\text{fold}}$: FoldMerge contribution to differential
  \item $E$: Total space of origami bundle
\end{itemize}

\vspace{0.5em}
\noindent\textbf{Origami Bundle Structure:}
\begin{equation}
  E = B \times_{\mathcal{M}_{\text{fold}}} F
  \eqtag{G}{TP}{BS}
\end{equation}
where $B$ is base spacetime, $F$ is folded fiber, and $\times_{\mathcal{M}_{\text{fold}}}$ denotes twisted product.

\vspace{0.5em}
\noindent\textbf{Folding Index Theorem:}
\begin{equation}
  \text{Index}(\mathcal{D}_{\text{fold}}) = \int_M \hat{A}(M) \wedge \text{ch}(\mathcal{F}_{\text{fold}})
  \eqtag{G}{TP}{IT}
\end{equation}

\vspace{0.5em}
\noindent\textbf{Chern Classes of Folded Bundles:}
\begin{equation}
  c_k(\mathcal{E}_{\text{fold}}) = \frac{1}{(2\pi i)^k} \text{Tr}(F_{\text{fold}}^k)
  \eqtag{G}{TP}{CC}
\end{equation}

\vspace{0.5em}
\noindent\textbf{Euler Characteristic:}
\begin{equation}
  \chi_{\text{fold}}(E) = \sum_{k=0}^{\dim E} (-1)^k \dim H^k_{\text{fold}}(E)
  \eqtag{G}{TP}{EC}
\end{equation}

\noindent\textbf{Physical Significance:} The topological structure of folded dimensions determines their physical properties. The folding cohomology encodes conservation laws, while characteristic classes determine anomalies and topological phases. The index theorem relates spectral properties to topology, crucial for understanding fermion zero modes in folded backgrounds.

%==============================================================================
% End of Equation Module
%==============================================================================