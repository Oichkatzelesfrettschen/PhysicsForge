%==============================================================================
% Equation Module: GEM Gravitoelectric Field Definition
% Source: PAIS_UNIFIED.md Section 2.1 (GEM theory foundation)
% Tag: eq:pais:gravitoelectric-field
%==============================================================================

\begin{equation}
  \mathbf{E}_g = -\nabla \Phi_g - \frac{\partial \mathbf{A}_g}{\partial t}
  \eqtag{P}{GEM}{EG}
  \label{eq:pais:gravitoelectric-field}
\end{equation}

\noindent\textbf{Physical Interpretation:} The gravitoelectric field $\mathbf{E}_g$ is the GEM analog of the electric field in electromagnetism. In the static limit ($\partial\mathbf{A}_g/\partial t = 0$), this reduces to the standard Newtonian gravitational acceleration $\mathbf{g} = -\nabla\Phi_g$, where $\Phi_g = -GM/r$ is the gravitational potential.

\noindent\textbf{Dynamic Effects:} The time-varying vector potential term $\partial\mathbf{A}_g/\partial t$ represents gravitational induction effects analogous to Faraday's law in electromagnetism. A time-varying gravitomagnetic field $\mathbf{B}_g = \nabla \times \mathbf{A}_g$ induces a gravitoelectric field, enabling gravitational wave generation and propagation.

\noindent\textbf{Connection to Metric:} The gravitoelectric potential relates to metric perturbations via $\Phi_g = -c^2 h_{00}/2$ in the weak-field limit, where $h_{\mu\nu}$ represents deviations from Minkowski spacetime: $g_{\mu\nu} = \eta_{\mu\nu} + h_{\mu\nu}$.

\noindent\textbf{Experimental Measurement:} Gravitoelectric fields are measured by test particle accelerations: $\mathbf{a} = \mathbf{E}_g$. Precision gravimeters achieve sensitivity $\sim 10^{-10}$ m/s$^2$, sufficient to detect tidal gravitoelectric field gradients from planetary masses.
