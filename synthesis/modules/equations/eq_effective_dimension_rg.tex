%==============================================================================
% Equation: Effective Dimension Renormalization Group Flow
% Source: Theoretical synthesis (dimensional running with energy scale)
%         Inspired by AdS/CFT holographic dimension (Maldacena)
%         Fractal RG from Wilson renormalization
% Framework: Unified | Domain: QM+GR | Status: Theoretical
%==============================================================================
% Describes how effective spacetime dimensionality varies with energy scale
% (or equivalently, probing length scale) via renormalization group equations.
% Unifies Aether's integer dimensional hierarchy, Genesis's fractal dimensions,
% and standard 4D spacetime into a single scale-dependent framework.
%==============================================================================

\begin{equation}
  \frac{\dd D_{\text{eff}}}{\dd \log \mu} = \beta_D(g, D_{\text{eff}}, \lambda)
  \eqtag{P}{QM}{T}
  \label{eq:unified:dimension-rg-flow}
\end{equation}

\noindent
where:
\begin{itemize}
  \item $D_{\text{eff}}(\mu)$: Effective spacetime dimension at energy scale $\mu$
  \item $\mu$: Renormalization scale (energy or inverse length)
  \item $\beta_D$: Dimensional beta function (anomalous dimension)
  \item $g = \{g_i\}$: Set of coupling constants (gravitational, gauge, scalar-ZPE)
  \item $\lambda$: Fractal/origami parameter (folding angle, Hausdorff exponent)
\end{itemize}

%==============================================================================
% BETA FUNCTION SPECIFICATION
%==============================================================================

\paragraph{Dimensional Beta Function:} Explicit form derived from fractal geometry and hypercomplex algebra:

\begin{equation}
  \beta_D(g, D, \lambda) = \alpha_0 \frac{g_{\text{grav}}^2}{16\pi^2}
    \left( D - D_{\text{base}} \right)
    + \alpha_1 \frac{g_{\text{scalar}}^2}{8\pi^2} \log\left(1 + \frac{\mu}{\mu_{\text{Planck}}}\right)
    + \alpha_2 \lambda \sin^2\left(\frac{\pi D}{D_{\text{max}}}\right)
  \eqtag{P}{QM}{T}
  \label{eq:unified:beta-function-dimension}
\end{equation}

\noindent
where:
\begin{itemize}
  \item $\alpha_0, \alpha_1, \alpha_2$: Dimensionless coefficients (framework-dependent)
  \item $g_{\text{grav}}$: Gravitational coupling $\sim \sqrt{G\mu^2/\hbar c^3}$
  \item $g_{\text{scalar}}$: Scalar-ZPE coupling strength
  \item $D_{\text{base}} = 4$: Macroscopic base dimensionality
  \item $D_{\text{max}} = 2048$: Maximum Cayley-Dickson dimension
  \item $\mu_{\text{Planck}} = \sqrt{\hbar c^5 / G} \approx 1.22 \times 10^{19}$ GeV
\end{itemize}

%==============================================================================
% SOLUTION AND FIXED POINTS
%==============================================================================

\paragraph{Fixed Points:} Dimensional RG flow has fixed points where $\beta_D = 0$:

\begin{equation}
  D_{\text{eff}}^* \quad \text{such that} \quad \beta_D(g, D^*, \lambda) = 0
  \label{eq:unified:dimension-fixed-points}
\end{equation}

\noindent
Typical fixed point structure:
\begin{itemize}
  \item \textbf{IR fixed point} ($\mu \ll \mu_{\text{Planck}}$): $D^*_{\text{IR}} = 4$ (classical spacetime)
  \item \textbf{Intermediate fixed point} ($\mu \sim 1$ TeV): $D^*_{\text{int}} \approx 4 + \epsilon$ (fractal corrections, $\epsilon \sim 0.1-0.5$)
  \item \textbf{UV fixed point} ($\mu \to \mu_{\text{Planck}}$): $D^*_{\text{UV}} \approx 8$ (octonion structure)
  \item \textbf{Trans-Planckian limit} ($\mu \gg \mu_{\text{Planck}}$): $D^*_{\text{TP}} \to D_{\text{max}}$ (full Cayley-Dickson hierarchy)
\end{itemize}

\paragraph{Stability Analysis:} Stability of fixed points determined by:
\begin{equation}
  \omega_D = \frac{\partial \beta_D}{\partial D}\bigg|_{D = D^*}
  \label{eq:unified:stability-exponent}
\end{equation}

\begin{itemize}
  \item $\omega_D < 0$: Stable (IR attractive)
  \item $\omega_D > 0$: Unstable (UV repulsive)
  \item $\omega_D = 0$: Marginal (logarithmic corrections)
\end{itemize}

%==============================================================================
% SCALE-DEPENDENT DIMENSIONAL HIERARCHY
%==============================================================================

\paragraph{Explicit Solution:} For weak coupling and small $\lambda$, perturbative solution:

\begin{equation}
  D_{\text{eff}}(\mu) = D_{\text{base}}
    + \sum_{n=1}^{N} \Delta D_n \cdot \Theta\left(\mu - \mu_{\text{threshold},n}\right)
    \cdot \left(1 - \exp\left(-\frac{\mu - \mu_{\text{threshold},n}}{\mu_n}\right)\right)
  \eqtag{P}{QM}{T}
  \label{eq:unified:dimensional-hierarchy-solution}
\end{equation}

\noindent
where:
\begin{itemize}
  \item $\Delta D_n$: Dimensional jump at $n$-th threshold (related to Cayley-Dickson doubling)
  \item $\mu_{\text{threshold},n}$: Energy threshold for $n$-th dimensional activation
  \item $\mu_n$: Characteristic smoothing scale
  \item $\Theta(x)$: Heaviside step function
\end{itemize}

\paragraph{Dimensional Thresholds:} Correspondence to Cayley-Dickson levels:

\begin{align}
  \mu_{\text{threshold},1} &\sim 1 \text{ GeV} \quad &&(\text{QCD scale, fractal onset}) \nonumber \\
  \mu_{\text{threshold},2} &\sim 100 \text{ GeV} \quad &&(\text{Electroweak scale, } \mathbb{H} \text{ structure}) \nonumber \\
  \mu_{\text{threshold},3} &\sim 10 \text{ TeV} \quad &&(\text{Octonion activation, } \mathbb{O}) \nonumber \\
  \mu_{\text{threshold},4} &\sim 10^{3} \text{ TeV} \quad &&(\text{Sedenion level, } \mathbb{S}) \nonumber \\
  \mu_{\text{threshold},n} &\sim \mu_{\text{Planck}} \cdot 2^{-(11-n)} \quad &&(\text{Higher Cayley-Dickson levels})
  \label{eq:unified:threshold-hierarchy}
\end{align}

%==============================================================================
% WORKED EXAMPLE: IR TO PLANCK SCALE
%==============================================================================

\paragraph{Worked Example:} Dimensional flow from IR to Planck scale with parameters:
\begin{align*}
  D_{\text{base}} &= 4 \\
  \alpha_0 &= 0.1, \quad \alpha_1 = 0.05, \quad \alpha_2 = 0.02 \\
  g_{\text{grav}}(\mu) &= \sqrt{G\mu^2/(\hbar c^3)} \\
  g_{\text{scalar}} &= 0.3 \quad \text{(dimensionless)} \\
  \lambda &= 0.5
\end{align*}

At low energy ($\mu = 1$ GeV $\ll \mu_{\text{Planck}}$):
\begin{align*}
  g_{\text{grav}} &\approx 10^{-19} \\
  \beta_D &\approx 0.05 \cdot \frac{0.09}{8\pi^2} \cdot \log(10^{-19}) + 0.02 \cdot 0.5 \cdot 1 \\
        &\approx -0.0002 + 0.01 \approx 0.01
\end{align*}
Positive $\beta_D$ indicates slow dimensional growth with increasing energy.

At Planck scale ($\mu = \mu_{\text{Planck}}$):
\begin{align*}
  g_{\text{grav}} &\approx 1 \\
  \beta_D &\approx 0.1 \cdot \frac{1}{16\pi^2} \cdot (D - 4) + 0 + 0.01 \\
        &= 0.0006 (D - 4) + 0.01
\end{align*}
Fixed point: $\beta_D = 0 \implies D^* \approx 4 + 0.01/0.0006 \approx 21$ (intermediate Cayley-Dickson level).

%==============================================================================
% CONNECTION TO FRACTAL GEOMETRY
%==============================================================================

\paragraph{Fractal Interpretation:} The running dimension $D_{\text{eff}}(\mu)$ corresponds to the fractal dimension measured at resolution $\sim 1/\mu$:

\begin{equation}
  D_{\text{eff}}(\mu) = \lim_{\epsilon \to \hbar c/\mu} \frac{\log N(\epsilon)}{\log(1/\epsilon)}
  \label{eq:unified:fractal-rg-connection}
\end{equation}

where $N(\epsilon)$ is the box-counting function for spacetime structure at scale $\epsilon$. This unifies the RG picture with Hausdorff dimensional analysis.

%==============================================================================
% PHYSICAL INTERPRETATION
%==============================================================================

\paragraph{Physical Meaning:}
\begin{itemize}
  \item At macroscopic scales ($\mu \sim$ eV), spacetime appears strictly 4-dimensional
  \item Fractal corrections emerge at nuclear scales ($\mu \sim$ GeV), making $D_{\text{eff}} \approx 4.1-4.3$
  \item Hypercomplex structure (quaternions, octonions) becomes relevant at TeV-PeV scales
  \item Full Cayley-Dickson hierarchy accessible only at trans-Planckian energies
  \item Origami folding parameter $\lambda$ determines smoothness of dimensional transitions
  \item Strong scalar-ZPE coupling accelerates dimensional growth with energy
\end{itemize}

%==============================================================================
% EXPERIMENTAL PREDICTIONS
%==============================================================================

\paragraph{Experimental Tests:}
\begin{itemize}
  \item \textbf{High-energy scattering}: Deviations from 4D cross-sections at LHC/FCC energies
    \begin{equation*}
      \sigma(\mu) \propto \mu^{2-D_{\text{eff}}(\mu)} \quad \text{(modified dimensional scaling)}
    \end{equation*}
  \item \textbf{Gravitational wave propagation}: Extra polarization modes if $D_{\text{eff}} > 4$ at merger energies
  \item \textbf{Black hole thermodynamics}: Entropy should scale as $S \sim A^{D_{\text{eff}}/2}$ instead of $S \sim A$
  \item \textbf{Cosmic ray anomalies}: Ultra-high-energy cosmic rays probe $D_{\text{eff}} > 4$ regime
  \item \textbf{Dimensional spectroscopy}: Resonances at thresholds $\mu_{\text{threshold},n}$ detectable as sharp features in scattering amplitudes
\end{itemize}

%==============================================================================
% HOLOGRAPHIC INTERPRETATION
%==============================================================================

\paragraph{Holographic Duality:} Dimensional RG flow has holographic interpretation via AdS/CFT:
\begin{equation}
  D_{\text{eff}}(\mu) \longleftrightarrow D_{\text{AdS}}(r) \quad \text{with} \quad r \sim \frac{L_{\text{AdS}}^2}{\hbar c/\mu}
  \label{eq:unified:holographic-dimension}
\end{equation}

where $r$ is the AdS radial coordinate and $L_{\text{AdS}}$ is the AdS radius. Flow toward UV (large $\mu$) corresponds to moving into the AdS interior, where effective dimension increases.

%==============================================================================
% DEPENDENCIES AND CONNECTIONS
%==============================================================================
% Dependencies: Ch02 (Cayley-Dickson hierarchy provides dimensional targets)
%               Ch05 (Fractal calculus, Hausdorff dimension)
%               Ch07-Ch10 (Aether scalar-ZPE coupling)
%               Ch13 (Genesis origami parameter lambda)
%
% Forward references: Ch20 Section 6 (detailed RG derivation)
%                     Ch21 (Unified framework using scale-dependent dimension)
%                     Ch22-Ch24 (Experimental tests of dimensional running)
%
% Related equations: eq:unified:cayley-to-fractal (static dimensional mapping)
%                    eq:genesis:origami-folding (geometric folding mechanism)
%==============================================================================
