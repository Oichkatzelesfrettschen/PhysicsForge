%==============================================================================
% Equation: Complete Dimensional Mapping (Unified Framework)
% Source: Alpha001.06 (Cayley-Dickson sections), Maximal_Extraction_SET1_SET2.md
% Framework: Unified | Domain: MATH | Status: Theoretical
%==============================================================================
% Establishes the complete dimensional transformation mapping between:
%   - Integer Cayley-Dickson dimensions (2^n: 1, 2, 4, 8, ..., 2048)
%   - Fractal/origami dimensions (non-integer Hausdorff dimension)
%   - Negative dimensions (virtual/dual spaces)
%   - Exceptional Lie group embedding dimensions (E_8 = 248, etc.)
%
% This mapping resolves the apparent conflict between Aether's integer
% dimensional hierarchy and Genesis's fractal dimensional structure,
% showing they are complementary descriptions at different scales.
%==============================================================================

\begin{equation}
  \mathcal{D}_{\text{unified}}: \mathbb{D}_{\text{CD}} \leftrightarrow \mathbb{D}_{\text{fractal}}
    \leftrightarrow \mathbb{D}_{\text{negative}} \leftrightarrow \mathbb{D}_{\text{Lie}}
  \eqtag{U}{MATH}{T}
  \label{eq:unified:dimensional-mapping}
\end{equation}

%==============================================================================
% MAPPING COMPONENTS
%==============================================================================

% 1. Cayley-Dickson to Fractal Dimension Mapping
% Integer dimensions 2^n map to effective fractal dimensions via logarithmic scaling:
\begin{equation}
  D_{\text{fractal}}(n) = D_0 + \alpha \log_2(2^n) + \beta \sum_{k=1}^{n} \frac{1}{2^k}
  \label{eq:unified:cd-to-fractal}
\end{equation}
where:
\begin{itemize}
  \item $D_0$: Base fractal dimension (typically 3-4 for physical space)
  \item $\alpha$: Logarithmic scaling coefficient
  \item $\beta$: Fractal correction coefficient
  \item $n$: Cayley-Dickson iteration level ($n = 0, 1, 2, \ldots, 11$ for up to 2048D)
\end{itemize}

% 2. Fractal to Negative Dimension Mapping
% Fractal dimensions can extend into negative regime via analytic continuation:
\begin{equation}
  D_{\text{negative}}(D_f) = -\frac{D_f}{1 + D_f} \cdot \zeta(-D_f)
  \label{eq:unified:fractal-to-negative}
\end{equation}
where $\zeta(s)$ is the Riemann zeta function, providing regularization.

% 3. Lie Group Embedding Dimension Correspondence
% Exceptional Lie groups embed in Cayley-Dickson hierarchy:
\begin{equation}
  \begin{aligned}
    G_2 &\leftrightarrow \mathbb{O} \quad (\text{8D octonions}) \\
    F_4 &\leftrightarrow \mathbb{S} \quad (\text{16D sedenions, Jordan algebra}) \\
    E_6 &\leftrightarrow 2^5\text{D} \quad (\text{32D pathions}) \\
    E_7 &\leftrightarrow 2^6\text{D} \quad (\text{64D chingons}) \\
    E_8 &\leftrightarrow 2^7\text{D} \quad (\text{128D, extended to 248 roots})
  \end{aligned}
  \label{eq:unified:lie-cd-correspondence}
\end{equation}

%==============================================================================
% COMPLETE TRANSFORMATION FORMULA
%==============================================================================

% General dimensional transformation operator:
\begin{equation}
  \mathcal{T}_{\text{dim}}: D_{\text{in}} \mapsto D_{\text{out}}
    = \mathcal{F}_{\text{scale}}(D_{\text{in}}) \cdot \mathcal{P}_{\text{project}}
    \cdot \mathcal{E}_{\text{embed}}
  \label{eq:unified:dim-transform-operator}
\end{equation}

% Explicit components:
\begin{align}
  \mathcal{F}_{\text{scale}}(D) &= \exp\left( \gamma \log(D + 1) \right)
    \label{eq:unified:scale-function} \\
  \mathcal{P}_{\text{project}} &= \sum_{i} w_i \, P_i
    \quad \text{(projection onto subspaces)}
    \label{eq:unified:projection-operator} \\
  \mathcal{E}_{\text{embed}} &= \prod_{j} E_j^{\alpha_j}
    \quad \text{(exceptional group embeddings)}
    \label{eq:unified:embedding-operator}
\end{align}

%==============================================================================
% ORIGAMI DIMENSION FOLDING
%==============================================================================

% Origami folding relates higher dimensions to lower via geometric folding:
\begin{equation}
  D_{\text{origami}}(D_{\text{high}}, \theta)
    = D_{\text{low}} + \left( D_{\text{high}} - D_{\text{low}} \right)
      \cdot \cos^2\left( \frac{\theta}{2} \right)
  \label{eq:unified:origami-folding}
\end{equation}
where:
\begin{itemize}
  \item $D_{\text{high}}$: Higher dimensional space (e.g., 2048D)
  \item $D_{\text{low}}$: Lower dimensional projection (e.g., 4D)
  \item $\theta$: Folding angle ($\theta = 0$ fully unfolded, $\theta = \pi$ fully folded)
\end{itemize}

%==============================================================================
% SCALE-DEPENDENT DIMENSIONAL TRANSITION
%==============================================================================

% Effective dimension depends on probing scale (energy/length):
\begin{equation}
  D_{\text{eff}}(E) = D_{\text{base}} + \sum_{n=1}^{N} \Delta D_n
    \cdot \Theta\left( E - E_{\text{threshold},n} \right)
  \label{eq:unified:scale-dependent-dimension}
\end{equation}
where:
\begin{itemize}
  \item $D_{\text{base}}$: Macroscopic dimension (4D spacetime)
  \item $\Delta D_n$: Dimensional increment at threshold $n$
  \item $E_{\text{threshold},n}$: Energy scale where dimension $n$ becomes accessible
  \item $\Theta(x)$: Heaviside step function
\end{itemize}

% Example hierarchy:
\begin{equation}
  \begin{aligned}
    E < E_{\text{QCD}} &\implies D_{\text{eff}} = 4 \quad \text{(classical spacetime)} \\
    E_{\text{QCD}} < E < E_{\text{EW}} &\implies D_{\text{eff}} \approx 4 + \epsilon_1
      \quad \text{(fractal corrections)} \\
    E_{\text{EW}} < E < E_{\text{Planck}} &\implies D_{\text{eff}} \approx 8-16
      \quad \text{(hypercomplex structure)} \\
    E > E_{\text{Planck}} &\implies D_{\text{eff}} \to 248-2048
      \quad \text{(full dimensional hierarchy)}
  \end{aligned}
  \label{eq:unified:dimension-energy-hierarchy}
\end{equation}

%==============================================================================
% INVERSE MAPPING (Effective to Fundamental Dimensions)
%==============================================================================

% Given an effective fractal dimension, recover underlying Cayley-Dickson level:
\begin{equation}
  n_{\text{CD}}(D_{\text{fractal}}) = \left\lfloor
    \frac{D_{\text{fractal}} - D_0}{\alpha} + \mathcal{O}(\beta)
  \right\rfloor
  \label{eq:unified:inverse-cd-mapping}
\end{equation}

%==============================================================================
% PHYSICAL INTERPRETATION
%==============================================================================
% This dimensional mapping resolves the apparent paradox:
%   - Aether framework: Uses integer Cayley-Dickson dimensions (2, 4, 8, ..., 2048)
%   - Genesis framework: Uses fractal/origami dimensions (non-integer Hausdorff)
%   - Unified view: Integer dimensions are skeleton, fractal fills intermediate scales
%
% Physical meaning at different scales:
%   1. Macroscopic (E < GeV): Effective 4D spacetime
%   2. Nuclear (GeV < E < TeV): Fractal corrections, D ~ 4 + epsilon
%   3. Electroweak (TeV scale): Hypercomplex 8D structure becomes relevant
%   4. Planck scale: Full Cayley-Dickson hierarchy accessible
%   5. Trans-Planckian: Origami folding mediates between 2048D and lower dimensions
%
% Origami folding allows smooth transition between discrete dimensional levels,
% explaining how 2048D structure compactifies to observable 4D reality.
%
% Negative dimensions represent dual/virtual spaces (e.g., wormhole mouths,
% quantum tunneling paths) regularized via zeta function.
%
% Experimental implications:
%   - Dimensional transitions detectable via resonance spectroscopy
%   - Fractal corrections to scattering amplitudes at high energies
%   - Origami folding signatures in cosmic ray anomalies
%   - E_8 lattice fingerprints in crystal vibrational spectra
%
% See Ch20 for detailed derivations of each mapping component.
% See Ch02 for Cayley-Dickson construction details.
% See Ch05 for fractal calculus foundations.
%==============================================================================

% Dependencies: Ch02 (Cayley-Dickson), Ch03 (Lie groups), Ch05 (fractals),
%               Ch20 (dimensional reconciliation)
% Forward references: Ch22-26 (experimental validation of dimensional transitions)
%==============================================================================
