%==============================================================================
% Equation: Quantum Sensor Sensitivity to Scalar Fields
% Framework: Genesis | Domain: QM | Status: Experimental
%==============================================================================
\begin{equation}
  \delta\phi_{\text{min}} = \frac{\hbar}{\sqrt{N}\tau g}\sqrt{1 + \frac{T_2^*}{T_2}\left(\frac{\tau}{T_2}\right)^2}, \quad
  S_{\phi}(\omega) = \frac{2\delta\phi_{\text{min}}^2}{\Delta f}
  \eqtag{X}{QM}{E}
  \label{eq:sensor:quantum-sensitivity}
\end{equation}
% Notes: Minimum detectable scalar field amplitude using quantum sensors (NV centers,
% atomic clocks, or SQUIDs). The sensitivity $\delta\phi_{\text{min}}$ scales with number of
% qubits $N$, interrogation time $\tau$, coupling strength $g$, and coherence times
% $T_2$ (spin echo) and $T_2^*$ (free induction decay). The power spectral density $S_{\phi}$
% measures field fluctuations over bandwidth $\Delta f$. Diamond NV magnetometers achieve
% $\delta\phi_{\text{min}} \sim 10^{-19}$ eV with $N \sim 10^6$ spins, sufficient to detect
% Genesis nodespace oscillations if $g \gtrsim 10^{-3}$ Hz/eV. Decoherence from scalar
% noise ($T_2 < T_2^*$) degrades sensitivity but also provides a detection mechanism.
% Dependencies: eq:genesis:quasiparticle-excitation, eq:qubit-coherence-enhanced
%==============================================================================
