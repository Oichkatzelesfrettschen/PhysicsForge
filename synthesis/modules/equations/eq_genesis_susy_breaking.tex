%==============================================================================
% Equation Module: Genesis SUSY Breaking
% Source: Chapter 14 - Genesis Superforce Applications
% Date: 2025-10-23
%==============================================================================

\begin{equation}
  \boxed{
  M_{\text{SUSY}} = M_{\text{Planck}} \cdot \exp\left(-\frac{8\pi^2}{g^2 N_{\text{nodes}}^{1/3}}\right)
  }
  \eqtag{G}{SB}{MS}
  \label{eq:module:genesis-susy-scale}
\end{equation}

\noindent where:
\begin{itemize}[noitemsep]
  \item $M_{\text{SUSY}}$: Supersymmetry breaking scale
  \item $M_{\text{Planck}} = \sqrt{\hbar c / G} = 1.22 \times 10^{19}$ GeV
  \item $g$: Gauge coupling at high scale
  \item $N_{\text{nodes}}$: Total number of nodes in nodespace
\end{itemize}

\vspace{0.5em}
\noindent\textbf{F-term Breaking:}
\begin{equation}
  F_i = \frac{\partial W}{\partial \phi_i} + \mathcal{M}_{\text{fold}}[\phi_i] \neq 0
  \eqtag{G}{SB}{FT}
\end{equation}

\vspace{0.5em}
\noindent\textbf{Soft Breaking Lagrangian:}
\begin{equation}
  \mathcal{L}_{\text{soft}} = -\frac{1}{2} M_a \lambda_a \lambda_a - m_{ij}^2 \phi_i^* \phi_j - A_{ijk} y_{ijk} \phi_i \phi_j \phi_k + \text{h.c.}
  \eqtag{G}{SB}{LS}
\end{equation}

\vspace{0.5em}
\noindent\textbf{Gaugino Mass Ratios:}
\begin{equation}
  M_1 : M_2 : M_3 = 1 : \varphi : \varphi^2
  \eqtag{G}{SB}{MR}
\end{equation}
where $\varphi = (1+\sqrt{5})/2$ is the golden ratio.

\vspace{0.5em}
\noindent\textbf{Anomaly Mediation with Genesis:}
\begin{equation}
  m_{\lambda}^{\text{AMSB}} = \frac{\beta_g}{g} m_{3/2} \left(1 + \sum_{n=1}^{\infty} \frac{a_n}{\varphi^n} \right)
  \eqtag{G}{SB}{AM}
\end{equation}

\noindent\textbf{Physical Significance:} SUSY breaking emerges naturally from nodespace topology rather than being imposed. The golden ratio appears in mass ratios due to fractal scaling. The enormous number of nodes would preserve SUSY without dimensional folding, explaining why SUSY breaking requires special mechanisms.

%==============================================================================
% End of Equation Module
%==============================================================================