%==============================================================================
% Equation: Gravitational Wave Detection Sensitivity
% Framework: Pais | Domain: GR | Status: Experimental
%==============================================================================
\begin{equation}
  h_{\text{min}} = \sqrt{S_h(f)} = \sqrt{\frac{4k_B T}{m\omega_0 Q L^2} + \frac{\hbar}{m\omega_0 L^2} + S_{\text{shot}}(f) + S_{\phi}(f)}, \quad
  h = \frac{\Delta L}{L}
  \eqtag{P}{GR}{E}
  \label{eq:gw:detection-sensitivity}
\end{equation}
% Notes: Minimum detectable gravitational wave strain for interferometric detectors.
% The noise spectral density $S_h(f)$ includes thermal noise (first term with temperature $T$,
% test mass $m$, resonant frequency $\omega_0$, quality factor $Q$, and arm length $L$),
% quantum noise (second term from Heisenberg uncertainty), shot noise $S_{\text{shot}}$,
% and Aether scalar field noise $S_{\phi}$ which couples to the GW signal through the
% Pais framework. The strain $h$ is the fractional length change $\Delta L/L$. Advanced
% detectors like LIGO achieve $h_{\text{min}} \sim 10^{-23}$ at 100 Hz. The scalar noise
% $S_{\phi}$ provides a smoking-gun signature for Pais-Aether coupling in GW observations.
% Dependencies: eq:pais:gravitomagnetic-field, eq:aether:metric-perturbation
%==============================================================================
