%==============================================================================
% Equation: Origami Dimensional Folding Mechanism
% Source: math5GenesisFrameworkUnveiled.md (origami dimensions, lines 85-250)
%         Alpha001.06 (fold-merge operators, lines 7000-7500)
% Framework: Genesis | Domain: MATH | Status: Theoretical
%==============================================================================
% Mathematical description of Genesis framework's origami folding mechanism,
% which enables dimensional compactification from fundamental 2048D Cayley-
% Dickson structure down to observable 4D spacetime. This mechanism differs
% from Kaluza-Klein compactification through explicit geometric folding
% parameterized by folding angles.
%==============================================================================

\begin{equation}
  D_{\text{folded}}(D_{\text{high}}, \{\theta_i\}, \{w_i\})
    = D_{\text{low}} + \sum_{i=1}^{N_{\text{folds}}} w_i (D_{\text{high}} - D_{\text{low}})
      \cos^2\left(\frac{\theta_i}{2}\right) \prod_{j<i} \sin^2\left(\frac{\theta_j}{2}\right)
  \eqtag{G}{MATH}{T}
  \label{eq:genesis:origami-folding}
\end{equation}

\noindent
where:
\begin{itemize}
  \item $D_{\text{high}}$: Fundamental high-dimensional space (e.g., $2048$D Cayley-Dickson)
  \item $D_{\text{low}}$: Target low-dimensional projection (typically $4$D spacetime)
  \item $N_{\text{folds}}$: Number of sequential origami folds applied
  \item $\theta_i$: Folding angle for the $i$-th fold ($\theta_i \in [0, \pi]$)
  \item $w_i$: Weight factor for $i$-th fold (satisfying $\sum_{i=1}^{N_{\text{folds}}} w_i = 1$)
  \item Product term $\prod_{j<i} \sin^2(\theta_j/2)$: Sequential folding dependency
\end{itemize}

%==============================================================================
% LIMITING CASES
%==============================================================================

\paragraph{Limiting Behavior:}
\begin{itemize}
  \item \textbf{Fully unfolded} ($\theta_i = 0$ for all $i$):
    \begin{equation*}
      D_{\text{folded}} = D_{\text{low}} + (D_{\text{high}} - D_{\text{low}}) \sum_i w_i = D_{\text{high}}
    \end{equation*}
    All dimensions are accessible.

  \item \textbf{Fully folded} ($\theta_i = \pi$ for all $i$):
    \begin{equation*}
      D_{\text{folded}} = D_{\text{low}}
    \end{equation*}
    Only the base low-dimensional space remains observable.

  \item \textbf{Single fold} ($N_{\text{folds}} = 1$, $w_1 = 1$):
    \begin{equation*}
      D_{\text{folded}} = D_{\text{low}} + (D_{\text{high}} - D_{\text{low}}) \cos^2\left(\frac{\theta_1}{2}\right)
    \end{equation*}
    Simple interpolation between low and high dimensions.
\end{itemize}

%==============================================================================
% WORKED EXAMPLE: 2048D TO 4D
%==============================================================================

\paragraph{Worked Example:} Map $2048$D to $4$D via three sequential folds:
\begin{align*}
  D_{\text{high}} &= 2048 \\
  D_{\text{low}} &= 4 \\
  N_{\text{folds}} &= 3 \\
  \theta_1 &= \pi/3, \quad \theta_2 = \pi/4, \quad \theta_3 = \pi/2 \\
  w_1 &= 0.5, \quad w_2 = 0.3, \quad w_3 = 0.2
\end{align*}

Calculate each term:
\begin{align*}
  \text{Term 1:} &\quad 0.5 \cdot 2044 \cdot \cos^2(\pi/6) = 0.5 \cdot 2044 \cdot 0.75 = 766.5 \\
  \text{Term 2:} &\quad 0.3 \cdot 2044 \cdot \cos^2(\pi/8) \cdot \sin^2(\pi/6) \\
                 &= 0.3 \cdot 2044 \cdot 0.854 \cdot 0.25 = 131.2 \\
  \text{Term 3:} &\quad 0.2 \cdot 2044 \cdot \cos^2(\pi/4) \cdot \sin^2(\pi/6) \cdot \sin^2(\pi/8) \\
                 &= 0.2 \cdot 2044 \cdot 0.5 \cdot 0.25 \cdot 0.146 = 7.46
\end{align*}

Therefore:
\begin{equation*}
  D_{\text{folded}} = 4 + 766.5 + 131.2 + 7.46 \approx 909.2
\end{equation*}

This intermediate folding leaves an effective $\sim 909$D structure, requiring additional folds or different parameters to achieve full compactification to $4$D.

%==============================================================================
% COMPLETE 2048D TO 4D FOLDING
%==============================================================================

\paragraph{Complete Compactification:} For maximal folding to $4$D, use:
\begin{equation}
  \theta_i = \pi - \epsilon_i \quad \text{with} \quad \epsilon_i \ll 1
  \label{eq:genesis:maximal-folding}
\end{equation}

In the limit $\epsilon_i \to 0$, all folds approach $\theta_i = \pi$ and $D_{\text{folded}} \to D_{\text{low}} = 4$.

Alternatively, employ hierarchical folding with exponentially weighted angles:
\begin{equation}
  \theta_i = \pi \left(1 - 2^{-i}\right), \quad w_i = \frac{2^{-i}}{\sum_{j=1}^{N} 2^{-j}}
  \label{eq:genesis:hierarchical-folding}
\end{equation}

This ensures systematic dimensional reduction from $2048$D through intermediate Cayley-Dickson levels (1024D, 512D, 256D, ..., 8D, 4D).

%==============================================================================
% COMPARISON TO KALUZA-KLEIN COMPACTIFICATION
%==============================================================================

\paragraph{Origami vs Kaluza-Klein:}

\begin{center}
\begin{tabular}{lll}
\toprule
\textbf{Feature} & \textbf{Origami Folding} & \textbf{Kaluza-Klein} \\
\midrule
Mechanism & Geometric folding (angles $\theta_i$) & Topological compactification \\
Parameters & Folding angles, weights & Compactification radii $R_i$ \\
Dimension change & Continuous via $\cos^2(\theta/2)$ & Discrete (compact vs non-compact) \\
Observable effects & Fractal corrections to scattering & Kaluza-Klein tower of massive modes \\
Energy scale & $E \sim \hbar c / (a_0 \theta)$ & $E \sim \hbar c / R$ \\
Flexibility & Adjustable folding patterns & Fixed topology (e.g., tori, Calabi-Yau) \\
\bottomrule
\end{tabular}
\end{center}

Key distinction: Origami folding allows \emph{continuous} variation of effective dimensionality through angular parameters, whereas Kaluza-Klein yields discrete spectra of compactified modes. Both mechanisms can coexist, with origami providing smooth transitions between Kaluza-Klein plateaus.

%==============================================================================
% PHYSICAL INTERPRETATION
%==============================================================================

\paragraph{Physical Meaning:}
\begin{itemize}
  \item Origami folding represents a \emph{dynamical} compactification where effective dimensionality varies with local spacetime curvature, scalar field configurations, and ZPE density
  \item Folding angles $\theta_i$ may be tied to vacuum expectation values of scalar fields, making dimensional structure environment-dependent
  \item The sequential product $\prod_{j<i} \sin^2(\theta_j/2)$ ensures that earlier folds modulate the effectiveness of later folds, creating hierarchical structure
  \item In high-ZPE regions (near black holes, cosmological singularities), folding may partially reverse ($\theta_i \to 0$), locally exposing higher dimensions
  \item Observable 4D spacetime emerges as an effective low-energy description with nearly complete folding ($\theta_i \approx \pi$)
\end{itemize}

%==============================================================================
% EXPERIMENTAL SIGNATURES
%==============================================================================

\paragraph{Experimental Tests:}
\begin{itemize}
  \item \textbf{Dimensional resonances}: Partial unfolding at high energies should produce resonances at $E_{\text{res},i} \sim \hbar c/(a_0 \theta_i)$
  \item \textbf{Gravitational wave polarization}: Extra polarization modes if dimensions partially unfold during black hole mergers
  \item \textbf{Collider anomalies}: Deviations from 4D scattering amplitudes at TeV scale if folding is incomplete
  \item \textbf{Casimir force modifications}: Folding geometry alters boundary conditions, producing measurable force corrections
  \item \textbf{Cosmological imprints}: Early universe may have had different folding configuration, leaving signatures in CMB
\end{itemize}

%==============================================================================
% FOLD-MERGE OPERATOR CONNECTION
%==============================================================================

\paragraph{Fold-Merge Operator:} The Genesis framework defines the fold-merge operator $\mathcal{F}\mathcal{M}$ (Alpha001.06) as:
\begin{equation}
  \mathcal{F}\mathcal{M} = K_{\text{origami-folding}}(x,t) \cdot K_{\text{recursive-fractal}}(x,t)
    \cdot K_{\text{modular-symmetry}}(x)
  \label{eq:genesis:fold-merge-operator}
\end{equation}

The origami folding kernel $K_{\text{origami-folding}}$ is constructed from the dimensional folding formula via:
\begin{equation}
  K_{\text{origami-folding}}(x,t) = \exp\left( -\frac{1}{2} \sum_{i=1}^{N}
    \frac{(\theta_i(x,t) - \theta_{i,0})^2}{\sigma_i^2} \right)
  \label{eq:genesis:origami-kernel}
\end{equation}

where $\theta_i(x,t)$ are spacetime-dependent folding angles, $\theta_{i,0}$ are equilibrium values, and $\sigma_i$ are folding fluctuation widths. This connects the geometric folding mechanism to quantum field kernel formalism.

%==============================================================================
% DEPENDENCIES AND CONNECTIONS
%==============================================================================
% Dependencies: Ch02 (Cayley-Dickson 2048D structure)
%               Ch13 (Genesis origami framework)
%               Ch18 (Dimensional conflict resolution)
%
% Forward references: Ch20 Section 5 (detailed origami folding derivation)
%                     Ch21 (Unified dimensional synthesis)
%                     Ch23 (Experimental dimensional spectroscopy)
%
% See also: eq:unified:origami-folding in eq_dimensional_mapping_unified.tex
%           for simplified single-fold formula
%==============================================================================
