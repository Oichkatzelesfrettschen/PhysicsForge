%==============================================================================
% Equation Module: GEM Gravitomagnetic Field Definition
% Source: PAIS_UNIFIED.md Section 2.6 (frame-dragging and Lense-Thirring)
% Tag: eq:pais:gravitomagnetic-field
%==============================================================================

\begin{equation}
  \mathbf{B}_g = \nabla \times \mathbf{A}_g = \frac{4G}{c^2} \frac{\mathbf{J} \times \mathbf{r}}{r^3}
  \eqtag{G}{GEM}{BG}
  \label{eq:pais:gravitomagnetic-field}
\end{equation}

\noindent\textbf{Physical Interpretation:} The gravitomagnetic field $\mathbf{B}_g$ is generated by mass currents $\mathbf{J} = \rho_m \mathbf{v}$ (moving matter) in exact analogy to how magnetic fields arise from electric currents in electromagnetism. The gravitomagnetic vector potential is $\mathbf{A}_g = (4G/c^2) \mathbf{J} \times \mathbf{r}/r^3$ for a localized current distribution.

\noindent\textbf{Frame-Dragging Effect:} A rotating mass with angular momentum $\mathbf{L}$ produces a gravitomagnetic dipole field:
\begin{equation*}
  \mathbf{B}_g = \frac{G}{c^2 r^3} \left[3(\mathbf{L} \cdot \hat{\mathbf{r}})\hat{\mathbf{r}} - \mathbf{L}\right]
\end{equation*}
This field causes gyroscopes to precess (Lense-Thirring effect), confirmed by Gravity Probe B with precession rate $\Omega_{\mathrm{LT}} = 37.2 \pm 7.2$ milliarcsec/year for Earth.

\noindent\textbf{Magnitude Estimation:} For Earth's rotation ($M_{\oplus} = 5.97 \times 10^{24}$ kg, $\omega_{\oplus} = 7.29 \times 10^{-5}$ rad/s), at orbital radius $r = 7000$ km:
\begin{equation*}
  |\mathbf{B}_g| \sim \frac{G M_{\oplus} R_{\oplus}^2 \omega_{\oplus}}{c^2 r^3} \sim 10^{-14} \text{ s}^{-1}
\end{equation*}
Vastly weaker than typical magnetic fields ($B_{\mathrm{Earth}} \sim 10^{-4}$ T $\equiv 10^8$ s$^{-1}$ in angular frequency units), explaining the difficulty of detecting gravitomagnetic effects.
