%==============================================================================
% Equation: Pais Tensor Gauge Theory Formulation
% Framework: Pais | Domain: GR | Status: Theoretical
%==============================================================================
\begin{equation}
  F_{\mu\nu\rho} = \partial_\mu h_{\nu\rho} + \partial_\nu h_{\rho\mu} + \partial_\rho h_{\mu\nu} - \frac{1}{2}(g_{\mu\nu}\partial_\rho h + g_{\nu\rho}\partial_\mu h + g_{\rho\mu}\partial_\nu h)
  \eqtag{G}{GR}{T}
  \label{eq:pais:tensor-gauge-theory}
\end{equation}
% Notes: Tensor gauge field strength for the Pais framework, where $h_{\mu\nu}$ is the
% metric perturbation tensor and $h = g^{\alpha\beta}h_{\alpha\beta}$ is its trace.
% This three-index tensor generalizes the electromagnetic field strength tensor $F_{\mu\nu}$
% to gravitational interactions, capturing torsion and non-Riemannian geometry effects.
% The symmetrization structure ensures gauge invariance under $h_{\mu\nu} \to h_{\mu\nu} + \partial_\mu\xi_\nu + \partial_\nu\xi_\mu$.
% In the weak-field limit, this reduces to linearized GEM theory, while at strong coupling
% it enables descriptions of spacetime torsion and the Pais superforce unification.
% Dependencies: eq:pais:superforce, eq:pais:gravitational-force
%==============================================================================
