%==============================================================================
% Equation Module: Pais Vacuum Energy Density
% Source: PAIS_UNIFIED.md - Superforce acting on spacetime geometry
% Tag: eq:pais:vacuum-energy-density
%==============================================================================

\begin{equation}
  \rho_{\mathrm{vac}} = \frac{F_{\mathrm{S}}}{4\pi r_P^2} = \frac{c^4}{4\pi G r_P^2} = \frac{c^7}{4\pi G^2 \hbar}
  \eqtag{G}{VAC}{ENERGY}
  \label{eq:pais:vacuum-energy-density}
\end{equation}

\noindent\textbf{Physical Interpretation:} The energy density of the vacuum at the Planck scale emerges from the Superforce acting over a Planck area. This represents the fundamental energy density scale of spacetime itself, with numerical value approximately $\rho_{\mathrm{vac}} \sim 5.16 \times 10^{113}$ J/m$^3$. This is the natural vacuum energy density before renormalization.

\noindent\textbf{Connection to Cosmological Constant:} The observed dark energy density $\rho_{\Lambda} \sim 10^{-9}$ J/m$^3$ is $\sim 10^{122}$ times smaller, constituting the cosmological constant problem. The Pais framework suggests that only long-wavelength coherent modes contribute to observable $\rho_{\Lambda}$, with short-wavelength fluctuations canceling.
