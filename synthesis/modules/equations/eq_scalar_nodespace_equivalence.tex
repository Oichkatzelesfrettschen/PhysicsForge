%==============================================================================
% Equation: Scalar field as coarse-grained nodespace density
% Source: Ch18 (Section 3.2: Emergent Continuum from Nodespace)
%         FRAMEWORK_CONFLICT_MATRIX_ANALYSIS.md (nodespace-lattice mapping)
% Framework: Unified (connects Aether scalar fields to Genesis nodespaces)
% Domain: QM + GR | Status: Theoretical (continuum limit derivation)
%==============================================================================
\begin{equation}
  \varphi(x,t) = \frac{1}{\bar{n}} \left( \frac{N(V,t)}{V} - \bar{n} \right)
  \eqtag{P}{QM}{T}
  \label{eq:unified:scalar_nodespace}
\end{equation}
% Notes: Establishes that Aether scalar field phi(x,t) emerges as the
% coarse-grained deviation of nodespace density from equilibrium.
%
% Parameters:
% - phi(x,t): Aether scalar field (continuum description)
% - N(V,t): Number of Genesis nodespaces within averaging volume V at time t
% - V: Coarse-graining volume (must satisfy V >> l_node^3 for continuum limit)
% - n_bar: Equilibrium nodespace number density (background)
%
% Physical interpretation:
% - At scales lambda >> l_node ~ l_Planck, discrete nodespace network
%   averages to continuous scalar field
% - Scalar field fluctuations = local density perturbations in nodespace graph
% - Quantum foam (Aether) = residual discreteness in continuum limit
%
% Continuum limit conditions:
% - Averaging volume: V ~ (10^{-30} m)^3 >> l_Planck^3 ~ (10^{-35} m)^3
% - Number of nodespaces in V: N >> 10^15 (ensures statistical averaging)
% - Observation timescale: tau_obs >> tau_foam ~ 10^{-43} s
%
% Dependencies: Ch08 (Aether scalar fields), Ch12 (Genesis nodespaces)
% Experimental signature: Breakdown of continuum description at lambda < 10^{-33} m
% would directly test this equivalence (requires beyond-Planck-scale probes).
%==============================================================================
