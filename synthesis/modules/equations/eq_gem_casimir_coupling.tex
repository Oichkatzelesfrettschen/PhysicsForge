%==============================================================================
% Equation Module: Pais-Casimir Effect Coupling
% Source: PAIS_UNIFIED.md Section 2.7 (GEM Casimir modifications)
% Tag: eq:pais:casimir-coupling
%==============================================================================

\begin{equation}
  F_{\mathrm{Casimir}}^{\mathrm{GEM}} = F_{\mathrm{Casimir}}^{\mathrm{EM}} \left(1 + \frac{G\rho}{c^2}\right) = -\frac{\pi^2 \hbar c}{240 d^4} \left(1 + \frac{G\rho}{c^2}\right)
  \eqtag{G}{CASIMIR}{GEM}
  \label{eq:pais:casimir-coupling}
\end{equation}

\noindent\textbf{Physical Interpretation:} GEM theory modifies the Casimir force through gravitational contributions coupled to the vacuum energy between conducting plates separated by distance $d$. The correction factor $(1 + G\rho/c^2)$ is typically $\sim 10^{-27}$ for laboratory densities $\rho \sim 10^3$ kg/m$^3$, but becomes significant near neutron stars where $\rho \sim 10^{17}$ kg/m$^3$.

\noindent\textbf{Enhancement Mechanism:} The coupling enables enhanced vacuum energy extraction through combined electromagnetic-gravitational engineering. By engineering both electromagnetic boundary conditions (plate geometry, material properties) and gravitational field gradients (mass distributions, rotating systems), the Casimir force can be modulated beyond purely electromagnetic effects.

\noindent\textbf{Experimental Implications:} Precision Casimir force measurements in proximity to massive rotating bodies could detect the $G\rho/c^2$ correction. For Earth's density at surface, the correction is $\sim 10^{-27}$, requiring $10^{-30}$ N force resolution--approaching feasibility with cryogenic torsion balances and SQUID sensors.
