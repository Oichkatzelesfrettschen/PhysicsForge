%==============================================================================
% EQUATION MODULE: Genesis Planck Scale Discretization
% ID: eq_genesis_planck_scale_discretization.tex
% DEPENDENCIES: Planck units
%==============================================================================

\begin{equation}
  \Delta x \geq l_P, \quad \Delta t \geq t_P = \frac{l_P}{c}
  \eqtag{G}{PLANCK}{Q}
  \label{eq:genesis:planck-quantization}
\end{equation}

%------------------------------------------------------------------------------
% EQUATION DESCRIPTION
%------------------------------------------------------------------------------
% Fundamental quantization of spacetime at the Planck scale in the Genesis
% nodespace framework. Space and time cannot be subdivided below these scales.
%
% QUANTITIES:
% - Delta x: Minimum spatial separation between nodes
% - Delta t: Minimum temporal separation between events
% - l_P: Planck length = sqrt(hbar*G/c^3) ~ 1.616 x 10^-35 m
% - t_P: Planck time = l_P/c = sqrt(hbar*G/c^5) ~ 5.391 x 10^-44 s
% - c: Speed of light in vacuum
%
% PHYSICAL INTERPRETATION:
% - Below these scales, the concept of continuous spacetime breaks down
% - Quantum foam dominates at sub-Planck scales
% - Nodespace graph structure emerges from this discretization
% - Continuous spacetime is statistical approximation for large N nodes
%
% CONSEQUENCES:
% 1. UV cutoff for quantum field theory
% 2. Natural regularization of infinities
% 3. Minimum information content per volume: 1 bit per Planck volume
% 4. Holographic bound: S <= A/(4*l_P^2)
%
% EMERGENCE OF CONTINUITY:
% In limit N -> infinity with fixed total volume V:
% - Node density: n = N/V ~ 1/l_P^3
% - Continuum metric emerges from graph distance
% - Smooth manifold approximation valid for L >> l_P
%
% USAGE:
% Chapter 11, Section "Nodespaces: Fundamental Discrete Structure"
% Foundation for all discrete nodespace calculations
%==============================================================================