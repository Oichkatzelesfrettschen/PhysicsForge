%==============================================================================
% Equation: Higher-Dimensional Qudit State Space (Cayley-Dickson)
% Source: Cayley-Dickson construction applied to quantum information theory
% Framework: Mathematical | Domain: QM | Status: Theoretical
%==============================================================================
\begin{equation}
  \ket{\psi}_D = \sum_{i=0}^{D-1} c_i \ket{i}_D, \quad
  \sum_{i=0}^{D-1} |c_i|^2 = 1, \quad
  D = 2^n
  \eqtag{M}{QM}{T}
  \label{eq:quantum:higher-dim-qudit}
\end{equation}
%
% with computational basis operations defined via Cayley-Dickson algebra:
%
\begin{equation}
  \hat{U}_{\text{CD}}^{(n)} \ket{j}_D \ket{k}_D =
  \ket{(j \otimes_{\text{CD}} k) \bmod D}_D
  \label{eq:quantum:cayley-dickson-gate}
\end{equation}
%
% where:
%   |psi>_D  = quantum state in D-dimensional Hilbert space
%   c_i      = complex amplitudes (i = 0, 1, ..., D-1)
%   |i>_D    = computational basis states
%   D        = qudit dimension (power of 2 for Cayley-Dickson)
%   n        = Cayley-Dickson construction level (C: n=1, H: n=2, O: n=3, ...)
%   U_CD^{(n)} = unitary gate using n-level Cayley-Dickson multiplication
%   tensor_CD = Cayley-Dickson multiplication operation
%
% Dimensional hierarchy:
%   n=1 (D=2):    qubits (standard quantum computing)
%   n=2 (D=4):    ququarts (quaternionic quantum mechanics)
%   n=3 (D=8):    octonionic qudits (non-associative gates)
%   n=4 (D=16):   sedenionic qudits
%   n=5 (D=32):   pathionic qudits
%   ...
%   n=11 (D=2048): maximal Cayley-Dickson quantum states
%
% Computational advantages:
% - Certain graph isomorphism problems: O(D log D) vs O(D^2) classical
% - Grover search in D dimensions: O(sqrt(D)) queries (quadratic speedup)
% - Quantum simulation of D-level systems: native representation
%
% Challenges:
% - Higher dimensions lose commutativity (n>1), associativity (n>2)
% - Error rates scale with dimension: epsilon_D ~ D * epsilon_2
% - Gate complexity increases: Gates in O(n) ~ O(D^2) for general unitaries
%
% Experimental implementations:
% - Photonic qudits: Orbital angular momentum (D=8-16 demonstrated)
% - Trapped ions: Multiple electronic levels (D=4-8)
% - Superconducting: Transmon higher levels (D=3-5, anharmonicity limited)
%
% Dependencies: Ch02 (Cayley-Dickson construction), Ch01 (quantum mechanics)
% Cross-references: Ch27 (quantum algorithms), Ch03 (Lie group gates)
%==============================================================================
