%==============================================================================
% Equation Module: Pais Framework and Cosmological Constant Problem
% Source: PAIS_UNIFIED.md Section 1.3 (vacuum energy implications)
% Tag: eq:pais:cosmological-constant
%==============================================================================

\begin{equation}
  \Lambda_{\mathrm{eff}} = \frac{8\pi G}{c^4} \rho_{\mathrm{vac}}^{\mathrm{eff}} = \frac{8\pi G}{c^4} \left\langle \rho_{\mathrm{vac}} \right\rangle_{\mathrm{coherent}}
  \eqtag{G}{COSM}{CONST}
  \label{eq:pais:cosmological-constant}
\end{equation}

\noindent\textbf{Cosmological Constant Problem:} The vacuum energy density from quantum field theory (QFT) naively predicts $\rho_{\mathrm{vac}}^{\mathrm{QFT}} \sim E_P^4/(\hbar c)^3 \sim 10^{113}$ J/m$^3$, while observations constrain $\rho_{\Lambda}^{\mathrm{obs}} \sim 10^{-9}$ J/m$^3$--a discrepancy of $\sim 10^{122}$ orders of magnitude, the worst prediction in physics.

\noindent\textbf{Pais Coherence Resolution:} The Superforce framework proposes that only long-wavelength, coherent modes of vacuum fluctuations contribute to the effective cosmological constant:
\begin{equation*}
  \left\langle \rho_{\mathrm{vac}} \right\rangle_{\mathrm{coherent}} = \int_{0}^{\lambda_{\mathrm{max}}} \rho_{\mathrm{vac}}(k) \, \Theta_{\mathrm{coherence}}(k) \, dk
\end{equation*}
where $\Theta_{\mathrm{coherence}}(k)$ is a coherence function that suppresses contributions from wavelengths $\lambda < \lambda_{\mathrm{crit}}$ due to phase cancellation. The critical wavelength is set by scalar field dynamics: $\lambda_{\mathrm{crit}} \sim \hbar/(m_{\phi} c)$ where $m_{\phi}$ is the scalar mediator mass.

\noindent\textbf{Dimensional Verification:} For $m_{\phi} \sim 10^{-3}$ eV/c$^2$ (dark energy scale), $\lambda_{\mathrm{crit}} \sim 0.2$ mm. Modes with $\lambda \gg \lambda_{\mathrm{crit}}$ maintain coherence over cosmological timescales, contributing $\rho_{\Lambda}^{\mathrm{eff}} \sim (m_{\phi} c^2)^4/(\hbar c)^3 \sim 10^{-9}$ J/m$^3$, matching observations.
