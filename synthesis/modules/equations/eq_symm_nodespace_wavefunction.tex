%==============================================================================
% EQUATION MODULE: Genesis Nodespace Wavefunction
% ID: eq_genesis_nodespace_wavefunction.tex
% DEPENDENCIES: Nodespace basis states, Hilbert space
%==============================================================================

\begin{equation}
  |\Psi_{\text{node}}\rangle = \sum_i c_i |\mathcal{N}_i\rangle
  \eqtag{X}{NODE}{PSI}
  \label{eq:genesis:nodespace-wavefunction}
\end{equation}

%------------------------------------------------------------------------------
% EQUATION DESCRIPTION
%------------------------------------------------------------------------------
% The nodespace wavefunction describes the quantum state of a nodespace region
% as a superposition of basis states in the nodespace Hilbert space.
%
% COMPONENTS:
% - |Psi_node>: Quantum state vector of the nodespace region
% - c_i: Complex amplitude coefficients (normalized: sum_i |c_i|^2 = 1)
% - |N_i>: Basis states of the nodespace Hilbert space
%
% MATHEMATICAL STRUCTURE:
% - Lives in Hilbert space H_node
% - Basis states |N_i> form complete orthonormal set
% - Inner product: <N_i|N_j> = delta_{ij}
% - Completeness: sum_i |N_i><N_i| = I
%
% PHYSICAL MEANING:
% Each nodespace region exists in a quantum superposition of configurations.
% The coefficients c_i determine probability amplitudes for finding the
% nodespace in configuration |N_i> upon measurement.
%
% EVOLUTION:
% Evolves according to nodespace Schrodinger equation:
% i*hbar d|Psi>/dt = H_node |Psi>
%
% ENTANGLEMENT:
% Different nodespace regions can be entangled:
% |Psi_total> != |Psi_A> ⊗ |Psi_B>
%
% USAGE:
% Chapter 11, Section "Nodespace Wavefunction"
% Foundation for quantum nodespace dynamics
%==============================================================================