%==============================================================================
% Equation: Mission Delta-V Budget for Field Propulsion
% Framework: Pais | Domain: PROP | Status: Speculative
%==============================================================================
\begin{equation}
  \Delta v_{\text{total}} = \int_0^{t_{\text{burn}}} \frac{F_T(t)}{m(t)}dt = v_{\text{exhaust}}\ln\frac{m_0}{m_f} + \Delta v_{\text{field}}\left[1 - \exp\left(-\frac{E_{\text{field}}}{m_0 c^2}\right)\right]
  \eqtag{G}{PROP}{S}
  \label{eq:mission:delta-v-budget}
\end{equation}
% Notes: Total velocity change available for spacecraft using combined chemical and field
% propulsion. The first term is the classical Tsiolkovsky rocket equation with exhaust velocity
% $v_{\text{exhaust}}$, initial mass $m_0$, and final mass $m_f$. The second term represents
% additional delta-v from field propulsion with characteristic velocity $\Delta v_{\text{field}}$
% and field energy $E_{\text{field}}$. The exponential factor captures relativistic limitations
% and energy-to-mass conversion efficiency. Conventional rockets achieve $\Delta v \lesssim 15$ km/s;
% field propulsion could provide $\Delta v_{\text{field}} \sim 0.1c$ if $E_{\text{field}}/m_0c^2 \gtrsim 0.1$.
% Interstellar missions to Proxima Centauri require $\Delta v \sim 0.3c$, demanding GW-class
% power systems and Pais-effect thrust generation.
% Dependencies: eq:propulsion:power-budget, eq:pais:inertia-reduction
%==============================================================================
