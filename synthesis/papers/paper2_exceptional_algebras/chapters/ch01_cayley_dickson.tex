%==============================================================================
% PAPER 2, CHAPTER 1: Cayley-Dickson Algebras
%==============================================================================

\chapter{Cayley-Dickson Algebras: Beyond Complex Numbers}
\label{ch:p2:cayley-dickson}

\marginphysics{Electron spin requires quaternions; string theory requires octonions}

\section{The Spin Mystery: Why Quantum Mechanics Needs More Than Complex Numbers}

When physicists first discovered electron spin in the 1920s, complex numbers were not enough. A spinning electron does not behave like a rotating ball---it requires \emph{two} full rotations (720 degrees) to return to its original quantum state.\marginhistory{Stern-Gerlach experiment, 1922: First demonstration of spin quantization} One rotation by 360 degrees changes the wavefunction's sign.

This bizarre property demands a number system beyond the complex plane. Wolfgang Pauli solved the puzzle with his famous spin matrices:
\begin{equation}
  \sigma_x = \begin{pmatrix} 0 & 1 \\ 1 & 0 \end{pmatrix}, \quad
  \sigma_y = \begin{pmatrix} 0 & -i \\ i & 0 \end{pmatrix}, \quad
  \sigma_z = \begin{pmatrix} 1 & 0 \\ 0 & -1 \end{pmatrix}
  \label{eq:p2:cayley:pauli-intro}
\end{equation}

These matrices satisfy $\sigma_i \sigma_j = i\epsilon_{ijk} \sigma_k$.\marginmath{Levi-Civita tensor $\epsilon_{ijk}$ encodes cross product} But there's a deeper pattern: these are the imaginary units of \textbf{quaternions}---the four-dimensional number system discovered by William Rowan Hamilton in 1843.

Hamilton famously carved the quaternion multiplication rules into a bridge in Dublin:\marginhistory{Brougham Bridge, Dublin, October 16, 1843}
\begin{equation}
  i^2 = j^2 = k^2 = ijk = -1
  \label{eq:p2:cayley:hamilton-carving}
\end{equation}

But nature does not stop at four dimensions. String theory requires ten dimensions. M-theory requires eleven. Grand unified theories embed the Standard Model in exceptional Lie groups living in 78, 133, or 248 dimensions.\marginphysics{Standard Model fits inside $E_6$ (78D) or $E_8$ (248D)}

\textbf{How do we build number systems for these higher dimensions?} The answer is the Cayley-Dickson construction: a recursive doubling process that generates $2^n$-dimensional algebras from one-dimensional real numbers up to 2048 dimensions and beyond.

Here's the remarkable fact: \textbf{every doubling costs us an algebraic property}.\margincaution{Loss of structure enables richer physics}
\begin{itemize}
  \item After $\mathbb{C}$ (2D): Commutativity lost. $ij \neq ji$.
  \item After $\mathbb{H}$ (4D): Associativity lost. $(xy)z \neq x(yz)$.
  \item After $\mathbb{O}$ (8D): Division algebra property lost. Zero divisors appear.
\end{itemize}

\section{The Doubling Principle}
\label{sec:p2:cayley:construction}

\subsection{Motivation: Why Pairs?}

The clever insight: \textbf{treat elements of the new algebra as ordered pairs} from the old algebra.\marginmath{Ordered pairs $(a,b)$ create next dimension} This is exactly how we construct complex numbers from reals:
\begin{equation}
  z = a + bi = (a, b) \quad \text{where } a, b \in \mathbb{R}
\end{equation}

Complex multiplication $(a_1, b_1) \cdot (a_2, b_2) = (a_1 a_2 - b_1 b_2, a_1 b_2 + a_2 b_1)$ emerges from $i^2 = -1$.\margincomp{Check: $(0,1)(0,1) = (-1,0)$}

\subsection{The Recursive Hierarchy}

Starting from $\mathbb{R}$ (1D), each doubling creates a new algebra:\marginphysics{Each algebra has distinct physical applications}
\begin{equation}
  \mathbb{R} \xrightarrow{2D} \mathbb{C} \xrightarrow{4D} \mathbb{H} \xrightarrow{8D} \mathbb{O} \xrightarrow{16D} \mathbb{S} \xrightarrow{32D} \mathbb{P}
\end{equation}

\textbf{The algebras}:
\begin{itemize}
  \item $\mathbb{R}$ (1D): Real numbers
  \item $\mathbb{C}$ (2D): Complex numbers
  \item $\mathbb{H}$ (4D): Quaternions (Hamilton, 1843)
  \item $\mathbb{O}$ (8D): Octonions (Graves/Cayley, 1845)
  \item $\mathbb{S}$ (16D): Sedenions
  \item $\mathbb{P}$ (32D): Pathions
\end{itemize}

At each step, the dimension doubles: $\dim(\mathcal{A}_{n+1}) = 2 \cdot \dim(\mathcal{A}_n)$.

\subsection{The Universal Multiplication Rule}

Elements of $\mathcal{A}_{n+1}$ are pairs $(a,b)$ with $a,b \in \mathcal{A}_n$:\marginmath{Single formula generates all algebras}
\begin{equation}
  (a,b) \cdot (c,d) = (ac - d^* b, da + bc^*)
  \label{eq:p2:cayley:formula}
\end{equation}

The conjugation operation is recursive: $(a,b)^* = (a^*, -b)$.

\subsection{Norm Preservation}

The quadratic norm is:\marginphysics{Norm-squared = Probability density in QM}
\begin{equation}
  \|x\|^2 = x \cdot x^* = \sum_{i=1}^{2^n} x_i^2
\end{equation}

For algebras through pathions (32D), the norm is multiplicative:\marginmath{$\|xy\| = \|x\|\,\|y\|$ through 32D}
\begin{equation}
  \|xy\| = \|x\|\,\|y\|
\end{equation}

\textbf{Physical consequence}: Probability conservation in quantum mechanics.

\section{The Classical Division Algebras}

\subsection{Real Numbers $\mathbb{R}$ (1D)}

All desirable properties:\marginmath{Commutative + Associative + Division}
\begin{itemize}
  \item Commutative: $ab = ba$
  \item Associative: $(ab)c = a(bc)$
  \item Division algebra: $ab = 0 \implies a=0$ or $b=0$
  \item Normed: $|ab| = |a|\,|b|$
\end{itemize}

\subsection{Complex Numbers $\mathbb{C}$ (2D)}

Complex numbers $z = a + bi$ with $i^2 = -1$.\marginphysics{Quantum interference requires complex amplitudes}

\textbf{Worked example}: $(3 + 4i)(1 + 2i)$:\marginex{Standard FOIL expansion}
\begin{align}
  (3 + 4i)(1 + 2i) &= 3 + 6i + 4i + 8i^2 \nonumber \\
  &= 3 + 10i - 8 = -5 + 10i
\end{align}

Check norm: $|3+4i| = 5$, $|1+2i| = \sqrt{5}$, $|-5+10i| = 5\sqrt{5}$. Indeed $5 \cdot \sqrt{5} = 5\sqrt{5}$.\margincomp{Norm multiplicativity verified}

\subsection{Quaternions $\mathbb{H}$ (4D)}

Quaternions $q = a + bi + cj + dk$ with three imaginary units:\marginmath{Three units: $i, j, k$}
\begin{equation}
  i^2 = j^2 = k^2 = ijk = -1
\end{equation}

Multiplication table:\marginphysics{Non-commutative: $ij = k \neq ji = -k$}
\begin{equation}
  \begin{array}{c|cccc}
    \cdot & 1 & i & j & k \\
    \hline
    1 & 1 & i & j & k \\
    i & i & -1 & k & -j \\
    j & j & -k & -1 & i \\
    k & k & j & -i & -1
  \end{array}
\end{equation}

\textbf{Worked example}: $(1 + i)(j + k) = j + k + k - j = 2k$\marginex{Expand using distributive law}

Reverse order: $(j + k)(1 + i) = j - k + k + j = 2j$. Different results!\margincaution{Non-commutativity at 4D}

\textbf{Physical significance}: Quaternions describe 3D rotations.\marginphysics{3D rotations → Unit quaternions} A rotation by angle $\theta$ about axis $\mathbf{n}$ is:
\begin{equation}
  q = \cos(\theta/2) + \sin(\theta/2)(n_x i + n_y j + n_z k)
\end{equation}

Rotating vector $\mathbf{v}$: $\mathbf{v}' = q \mathbf{v} q^{-1}$.\margincomp{More efficient than $3 \times 3$ matrices}

The Pauli spin matrices are quaternion units in disguise.\marginphysics{Electron spin → Quaternionic structure}

\subsection{Octonions $\mathbb{O}$ (8D)}

Octonions are eight-dimensional with basis $\{1, e_1, \ldots, e_7\}$. The seven imaginary units multiply via the \textbf{Fano plane}.\marginmath{Fano plane: 7 points, 7 lines}

\textbf{Non-associativity}: $(e_1 e_2) e_4 \neq e_1 (e_2 e_4)$.\margincaution{Associativity lost at 8D}

Using the Fano plane:\marginex{Products from Fano geometry}
\begin{align}
  (e_1 e_2) e_4 &= e_3 e_4 = e_6 \\
  e_1 (e_2 e_4) &= e_1 e_7 = -e_5
\end{align}

Since $e_6 \neq -e_5$, associativity fails.\marginmath{Grouping matters: $(xy)z \neq x(yz)$}

Octonions satisfy the weaker \textbf{Moufang identities}: $x(xy) = (xx)y$ and $(yx)x = y(xx)$.\marginmath{Alternative algebra: Moufang identities}

\textbf{Hurwitz theorem} (1898): $\mathbb{R}, \mathbb{C}, \mathbb{H}, \mathbb{O}$ are the \emph{only} normed division algebras.\marginhistory{Hurwitz: Only 1, 2, 4, 8 dimensions}

\textbf{Physical significance}:\marginphysics{Octonions → Exceptional symmetries}
\begin{itemize}
  \item Automorphism group is $G_2$ (smallest exceptional Lie group)
  \item Appear in $E_8 \times E_8$ heterotic string theory
  \item $\text{Spin}(8)$ triality connects vector/spinor representations
\end{itemize}

\section{Beyond Division Algebras}

\subsection{Sedenions $\mathbb{S}$ (16D)}

Sedenions contain \textbf{zero divisors}: non-zero $a, b$ with $ab = 0$.\margincaution{Zero divisors at 16D}

\textbf{Physical interpretation}: Zero divisors correspond to topological defects:\marginphysics{$ab = 0$ → Topological defects}
\begin{itemize}
  \item Cosmic strings (line defects)
  \item Monopoles (point defects)
  \item Domain walls (surface defects)
\end{itemize}

Properties lost:\marginmath{Non-alternative, not a division algebra}
\begin{itemize}
  \item Non-alternative
  \item Not a division algebra
  \item Contains zero divisors
\end{itemize}

\subsection{Pathions $\mathbb{P}$ (32D)}

Pathions (32D) connect to supersymmetry:\marginphysics{32D → Maximal supersymmetry}
\begin{itemize}
  \item $\mathcal{N}=8$ supergravity has 32 supercharges
  \item $E_8 \times E_8$ heterotic strings (rank 16 + 16 = 32)
\end{itemize}

\section{Systematic Loss of Structure}

\begin{table}[htb]
\centering
\caption{Properties of Cayley-Dickson algebras}
\begin{tabular}{lcccccc}
\toprule
Algebra & Dim & Commutative & Associative & Alternative & Division & Normed \\
\midrule
$\mathbb{R}$ & 1 & \checkmark & \checkmark & \checkmark & \checkmark & \checkmark \\
$\mathbb{C}$ & 2 & \checkmark & \checkmark & \checkmark & \checkmark & \checkmark \\
$\mathbb{H}$ & 4 & \texttimes & \checkmark & \checkmark & \checkmark & \checkmark \\
$\mathbb{O}$ & 8 & \texttimes & \texttimes & \checkmark & \checkmark & \checkmark \\
$\mathbb{S}$ & 16 & \texttimes & \texttimes & \texttimes & \texttimes & semi \\
$\mathbb{P}$ & 32 & \texttimes & \texttimes & \texttimes & \texttimes & semi \\
\bottomrule
\end{tabular}
\end{table}

\marginhistory{Frobenius (1878): Only $\mathbb{R}, \mathbb{C}, \mathbb{H}$ are associative division algebras}

\section{Connections to Exceptional Lie Groups}

The Cayley-Dickson algebras connect to exceptional Lie groups:

\subsection{$G_2$: Octonion Automorphisms}

$G_2$ is the automorphism group of octonions:\marginmath{$G_2 = \text{Aut}(\mathbb{O})$}
\begin{equation}
  G_2 = \{ g \in \text{GL}(7,\mathbb{R}) \mid g(xy) = g(x)g(y) \}
\end{equation}

Dimension: 14 (as a Lie group)\marginphysics{$G_2$ in M-theory compactifications}

\subsection{$E_8$: Hierarchical Structure}

The exceptional groups form a chain:\marginmath{$E_8 \supset E_7 \supset E_6 \supset F_4 \supset G_2$}
\begin{equation}
  E_8 \supset E_7 \supset E_6 \supset F_4 \supset G_2
\end{equation}

This parallels the Cayley-Dickson doubling hierarchy.\marginphysics{Heterotic strings: $E_8 \times E_8$}

\textbf{String theory}: $E_8 \times E_8$ heterotic strings arise from:
\begin{equation}
  T^{16} = \Lambda_{E_8} \oplus \Lambda_{E_8}
\end{equation}

\section{Summary}

We constructed the Cayley-Dickson tower from real numbers to high dimensions:

\textbf{Key results}:\marginxref{Next chapter: Exceptional Lie groups}
\begin{enumerate}
  \item Recursive doubling: $(a,b)(c,d) = (ac - d^*b, da + bc^*)$
  \item Classical division algebras: $\mathbb{R}, \mathbb{C}, \mathbb{H}, \mathbb{O}$ only
  \item Progressive structure loss at each doubling
  \item Connections to exceptional groups $G_2, F_4, E_6, E_7, E_8$
\end{enumerate}

\textbf{Forward bridge}: Chapter~\ref{ch:p2:exceptional-lie-groups} develops exceptional Lie groups in detail.

%==============================================================================
% END OF CHAPTER 1
%==============================================================================
